%\documentstyle[a4,11pt,ral]{article}
%\documentstyle[a4,ral]{article}
\documentclass[a4paper]{article}
%\usepackage{ral}

%  repdefs.sty
%  report style file like that for CGT book on Trust Region Methods
%  this version: 4 4 1999

% Formatting instructions

\textwidth  13.2cm
\textheight 21.5cm
\oddsidemargin -0.2mm                      % -4.2 mm
\evensidemargin 2.8cm
\topmargin -8.4mm                          % 2.64 cm

\def\baselinestretch{1.2}

\renewcommand{\textfraction}{0.05}

% Unit length for figures

\setlength{\unitlength}{1mm}

% Nice graphics for a horizontal line

\newcommand{\divider}{%
  \addtolength{\fboxsep}{-0.5mm}
  \centerline{\fbox{\rule{131mm}{0.25mm}}}
  \addtolength{\fboxsep}{0.5mm}
}

%  Modifications to standard book style to get correct spacings
%  in tables of contents as well as headers for chapter/part pages

% Float instructions

\setcounter{topnumber}{10}
\newcommand{\defaulttopfraction}{1.0}
\renewcommand{\topfraction}{\defaulttopfraction}
\setcounter{bottomnumber}{10}
\newcommand{\defaultbottomfraction}{1.0}
\renewcommand{\bottomfraction}{\defaultbottomfraction}

% TEMPORARY PAGE NUMBERING (by chapter)

%\newcommand{\pagenumbers}{\arabic{chapter}-\arabic{page}}
%\newcommand{\lchapter}[1]{\chapter{#1} \setcounter{page}{1}}
%\newcommand{\lpart}[1]{\setcounter{page}{1} \part{#1}}

% sequential page numbering

\newcommand{\pagenumbers}{\arabic{page}}
\newcommand{\lchapter}[1]{\chapter{#1}}
\newcommand{\lpart}[1]{\part{#1}}

\renewcommand{\thepage}{\pagenumbers}

% Section numbering

\setcounter{secnumdepth}{4}
\setcounter{tocdepth}{2}

% Equations, tables and figures numbered by chapter and section

\renewcommand{\theequation}{\arabic{section}.\arabic{equation}}
\renewcommand{\thetable}{\arabic{section}.\arabic{table}}
\renewcommand{\thefigure}{\arabic{section}.\arabic{figure}}

% Section headers with labels and counters reset to zero
% syntax: \lsection{label}, \lsubsection{label}, \lsubsubsection{label}

\newcommand{\lsection}[1]{\section{#1} \setcounter{equation}{0} \label{#1}}
\newcommand{\lsubsection}[1]{\subsection{#1} \label{#1}}
\newcommand{\lsubsubsection}[1]{\subsubsection{#1} \label{#1}}
\newcommand{\numsection}[1]{\section{#1}\setcounter{equation}{0}}
\newcommand{\bracketsection}[1]{\section{[#1]}\setcounter{equation}{0}}

% Begin and end equation with a label
% syntax: \beqn{label} ... \eeqn

\newcommand{\beqn}[1]{\begin{equation}\label{#1}}
\newcommand{\eeqn}{\end{equation}}

% Equation with and without a label
% syntax: \eqn{label}{equation}
%         \disp{label}{equation}

\newcommand{\eqn}[2]{\begin{equation}\label{#1}{#2}\end{equation}}
\newcommand{\disp}[1]{\[{#1}\]}

% Matrix with and without round brackets
% syntax: \mat{column format}{rows}
%         \arr{column format}{rows}

\newcommand{\vect}[1]{\left(\begin{array}{c}#1\end{array}\right)}
\newcommand{\mat}[2]{\left(\begin{array}{#1}#2\end{array}\right)}
\newcommand{\matinv}[2]{\left(\begin{array}{#1}#2\end{array}\right)^{-1}}
\newcommand{\arr}[2]{\begin{array}{#1}#2\end{array}}

% Put a reference with brackets
% syntax: \req{reference}

\newcommand{\req}[1]{(\ref{#1})}
\newcommand{\reqa}[1]{(\ref{#1}a)}
\newcommand{\reqb}[1]{(\ref{#1}b)}
\newcommand{\reqc}[1]{(\ref{#1}c)}
\newcommand{\reqd}[1]{(\ref{#1}d)}
\newcommand{\reqe}[1]{(\ref{#1}e)}

% References with page number

%\newcommand{\distref}[1]{, page~\pageref{#1}}
\newcommand{\distref}[1]{ [p.~\pageref{#1}]}
\newcommand{\preq}[1]{\req{#1}\distref{#1}}
\newcommand{\pref}[1]{\ref{#1}\distref{#1}}
\newcommand{\prea}[1]{\rea{#1}\distref{ass-#1}}

\newcommand{\ms}{\;\;\;\;}

\newcommand{\bctable}[1]{\begin{table}[htbp]
                         \begin{center}
                         \begin{tabular}{#1} }
\newcommand{\ectable}[1]{\end{tabular}
                         \caption{#1}
                         \end{center}
                         \end{table}}

\newcommand{\bcftable}[1]{\begin{figure}[htbp]
                          \begin{center}
                          \begin{tabular}{#1} }
\newcommand{\ecftable}[1]{\end{tabular}
                          \caption{#1}
                          \end{center}
                          \end{figure} }

\newcommand{\tim}[1]{\;\; \mbox{#1} \;\;}

% Theorem-like environments

\newtheorem{theorem}{Theorem}[section]
\newtheorem{lemma}[theorem]{Lemma}
\newtheorem{proposition}[theorem]{Proposition}
\newtheorem{corollary}[theorem]{Corollary}
\newtheorem{algorithm}{Algorithm}[section]
\newtheorem{definition}{Definition}
\newtheorem{example}{Example}[section]
\renewcommand{\theexample}{\arabic{section}.\arabic{example}}
\newcommand{\lex}[2]{\vspace{\baselineskip}
\noindent\framebox[\textwidth]{\parbox{0.95\textwidth}{
\begin{example} \label{#1} \rm #2 \end{example} } } \vspace{\baselineskip} }

% Enumeration indices

\renewcommand{\theenumi}{\arabic{enumi}}
\renewcommand{\labelenumi}{\theenumi.}
\renewcommand{\theenumii}{\theenumi\alph{enumii}}
\renewcommand{\labelenumii}{\theenumii.}
\renewcommand{\theenumiii}{\roman{enumiii}}
\renewcommand{\labelenumiii}{\theenumiii.}
\renewcommand{\theenumiv}{\theenumii(\roman{enumiii}\alph{enumiv})}
\renewcommand{\labelenumiv}{\theenumiv.}

% Calligraphic letters
% syntax: \cal[A-Z,a-z]

\newcommand{\calA}{{\cal A}} \newcommand{\calB}{{\cal B}}
\newcommand{\calC}{{\cal C}} \newcommand{\calD}{{\cal D}}
\newcommand{\calE}{{\cal E}} \newcommand{\calF}{{\cal F}}
\newcommand{\calG}{{\cal G}} \newcommand{\calH}{{\cal H}}
\newcommand{\calI}{{\cal I}} \newcommand{\calJ}{{\cal J}}
\newcommand{\calK}{{\cal K}} \newcommand{\calL}{{\cal L}}
\newcommand{\calM}{{\cal M}} \newcommand{\calN}{{\cal N}}
\newcommand{\calO}{{\cal O}} \newcommand{\calP}{{\cal P}}
\newcommand{\calQ}{{\cal Q}} \newcommand{\calR}{{\cal R}}
\newcommand{\calS}{{\cal S}} \newcommand{\calT}{{\cal T}}
\newcommand{\calU}{{\cal U}} \newcommand{\calV}{{\cal V}}
\newcommand{\calW}{{\cal W}} \newcommand{\calX}{{\cal X}}
\newcommand{\calY}{{\cal Y}} \newcommand{\calZ}{{\cal Z}}

\newcommand{\blambdastar}{\stackrel{*}{\blambda}}
\newcommand{\bfxstar}{\stackrel{*}{\bfx}}
\newcommand{\bfrstar}{\stackrel{*}{\bfr}}

% Overlined math
% syntax: \bar[A-Z,a-z]

\newcommand{\bara}{\overline{a}} \newcommand{\barb}{\overline{b}}
\newcommand{\barc}{\overline{c}} \newcommand{\bard}{\overline{d}}
\newcommand{\bare}{\overline{e}} \newcommand{\barf}{\overline{f}}
\newcommand{\barg}{\overline{g}} \newcommand{\barh}{\overline{h}}
\newcommand{\bari}{\overline{\imath}} \newcommand{\barj}{\overline{\jmath}}
\newcommand{\bark}{\overline{k}} \newcommand{\barl}{\overline{l}}
\newcommand{\barm}{\overline{m}} \newcommand{\barn}{\overline{n}}
\newcommand{\baro}{\overline{o}} \newcommand{\barp}{\overline{p}}
\newcommand{\barq}{\overline{q}} \newcommand{\barr}{\overline{r}}
\newcommand{\bars}{\overline{s}} \newcommand{\bart}{\overline{t}}
\newcommand{\baru}{\overline{u}} \newcommand{\barv}{\overline{v}}
\newcommand{\barw}{\overline{w}} \newcommand{\barx}{\overline{x}}
\newcommand{\bary}{\overline{y}} \newcommand{\barz}{\overline{z}}

\newcommand{\barA}{\overline{A}} \newcommand{\barB}{\overline{B}}
\newcommand{\barC}{\overline{C}} \newcommand{\barD}{\overline{D}}
\newcommand{\barE}{\overline{E}} \newcommand{\barF}{\overline{F}}
\newcommand{\barG}{\overline{G}} \newcommand{\barH}{\overline{H}}
\newcommand{\barI}{\overline{I}} \newcommand{\barJ}{\overline{J}}
\newcommand{\barK}{\overline{K}} \newcommand{\barL}{\overline{L}}
\newcommand{\barM}{\overline{M}} \newcommand{\barN}{\overline{N}}
\newcommand{\barO}{\overline{O}} \newcommand{\barP}{\overline{P}}
\newcommand{\barQ}{\overline{Q}} \newcommand{\barR}{\overline{R}}
\newcommand{\barS}{\overline{S}} \newcommand{\barT}{\overline{T}}
\newcommand{\barU}{\overline{U}} \newcommand{\barV}{\overline{V}}
\newcommand{\barW}{\overline{W}} \newcommand{\barX}{\overline{X}}
\newcommand{\barY}{\overline{Y}} \newcommand{\barZ}{\overline{Z}}

\newcommand{\gambar}{\overline{\gamma}}

% Hatted math
% syntax: \hat[A-Z,a-z]

\newcommand{\hata}{\hat{a}} \newcommand{\hatb}{\hat{b}}
\newcommand{\hatc}{\hat{c}} \newcommand{\hatd}{\hat{d}}
\newcommand{\hate}{\hat{e}} \newcommand{\hatf}{\hat{f}}
\newcommand{\hatg}{\hat{g}} \newcommand{\hath}{\hat{h}}
\newcommand{\hati}{\hat{\imath}} \newcommand{\hatj}{\hat{\jmath}}
\newcommand{\hatk}{\hat{k}} \newcommand{\hatl}{\hat{l}}
\newcommand{\hatm}{\hat{m}} \newcommand{\hatn}{\hat{n}}
\newcommand{\hato}{\hat{o}} \newcommand{\hatp}{\hat{p}}
\newcommand{\hatq}{\hat{q}} \newcommand{\hatr}{\hat{r}}
\newcommand{\hats}{\hat{s}} \newcommand{\hatt}{\hat{t}}
\newcommand{\hatu}{\hat{u}} \newcommand{\hatv}{\hat{v}}
\newcommand{\hatw}{\hat{w}} \newcommand{\hatx}{\hat{x}}
\newcommand{\haty}{\hat{y}} \newcommand{\hatz}{\hat{z}}

\newcommand{\hatA}{\hat{A}} \newcommand{\hatB}{\hat{B}}
\newcommand{\hatC}{\hat{C}} \newcommand{\hatD}{\hat{D}}
\newcommand{\hatE}{\hat{E}} \newcommand{\hatF}{\hat{F}}
\newcommand{\hatG}{\hat{G}} \newcommand{\hatH}{\widehat{H}}
\newcommand{\hatI}{\hat{I}} \newcommand{\hatJ}{\hat{J}}
\newcommand{\hatK}{\hat{K}} \newcommand{\hatL}{\hat{L}}
\newcommand{\hatM}{\hat{M}} \newcommand{\hatN}{\hat{N}}
\newcommand{\hatO}{\hat{O}} \newcommand{\hatP}{\hat{P}}
\newcommand{\hatQ}{\hat{Q}} \newcommand{\hatR}{\hat{R}}
\newcommand{\hatS}{\hat{S}} \newcommand{\hatT}{\hat{T}}
\newcommand{\hatU}{\hat{U}} \newcommand{\hatV}{\hat{V}}
\newcommand{\hatW}{\hat{W}} \newcommand{\hatX}{\hat{X}}
\newcommand{\hatY}{\hat{Y}} \newcommand{\hatZ}{\hat{Z}}

% Common mathematical symbols

\renewcommand{\Re}{\hbox{I\hskip -1.5pt R}}
\newcommand{\R}[1]{\Re^{{#1}}}
\newcommand{\Na}{\hbox{I\hskip -1.5pt N}}
\newcommand{\smallRe}
     {\mbox{\raisebox{-0.8pt}{\footnotesize I\hskip -1.5pt R\hskip -0.5pt}}}
\newcommand{\tinyRe}
     {\mbox{\raisebox{-0.8pt}{\scriptsize I\hskip -1.5pt R\hskip -0.5pt}}}

% Fractions

\newcommand{\sfrac}[2]{{\scriptstyle \frac{#1}{#2}}}

\newcommand{\eighth}{\sfrac{1}{8}}
\newcommand{\sixth}{\sfrac{1}{6}}
\newcommand{\quarter}{\sfrac{1}{4}}
\newcommand{\fourth}{\sfrac{1}{4}}
\newcommand{\third}{\sfrac{1}{3}}
\newcommand{\half}{\sfrac{1}{2}}
\newcommand{\halfT}{\sfrac{T}{2}}
\newcommand{\threequarters}{\sfrac{3}{4}}
\newcommand{\twothirds}{\sfrac{2}{3}}
\newcommand{\threehalves}{\sfrac{3}{2}}
\newcommand{\fivehalves}{\sfrac{5}{2}}
\newcommand{\fivesixths}{\sfrac{5}{6}}
\newcommand{\twelth}{\sfrac{1}{12}}
\newcommand{\twentyforth}{\sfrac{1}{24}}
\newcommand{\hundreth}{\sfrac{1}{100}}

\newcommand{\bigfrac}[2]{\displaystyle\frac{#1}{#2}}

% Indexing utilities

\newcommand{\bold}[1]{{\bf{#1}}}
\newcommand{\ind}[1]{\index[default]{#1}}
\newcommand{\indb}[1]{\index[default]{#1|bold}}
\newcommand{\notindex}[1]{\index[notation]{#1}}
\newcommand{\autindex}[1]{\index[authors]{#1}}
%\newcommand{\indi}[1]{\index[default]{#1|textit}}
%\newcommand{\sect}[1]{#1\raisebox{1pt}{$\scriptstyle+$}}
%\newcommand{\sect}[1]{#1ff}
%\newcommand{\sect}[1]{\raisebox{0.5pt}{[}#1\raisebox{0.5pt}{]}}
\newcommand{\sect}[1]{\underline{#1}}
\newcommand{\indi}[1]{\index[default]{#1|sect}}

% Assumptions and their cross-references

\newcommand{\ass}[2]{\label{ass-#1}
                     \begin{list}{}{\setlength{\leftmargin}{2 cm}}
                     \item \hspace{-1.9cm} \framebox[1.5cm]{\bf #1} #2
                     \end{list} }
\newcommand{\assd}[1]{\noindent
                      \framebox[1.5cm]{\bf #1} [p.~\pageref{ass-#1}]:}
\newcommand{\rea}[1]{#1}

% Historical notes

\newcommand{\notes}[1]
  {\section*{Notes and References for Section \thesection}{\small #1}}
\newcommand{\subnotes}[1]
  {\subsection*{Notes and References for Subsection \thesubsection}{\small #1}}
\newcommand{\subsubnotes}[1]
  {\subsection*{Notes and References for Subsection \thesubsubsection}
  {\small #1}}

% Machine epsilon

\newcommand{\epsmach}{\epsilon_{\mbox{\tiny M}}}

% Misc other utilities

\newcommand{\sx}{s}
\newcommand{\sy}{s^y}
\newcommand{\sn}{s^n}
\newcommand{\snk}{s^n_k}
\newcommand{\sny}{s^n_y}
\newcommand{\sz}{\scriptsize}
\newcommand{\itt}[1]{\item[\tt #1]}
\newcommand{\eqdef}{\stackrel{\rm def}{=}}
\newcommand{\blackbox}{\vrule height 5pt width 5pt depth 0pt}
\newcommand{\bigprod}{\displaystyle \prod}
\newcommand{\bigsum}{\displaystyle \sum}
\newcommand{\bigmin}{\displaystyle \min}
\newcommand{\bigmax}{\displaystyle \max}
\newcommand{\biglim}{\displaystyle \lim}
\newcommand{\bigsup}{\displaystyle \sup}
\newcommand{\spanset}[1]{\mbox{span}\left\{ #1 \right\}}
\newcommand{\smallspanset}[1]{\mbox{\small span}\left\{ #1 \right\}}
\newcommand{\ip}[2]{\langle #1, #2 \rangle}
\newcommand{\bip}[2]{\left\langle #1, #2 \right\rangle}
\newcommand{\cvect}[1]{\left( \begin{array}{c} #1 \end{array} \right) }
\newcommand{\lvect}[1]{\left( \begin{array}{l} #1 \end{array} \right) }
\newcommand{\rvect}[1]{\left( \begin{array}{r} #1 \end{array} \right) }
\newcommand{\matr}[2]{\left( \begin{array}{#1} #2 \end{array} \right) }
\newcommand{\deter}[2]{\left| \begin{array}{#1} #2 \end{array} \right| }
\newcommand{\tr}{{\rm tr}}
\newcommand{\mod}{{\rm mod}}
\newcommand{\sgn}{{\rm sgn}}
\newcommand{\sign}{{\rm sign}}
\newcommand{\diag}{{\rm diag}}
\newcommand{\slimk}[1]{\lim_{\stackrel{k \rightarrow \infty}{k \in #1}}}
\newcommand{\limk}{\lim_{k \rightarrow \infty}}
\newcommand{\biglimk}{\displaystyle \lim_{k \rightarrow \infty}}
\newcommand{\kap}[1]{\kappa_{\mbox{\tiny #1}}}
\newcommand{\ul}[1]{\underline{#1}}
\newcommand{\s}[1]{^{\mbox{\protect\tiny #1}}}
\newcommand{\sub}[1]{_{\mbox{\protect\tiny #1}}}
\newcommand{\ii}[1]{\{1, \ldots, #1 \}}
\newcommand{\deltaf}[1]{\delta \! f_{#1}}
\newcommand{\deltam}[1]{\delta m_{#1}}
\newcommand{\ri}[1]{{\rm ri}\{#1\}}
\newcommand{\cl}[1]{{\rm cl}\{#1\}}
\newcommand{\di}[1]{{\rm dist}(#1)}
\newcommand{\dis}[2]{{\rm dist}_{{#1}}(#2)}
\newcommand{\inte}[1]{{\rm int}\{#1\}}
\newcommand{\halfmu}{\frac{1}{2 \mu}}
\newcommand{\oneovermu}{\frac{1}{\mu}}
\newcommand{\nusup}{\nu_{\mbox{\tiny max}}}
\newcommand{\nuinf}{\nu_{\mbox{\tiny min}}}
\newcommand{\Delmax}{\Delta_{\mbox{\tiny max}}}
\newcommand{\pr}{^{\prime}}
\newcommand{\co}{\mbox{co}}
\newcommand{\st}{\,\mid\,}
\newcommand{\triple}[3]{{#1},{#2},{#3}}
\newcommand{\dual}{_{\mbox{\tiny D}}}
\newcommand{\bigleft}[1]{\left#1\arr{c}{\\}\!\!\!\!\!\!}
\newcommand{\bst}{\,\bigleft | \right.}
\newcommand{\E}{_{\calE}}
\newcommand{\I}{_{\calI}}
\newcommand{\Ip}{^{\calI +}}
\renewcommand{\Im}{^{\calI -}}
\newcommand{\ex}{\epsilon_x}
\newcommand{\ey}{\epsilon_y}
\newcommand{\e}{\epsilon}
\newcommand{\yc}{y\s{CS}}
\newcommand{\soc}{s\s{CS}}
\newcommand{\tx}{t}
\newcommand{\tendsup}{\nearrow}
\newcommand{\tendsdown}{\searrow}
\newcommand{\xl}{x_{\ell}}
\newcommand{\betamax}{\beta_{\scriptscriptstyle \max}}
\newcommand{\sigmamax}{\sigma_{\scriptscriptstyle \max}}
\newcommand{\cmax}{c_{\scriptscriptstyle \max}}
\newcommand{\fmin}{f_{\scriptscriptstyle \min}}
\newcommand{\nc}{n\s{C}}
\newcommand{\px}{p}
\newcommand{\floor}[1]{\lfloor #1 \rfloor}
\newcommand{\substackrel}[2]{_{\stackrel{\scriptstyle #1}{#2}}}
\newcommand{\su}[1]{_{\mbox{\raisebox{-1.5pt}{$\scriptstyle #1$}}}}
\newcommand{\suk}{\su{k}}
\newcommand{\midleft}[1]{\left#1\rule[-0.12cm]{0cm}{0.24cm}}



\renewcommand{\theequation}{\arabic{section}.\arabic{equation}}
\renewcommand{\thefigure}{\arabic{section}.\arabic{figure}}

\title{The SIF Reference Report (revised version)}
\author{Andrew R. Conn, Nicholas I. M. Gould, \\ Dominique Orban
and Philippe L. Toint}

%
%   The SIF reference report
%

\def\baselinestretch{1.0}
%\input{/usr2/nimg/sp/definitions}

\newcommand{\bdmath}{\begin{displaymath}}
\newcommand{\edmath}{\end{displaymath}}

\usepackage{hyperref}
\hypersetup{colorlinks=true,linkcolor=red,citecolor=green,pdfborder={0 0 0}}

\begin{document}

\bibliographystyle{plain}

\maketitle

\begin{abstract}
In this report, we provide a definitive description of the standard
input format.
It is intended to be used primarily as a reference document.
\end{abstract}

%\begin{center}
%{\bf UPDATES IN BOLD TYPEFACE}
%\end{center}

{\small
\tableofcontents
}

\section{\label{S1.1}Introduction}
\setcounter{figure}{0}

The mathematical   modelling of many real-world  applications involves
the  minimization or maximization of  a function of unknown parameters
or   variables.   Frequently   these  parameters   have known  bounds;
sometimes there are more general relationships between the parameters.
When the number of variables is  modest,  say up  to ten, the input of
such  a   problem to  an  optimization procedure  is    usually fairly
straightforward. Unfortunately many application areas  now require the
solution of optimization problems with thousands of variables; in this
case merely the input of the problem data  is extremely time-consuming
and prone to error. Moreover,  the mathematical programming  community
is only now designing algorithms for solving problems of this scale.

The format described in this report
was motivated directly by the  difficulties the authors were
experiencing entering test examples to the {\sf  LANCELOT} large-scale
nonlinear optimization  package.  It soon    became apparent  that  if
others  were to  be  encouraged to carry  out similar  tests  and even
enticed to use our software, the process of specifying problems had to
be considerably simplified. Thus we were inevitably drawn to provide a
preliminary   version of what   is described here:
a standard input format  (SIF) for nonlinear programming
problems, together with an appropriate translator from the  input file
to  the form required  by the  authors'  minimization software.  While
understandably reflecting   our views  and   experience, the   present
proposal is intended to be broadly applicable.

During the subsequent (and successive)  stages of development of these
preliminary ideas,  various important considerations  were  discussed.
These strongly influenced the present proposal.

\begin{itemize}
\item
There are many  reasons for  proposing  a standard input  format.  The
most obvious  one  is  the increased consistency    in  coding
nonlinear programming problems, and the resulting improvement  in code
reliability. As every problem is treated in a similar and standardized
way, it is more difficult to  overlook  certain aspects of the problem
definition.  The  provision of a SIF  file for a given  problem also
allows some  elementary (and very  often helpful) automatic  error and
consistency checking.


\item
A  further advantage of   having a standard input format   is the long
awaited  possibility  of having  a  portable  testbed   of  meaningful
problems.  Moreover,  such a testbed that  can  be  expected  to grow.  The
authors  soon experienced  the daunting   difficulties associated with
specifying large  scale  problems ---  not  only the   difficulty  of
writing down  the specification correctly but also  the  actual coding
(and frequent  re-coding) of a particular problem  which often results
in non-trivial differences between the initial and  final data.  These
differences could be a   major obstacle to  valid comparisons  between
competing optimization codes.  By contrast, having a SIF file allows
simple and unambiguous data transfer via  diskette, tape or electronic
mail. The success of the NETLIB and Harwell/Boeing problem collections
for linear  programming and sparse linear algebra   \cite{Gay85,DuffGrimLewi89}
  is a  good   recommendation  for  such
flexibility.   The  formality  required   by  the  SIF   approach may
admittedly appear formidable  for very simple  problems,  but is  soon
repaid when dealing with more complex ones.

\item
Of course, the SIF format should cover a large  part of the practical
optimization  problems  that users  may   want  to specify.   Explicit
provision should be made not only for unconstrained problems, but also
for constraints of different types  and complexity: simple  bounds  on
the variables,
linear  and/or  nonlinear  equations and  inequalities
should be handled without trouble. Special structure
of the problem at
hand is  also a  mandatory  part of  an  SIF  file. For example,  the
structure   of  least-squares
problems must  be   described  in  an
exploitable   form.   Sparsity
of  relevant  matrices  and   partial separability
of involved  nonlinear functions must be  included in the
standard problem   description   when  they  are known.   Finally, the
special case of systems of nonlinear equations should also be covered.

\item
The  existence and worldwide success of  the MPS standard input format
for linear programming  must be considered as  a {\em de facto} basis
for any attempt to define an  SIF for  nonlinear problems. The number
of  problems already  available  in this format   is  large,  and many
nonlinear problems arise as a refinement of existing linear ones whose
linear part and sparsity
structure are expected to be described in the
MPS
format. It therefore seems reasonable to require that an  SIF for
nonlinear programming problems  should conform to  the  MPS format. We
were thus led to choose a  standard that  corresponds to MPS,
augmented with  additional  constructs and  structures,  thus allowing
nonlinearity, and the general features that we wished, to be described
properly.


\item
The requirement of compatibility with the MPS format  has a  number of
consequences, not all of which are pleasant. The first one is that the
new SIF must  be based on  fixed format
for the SIF file.  Indeed, blanks
are  significant characters in MPS,
when they appear in the right data fields,
and  cannot be used as general  separators for free format
input in any compatible  system.  The second one  is the {\em a
priori}  existence of  a ``style'' for keywords  and overall layout of
the problem  description, a format which is  not always ideally suited
to  nonlinear problems.   Our   present proposal  accepts  these
limitations.

\item
The SIF should not be dependent on  a specific operating system and/or
manufacturer. In this respect, it must avoid relying on tools that may
be excellent but are too specific (yacc and lex, for example). This of
course does not prevent any implementation of an SIF interpreter using
whatever facilities are locally available.

\item
In principle,  the SIF should  not be dependent on  a particular high
level programming language. However,   as the intention  is that  SIF
files may  be converted into  executable programs, restrictions on the
symbolic  names   allowed  by   different  programming   languages may
influence the choice of names  within the problem  description itself.
For  instance, in Fortran,
symbolic   variables  may only  contain up  to   six characters from a
restricted set.  We  have chosen  to base the present proposal,  where
necessary, on Fortran
as  this appears to  be the most restrictive  of the more popular high
level languages.   This  dependence has  been isolated  as    much  as
possible.
\end{itemize}

The   authors are very  well  aware of the  shortcomings  of the  SIF
approach when compared to more elaborate modelling languages
(see, for
example, GAMS \cite{GAMS88},
AMPL \cite{FourGayKern87},
and OMP \cite{OMP87}.
These probably remain the best way to allow easy  and error free input
of large  problems. However, we contend   that there is  at present no
language  in   the public   domain which   satisfactorily handles  the
nonlinear aspects  of mathematical  programming  problems.   While the
advent  of  a  tool of   this   nature  is  very much   hoped  for, it
nevertheless seems necessary  to provide something like  the SIF now.
This  (we  hope,  intermediate)  step  is    indeed crucial  for   the
development   and comparison  of  algorithms for   solving large scale
nonlinear problems,   without  which a  more   elaborate tool would be
meaningless   anyway.  The SIF   for nonlinear problems   may also be
considered as a  first attempt to specify  the minimal structures
that
should be present in a true modelling  language
for such problems. It
is also of interest to develop a relatively simple input format, given
that   researchers  developing new  optimization   methods may have to
implement  their own code  for translating the  SIF file into  a form
suitable for their algorithms.  At this level, some compromise between
completeness and simplicity  seems necessary.  Finally,  the existence
and availability of modelling languages
for linear programming
for  a number of years has not yet made the MPS
format irrelevant.

Hence,  the reader  should   be aware   that what  sometimes appear as
unnecessarily restrictive  ``features'' of  the proposed  standard are
often direct consequences of the considerations outlined above.


In the next section, we explain how we propose to exploit the
structure in problems of the form
\req{objective}--\req{inequality_constraints} . We do this
both in  general and for a number of examples. Details of
the  way such  structure
may be expressed in  a standard data input
format  follow  in Section~3. The  input of nonlinear  information for
element  and group  functions
is  covered  in Section~4 and Section~5
respectively.  The formats proposed  in Sections~3--5 are quite rigid.
A  more flexible,  free-form,
input is considered  in  Section~6.  The relationship to existing work
is presented in Section~7 and conclusions drawn in Section~7.

\section[An introduction to nonlinear optimization
problem structure]{\label{intro_struct}An introduction to nonlinear
optimization  \protect\\ problem structure}
\setcounter{figure}{0}


As we have already mentioned, structure
is an integral and significant
aspect   of large-scale problems.   Structure  is  often  equated with
sparsity; indeed the   two are  closely linked when   the  problem  is
linear.    However,  sparsity
is  not the  most  important  phenomenon associated with  a  nonlinear
function; that role   is played   by invariant  subspaces.   The  {\em
invariant subspace}
of a  function $f(x)$ is the
set of all vectors $w$ for which  $f(x  + w) =  f(x)$ for all possible
vectors  $x$.  This  phenomenon  encompasses  function  sparsity.
For instance, the function
\[f(x_1 , x_2 ,\cdots,x_{1000}) = x^{2}_{500}\]
has a gradient
and Hessian
matrix  each   with a  single nonzero,   has an  invariant subspace of
dimension 999, and is, by  almost any  criterion, sparse.  However the
function
\[f(x_1 , x_2 ,\cdots,x_{1000}) = (x_1 + \cdots + x_{1000})^2\]
has a  completely   dense Hessian
matrix  but  still has  an invariant subspace
of dimension 999,  the set of  all vectors orthogonal to a  vector  of
ones.   The  importance  of  invariant  subspaces   is that  nonlinear
information is not  required for a  function in this subspace.  We are
particularly interested in functions which have large (as a percentage
of the overall number of variables) invariant subspaces.
This   allows  for efficient   storage and   calculation of derivative
information.  The   penalty  is,  of   course, the   need  to  provide
information about the subspace to an optimization procedure.

A  particular objective function
$F(x)$ is  unlikely  to have  a large
invariant subspace itself.
However, many  reasonably behaved functions may  be expressed as a sum
of {\em element} functions,
each of which does have a large invariant  subspace.
This is certainly true  if the function is sufficiently differentiable
and   has  a sparse  Hessian
matrix   \cite{GrieToin82a}.    Thus, rather  than   storing a
function as  itself, it pays  to store it as the  sum of its elements.
The elemental representation of a particular function  is  by no means
unique and there  may be  specific reasons for  selecting a particular
representation.  Specifying Hessian sparsity
is also supported in the  present proposal, but we  believe that it is
more efficient and also much easier to specify the invariant subspaces
directly.

{\sf LANCELOT} considers the problem of minimizing or maximizing
an objective function
of the form
\beqn{objective}
F( x )
= \sum_{i \in I_O}g_i \left (
\sum_{j \in J_i} w_{i,j} f_j (\bar{x}_j) +  a_i^T x  - b_i  \right)
+ \half \sum_{j=1}^n \sum_{k=1}^n h_{j,k} x_j x_k ,
\;
\eeqn
where $x = (x_1, x_2,\cdots, x_n)$, within the ``box'' region
\beqn{box}
l_i \leq x_i \leq u_i, \ms l \leq i \leq n
\eeqn
(where either bound on each variable may be  infinite),  and where the
variables are required to satisfy the extra conditions
\beqn{equality_constraints}
g_i \left(
\sum_{j \in J_{i}} w_{i,j}f_j (\bar{x}_j) + a_i^T x - b_i  \right) = 0
\; (i \in I_E)
\eeqn
and
\beqn{inequality_constraints}
0 \left \{ \begin{array}{ll}
           \leq \\ \geq
           \end{array} \right\}
g_i \left(
\sum_{j \in J_{i}} w_{i,j} f_j(\bar{x}_j) + a_i^T x - b_i \right)
\left\{ \begin{array}{ll}
\leq \\ \geq
\end{array} \right\}
 r_i,
\; (i \in I_I)
\eeqn
for some index  sets  $I_0, I_E$  and $I_I$   and  (possibly infinite)
values $r_i$.  The univariate functions $g_i$ are known as  {\em group
functions}.
The argument
\[ \sum_{j \in J_i} w_{i,j} f_j(\bar{x}_j) + a_i^T x - b_i \]
is known as  the  $i$-th {\em  group.}
The functions $f_j, j=1,\cdots,
n_e$, are  called   {\em  nonlinear} element functions.
 They are  functions of the  problem  variables
$\bar{x}_j$, where  the
$\bar{x}_j$ are either small subsets of  $x$ or such  that $f_j$ has a
large invariant   subspace
for  some other  reason.  The  constants  $w_{i,j}$ are  known as {\em
weights}, while
the function $a_i^T  x -  b_i$  is known  as the {\em linear}
element for the $i$-th group.
%{\bf New}
The additional term $\half \sum_{j=1}^n \sum_{k=1}^n h_{j,k} x_j x_k$
in the objective function is the {\em quadratic objective} group; the
leading $\half$ is there by convention.

It    is    more  common    to     call   the   group    functions
in \req{equality_constraints}  equality  constraint functions,
those   in
\req{inequality_constraints} inequality constraint functions
and the sum of those
in \req{objective} the objective function.

When   stating a  structured  nonlinear optimization  problem    of  the  form
\req{objective}--\req{inequality_constraints}, we  need to  specify  the group
functions,
linear and nonlinear elements
and the way that they all fit together.

\subsection{\label{S1.2}Problem, Elemental and Internal Variables}

A nonlinear element function $f_j$
is assumed  to be a function of the
problem variables $\bar{x}_j$, a  subset of the overall variables $x$.
Suppose that $\bar{x}_j$ has  $n_j$ components.  Then one can consider
the nonlinear element function
to be of the {\em structural} form $f_j
(v_1,\cdots,v_{n_j})$, where   we assign  $v_1  = \bar{x}_{j1},\cdots,
v_{n_j} =  \bar{x}_{jn_{j}}$.  The {\em  elemental} variables
for the element  function $f_j$ are  the variables  $v$  and, while we
need to associate  the particular values  $\bar{x}_j$ with $v$,  it is
the elemental variables which are important in defining  the character
of the nonlinear element functions.

As an example,  the first nonlinear element  function
for a particular problem might be
\beqn{1.2.1}
(x_{29} + x_3 - 2x_{17})e^{x_{29} - x_{17}}
\eeqn
which has the structural form
\beqn{1.2.2}
f_1 (v_1, v_2, v_3) = (v_1 + v_2 - 2v_3) e^{v_{1} - v_{3}},
\eeqn
where we need to assign $v_1 = x_{29}, v_2 =  x_3$ and $v_3 = x_{17}$.
For this example, there are three elemental variables.

The example may be used to illustrate a  further point. Although $f_1$
is a function of  three variables, the  function itself is really only
composed of {\em two} independent parts; the product of $v_1 + v_2 - 2
v_3$ with $e^{v_1 - v_3}$, or, if we write $u_1 = v_1  +  v_2 - 2 v_3$
and $u_2 =  v_1 - v_3$,   the product  of  $u_1$  with $e^{u_2}$.  The
variables $u_1$ and $u_2$ are known  as {\em  internal} variables
for   the   element  function.   They  are obtained   as  {\em  linear
combinations}  of the elemental  variables.   The important feature as
far as an optimization procedure is concerned  is that  each nonlinear
function  involves  as  few internal  variables  as possible,  as this
allows  for   compact    storage   and  more  efficient     derivative
approximation.

It frequently happens,  however, that a function does  not have useful
internal variables.
For instance, another element function
might have structural form
\beqn{1.2.3}
f_2 (v_1, v_2) = v_1 \sin v_2,
\eeqn
where for example  $v_1 =  x_6$  and  $v_2 =  x_{12}$.  Here, we  have
broken $f_2$ down  into as few  pieces as possible. Although there are
internal variables, $u_1 = v_1$ and $u_2 =  v_2$, they are the same in
this case  as   the  elemental variables
and  there is  no virtue  in
exploiting  them.  Moreover  it  can happen that   although there  are
special   internal  variables,
there  are  just as   many  internal as
elemental   variables  and it therefore   doesn't particularly help to
exploit them.  For instance, if
\beqn{1.2.4}
f_3 (v_1, v_2) = (v_1 + v_2) \log (v_1 - v_2),
\eeqn
where, for example,  $v_1 = x_{12}$ and $v_2  = x_2$, the function can
be formed as $u_1 \log u_2$, where $u_1 = v_1 + v_2$  and $u_2  = v_1-
v_2$. But as there are just as many internal  variables as elementals,
it will    not  normally  be   advantageous   to use  this    internal
representation.
Finally,  although   an element  function
may have  useful  internal
variables, the user may decide not to exploit  them.  The optimization
procedure  should still work  but at the expense of  extra storage
and computational effort.

In general, there will be a linear transformation
from the elemental variables to the internal ones. For example in
\req{1.2.2}, we have
\beqn{1.2.5}
\left( \begin{array}{c}
                         u_1 \\ u_2
                        \end{array} \right) =
                 \left( \begin{array}{rrr}
                         1 &  1 &  -2 \\
                         1 &  0 &  -1
                        \end{array} \right)
                 \left( \begin{array}{c}
                         v_1 \\ v_2\\ v_3
                        \end{array} \right)
\eeqn
while in \req{1.2.3}, we have
\beqn{1.2.6}
\left( \begin{array}{c}
                          u_1 \\ u_2
                        \end{array} \right) =
                 \left( \begin{array}{rr} 1 & 0 \\
                                          0 & 1
                         \end{array} \right)
                 \left( \begin{array}{c} v_1 \\ v_2
                 \end{array} \right)
\eeqn
In general the transformation will be of the form
\beqn{1.2.7}
u = Wv
\eeqn
and  this transformation is  {\em useful} if the  matrix $W$ has fewer
rows than columns.

\subsection{\label{S1.3}Element and Group Types}

It is quite  common  for   large nonlinear programming  problems to be
defined in terms of many nonlinear elements.
It is   also  common  that these   elements, although using  different
problem variables,  are structurally  the  same  as each  other.   For
instance, the function
\beqn{1.3.1}
\sum_{i = 1}^{n-1} (x_i x_{i + 1})^i
\eeqn
naturally decomposes into the sum of $n-1$ group functions,
$\alpha, \alpha^2, \cdots , \alpha^{n-1}$.
Each group
is a  nonlinear  element  function
$v_1 v_2$   of the  two  elemental variables
$v_1$  and $v_  2$ evaluated for  different pairs of problem
variables.
More commonly, the elements  may be arranged  into  a  few
classes; the  elements  within each class are  structurally  the same.
For example, the function
\beqn{1.3.2}
\sum_{i = 1}^{n-1} (x_i x_{i + 1} + x_1/x_i)^i
\eeqn
naturally decomposes into the sum of the same $n-1$ group functions.
Each group is the sum  of two nonlinear elements $v_1  v_2$ (where $v_1
= x_i$ and  $v_2 = x_{i+1}$) and $v_1 / v_2$ (where $v_1 = x_1$ and  $v_2 =
 x_{i}$).
A further common occurrence is the presence of elements which have the
same structure,
but which differ in using different problem variables
{\em and other auxiliary parameters}.
For instance, the function
\beqn{1.3.3}
\sum_{i = 1}^{n-1} (ix_i x_{i + 1})^i
\end{equation}
naturally decomposes into the sum of the same $n-1$ group functions.
Each  group is a  nonlinear  element
$p_1   v_1 v_2$  of the   single  parameter
$p_1$   and two elemental variables
$v_1$  and  $v_2$ evaluated  for  different  values of  the
parameter and pairs of problem variables.
Any two elements which are structurally the same are said to be of the
same {\em type.}
Thus  examples \req{1.3.1} and  \req{1.3.3} use a single element type,
where  as  \req{1.3.2} uses two   types.  When  defining the data  for
problems of the form \req{objective}--\req{inequality_constraints}, it
is unnecessary to define each nonlinear element in detail. All that is
actually needed is to specify the characteristics of the element types
and then to  identify each $f_j$  by its  type and the indices  of its
problem variables and (possibly) auxiliary parameters.

The same principal may be applied to group functions.
For example, the group functions that make up
\beqn{1.3.4}
\sum_{i = 1}^{n-1} (x_i x_{i + 1})^2
\eeqn
have different arguments but are structurally all the same, each being
of the form $g_i ( \alpha  ) = \alpha^2$.  As a  slightly more general
example, the group functions for
\beqn{1.3.5}
\sum_{i = 1}^{n-1} i(x_i x_{i + 1})^2
\eeqn
have different  arguments and  depend  upon  different  values   of  a
parameter
but are  still structurally all  the same, each being of the
form $g ( \alpha ) = p_1 \alpha^2$ for some  parameter $p_1$.  Any two
group functions which are structurally the same are  said to be of the
same  {\em type;} the  structural function is known  as the {\em group
type}
and its argument is  the {\em group-type variable.}
Once  again,  using group types  makes   the   task of  specifying the
characteristics  of  individual  group functions
more straightforward.
The group  type $g(\alpha ) = \alpha$  is  known as  the {\em trivial}
type.
Trivial groups  occur very frequently and  are considered to be
the default type.
It  is then only  necessary to  specify  non-trivial group types.

\subsection{\label{S1.4}An Example}

We now consider the small example problem,
\[
{\rm minimize} \;\;\; F(x_1, x_2,x_3)
\equiv x_1^2 + x_2^4 x_3^4 + x_2 \sin(x_1 + x_3) + x_1 x_3 + x_2
\]
subject to the bounds
$- 1 \leq x_2 \leq 1$ and $1 \leq x_3 \leq 2$.
There are a number of ways of casting this problem in the
form \req{objective}. Here, we consider partitioning $F$
into groups
as

\[ \begin{array}{ccccc}
(x_1)^2 & + & (x_2 x_3)^4 & + & (x_2 \sin(x_1 + x_3) + x_1 x_3 + x_2) \\
\uparrow  & & \uparrow & & \uparrow \\
\mbox{group 1} & & \mbox{group 2} & & \mbox{group 3}
\end{array} \]

Notice the following:
\begin{enumerate}
\item
group 1 uses the non-trivial group
function $g_1 (\alpha)  = \alpha^
2$.  The group  contains a  single {\em linear}  element;  the element
function is $x_1$.

\item
group 2 uses   the   non-trivial group  function
$g_2(\alpha)  =
\alpha^4$.  The group contains  a single {\em nonlinear} element; this
element function is  $x_2 x_3$.   The element  function has  {\em two}
elemental variables,
$v_1$ and $v_2$, say,  (with $v_1 = x_2$ and $v_2
= x_3$) but there is no useful transformation to internal variables.

\item
group 3 uses the trivial group function
$g_3 (\alpha) = \alpha$.  The
group contains two {\em nonlinear} elements and a  single {\em linear}
element $x_2$.  The first nonlinear element function is $x_2 \sin ( x_1
+ x_3 )$.  This  function has {\em  three} elemental variables, $v_1$,
$v_2$ and $v_3$,  say, (with $v_1 =  x_2 $, $v_2 =  x_1$  and  $v_3  =
x_3)$, but may be expressed in terms of {\em two} internal variables
$u_1$ and $u_2$, say,  where $u_1 = v_1 $  and $u_2= v_2  + v_3$.  The
second nonlinear element  function is $x_1  x_3$,  which has two
elemental variables $v_1$ and $v_2$ (with $v_1 = x_1$ and $v_2 =  x_3$)
and is  of the same type as the nonlinear element in group 2.
\end{enumerate}

Thus we see that we can consider our objective  function
to be  made up of three  groups;
the first and  second are non-trivial
(and  of different types) so we  will have to provide our optimization
procedure with function and derivative
values  for these at  some stage.  There
are  three nonlinear elements,
one from group two and  two more from  group three.
Again  this means
that  we   shall have to provide  function   and derivative
values for
these.  The first and  third nonlinear  element
are of  the same type,
while the  second  element is a  different type.  Finally one of these
element types, the second, has  a useful transformation from elemental
to internal variables so this transformation will need to be set up.

\subsection{\label{S1.5}A Second Example}

We   now  consider a different   sort   of example, the  unconstrained
problem,
\beqn{1.5.1}
{\rm minimize} \ms  F(x_1,\cdots,x_{1000})
\equiv \sum_{i = 1}^{999} \sin(x_i^2 + x_{1000}^2 + x_1 - 1)
        + \half \sin(x_{1000}^2).
\eeqn
Once again, there are a number of ways of  casting this problem in the
form \req{objective}, but the most natural is to consider the argument
of each sine function as a group --- the group  function
is  then $g_i
(\alpha ) =  p_1 \sin \alpha  $, $1 \leq  i \leq   1000$,  for various
values of  the parameter
$p_1$.  Each  group but  the   last  has two
nonlinear elements, $x_{1000}^2$ and $x_i^2$ $1 \leq i \leq 999$ and a
single linear element $x_1 - 1$.  The last has no linear element and a
single  nonlinear  element, $x_{1000}^2$.     A single   element type,
$v_1^2$, of the elemental variable,
$v_1$, covers all of the nonlinear elements.

Thus we see that we can consider our  objective function
to be made up of 1000 nontrivial  groups, all of  the same type,
so we
will have to  provide  our optimization  procedure  with function  and
derivative values
for these at some stage.  There  are 1999 nonlinear elements,
two from
each group
except the last,  but all of the  same  type and again we shall have to
provide function and derivative
values for these.  As there is so  much  structure
to this problem, it
would  be    inefficient   to  pass   the  data   group-by-group   and
element-by-element.   Clearly,  one   would  like to    specify   such
repetitious structures using a convenient shorthand.

\subsection{\label{S1.6}A Final Example}

As a third example, consider the constrained problem in the variables
$x_1 ,\cdots$, $x_{100}$ and y
\beqn{1.6.1}
{\rm minimize} \ms \half ((x_1 - x_{100}) x_2 + y)^2 +
2 x_1^2 + 2 x_1^{ } x_{100}^{ }
\eeqn
subject to the constraints
\beqn{1.6.2}
x_1 x_{i + 1} + (1 +
{\scriptstyle \frac{2}{i}}) x_i x_{100} + y \leq 0 \ms (1 \leq i \leq 99),
\eeqn
\beqn{1.6.3}
0 \leq (\sin x_i)^2 \leq \half \ms (1 \leq i \leq 100),
\eeqn
\beqn{1.6.4}
(x_1 + x_{100} )^2 = 1
\eeqn
and the simple bounds
\beqn{1.6.5}
-1 \leq x_i \leq i \ms (1 \leq i \leq 100).
\eeqn
As before, there are a number of  ways of casting  this problem in the
form   \req{objective}--\req{inequality_constraints}.   We   chose  to
decompose the problem as follows:
\begin{enumerate}
\item
the objective function comprises two groups, the first of which
uses the  non-trivial group function
$g (\alpha ) = \half \alpha^2$.  This  group contains a  single {\em
linear} element;
the element function is y. There  is also a nonlinear element
$( x_1 -  x_{100} ) x_2 $. This   element  function has  three
elemental variables, $v_1$, $v_2$ and $v_3$, say (with  $v_1  = x_1$,
$v_2  = x_{100}$ and $v_3 =  x_2$);  there  is a useful transformation
from elemental to internal variables of the form $u_1 = v_1 - v_2$ and
$u_2 = v_3$ and the  element function may then  be represented as $u_1
u_2$. 
%{\bf New} 
The second group may be considered as a quadratic
objective group, and written as
$\half( h_{1,1} x_1 x_1 + h_{1,100} x_1 x_{100}
 + h_{100,1} x_{100} x_1 )$,
where $h_{1,1} = 4$ and $h_{1,100} =$
$h_{100,1} = 2$.

\item
The next set of groups,
inequality  constraints,
$ x_1 x_{i+1} + (1 + 2/i) x_i x_{100} + y  \leq 0  \; {\rm  for} \; 1
\leq i \leq 99$ are of the form \req{inequality_constraints} with no lower
 bounds.
Each uses
the trivial group
function $g   ( \alpha )  = \alpha$  and contains a
single {\em linear} element,
$y$, and  two {\em nonlinear} elements
$x_i  x_{i+1}$ and $(1 + 2/i)
x_i x_{100}$.  Both nonlinear elements are of the  same type, $p_1 v_1
v_2$, for appropriate variables  $v_1$  and $v_2$  and parameter $p_1$,
and there is no useful transformation to internal variables.

\item
The following set of  groups,
again  inequality constraints,
$0 \leq (\sin{x_i})^2 \leq \half \;$ for $\; 1 \leq  i  \leq 100$, are of
the  form \req{inequality_constraints} with  both  lower  and   upper bounds.
Each uses the
non-trivial group  function
$g (\alpha  )  = \alpha^2$ and  contains a
single  {\em   nonlinear}  element
of   the type   $sin   v_1$  for an
appropriate variable $v_1$.   Notice  that  the group  types
for these
groups and for the  objective function group
are  both of the form $g( \alpha ) = p_1 \alpha^2$, for some parameter
$p_1$, and  it  may prove
more convenient to use this form to cover both sets of groups.

\item
The last group,
an equality constraint,
$(x_1 + x_{100} )^2 -  1 = 0$,
is of the  form  \req{equality_constraints}.
Again, this  group  uses  the trivial  group function
$g ( \alpha ) = \alpha$ and  contains  a single  {\em linear}
element,
$-1$, and a single {\em nonlinear} element
of the type $( v_1 + v_2 )^2$ for appropriate elemental variables
$v_1$ and  $v_2$.  Once
more, a single internal  variable,
$u_1 = v_1 +  v_2$ can be used  and
the element is then represented as $u_1^2$.
\end{enumerate}

Thus we see  that we can consider  our problem to be  made  up of  201
groups  of  two  different  types as well as an quadratic objective group
so  we   will  have to  provide  our
optimization procedure  with function and  derivative values
for  these at  some stage.   There are 200  nonlinear elements
of  four different types  and again this  means that we  shall have to
provide function and derivative
values for these.  As for the previous
example,  there is so  much structure
to this  problem that it would be  inefficient to pass the  data group-by-group
and  element-by-element.    Again, we will   introduce ways   to  specify this
repetitious structure using a convenient shorthand.

\section{\label{S2}The Standard Data Input Format}
\setcounter{figure}{0}

We now consider how to pass the data  for optimization  problems to
an  optimization procedure.  In  our  description, we will concentrate  on our
third example, as presented in Section~\ref{S1.6};  we will show how the input
file might be specified for this example to motivate the overall structure
of such  a file and
then follow this with the general syntax allowed.

The data which defines a particular problem is  written in a file in a
standard format. It is intended that this data  file is interpreted by
an appropriate decoding program and converted into a format useful for
input to an optimization package or program.  The  content of the file
is specified line by line. As our format is intended to  be compatible
with the MPS linear programming
format \cite{IBM69}, we preserve  the  MPS terminology and  call these
lines {\em cards.}

A SIF comprises one or more files.
The first of these files is known as the Standard {\em Data} Input
Format (SDIF).
As its name suggests,
data which describes how the parts of the optimization problem are
related, together with all fixed constants, are given in this file.
Indeed, a SIF for linear programming
problems can be completely specified by an MPS
file; the  SIF comprises  a single section,  the  SDIF file, and  that
section is merely the MPS file.

\subsection{\label{S2.1}Introduction to the Standard Data Input Format}

As we have just said,  the data  format  is designed to be  compatible
with the MPS
linear programming
format. There  are, however, extensions   to allow the  user to  input
nonlinear problems.  The user must prepare an input file consisting of
three types of cards:
\begin{itemize}
\item Indicator cards, which specify the type of data to follow.
\item Data cards, which contain the actual data.
\item Comment cards.
\end{itemize}

Indicator cards
contain  a simple keyword  to specify the type of data
that follows. The first character of  such  cards must be in column~1;
indicator  cards are the   only cards, with   the exception of comment
cards,
which start in column~1.  Possible indicator cards are given in
Table~\ref{Fig1}.
{\small
\bctable{lllc}
\multicolumn{1}{c}{Keyword} &
\multicolumn{1}{c}{Comments} &
\multicolumn{1}{c}{Presence} &
Described in \S \\
\hline
{\tt NAME}        &                          & mandatory & \ref{S2.2.1} \\
\cline{1-1} \cline{3-4}
                  &  \multicolumn{1}{c}{\rm either} &   &              \\
{\tt GROUPS}      &                          & mandatory & \ref{S2.2.6} \\
{\tt ROWS}        & synonym for {\tt GROUPS} &           & \ref{S2.2.6} \\
{\tt CONSTRAINTS} & synonym for {\tt GROUPS} &           & \ref{S2.2.6} \\
{\tt VARIABLES}   &                          & mandatory & \ref{S2.2.7} \\
{\tt COLUMNS}     & synonym for {\tt VARIABLES} &        & \ref{S2.2.7} \\
\cline{1-1} \cline{3-4}
                  & \multicolumn{1}{c}{\rm or}  &       &              \\
{\tt VARIABLES}   &                          & mandatory & \ref{S2.2.8} \\
{\tt COLUMNS}     & synonym for {\tt VARIABLES} &        & \ref{S2.2.8} \\
{\tt GROUPS}      &                          & mandatory & \ref{S2.2.9} \\
{\tt ROWS}        & synonym for {\tt GROUPS} &           & \ref{S2.2.9} \\
{\tt CONSTRAINTS} & synonym for {\tt GROUPS} &           & \ref{S2.2.9} \\
\hline
{\tt CONSTANTS}   &                          & optional  & \ref{S2.2.10} \\
{\tt RHS}         & synonym for {\tt CONSTANTS} &        & \ref{S2.2.10} \\
{\tt RHS'}        & synonym for {\tt CONSTANTS} &        & \ref{S2.2.10} \\
{\tt RANGES}      &                          & optional  & \ref{S2.2.11} \\
{\tt BOUNDS}      &                          & optional  & \ref{S2.2.12} \\
{\tt START POINT} &                          & optional  & \ref{S2.2.13} \\
{\tt QUADRATIC}   &                          & \bf optional  & \ref{S2.2.13a} \\
{\tt HESSIAN}     & \bf synonym for {\tt QUADRATIC} &        & \ref{S2.2.13a} \\
{\tt QUADS}       & \bf synonym for {\tt QUADRATIC} &        & \ref{S2.2.13a} \\
{\tt QUADOBJ}     & \bf synonym for {\tt QUADRATIC} &        & \ref{S2.2.13a} \\
{\tt QSECTION}    & \bf synonym for {\tt QUADRATIC} &        & \ref{S2.2.13a} \\
{\tt ELEMENT TYPE}&                          & optional  & \ref{S2.2.14} \\
{\tt ELEMENT USES}&                          & optional  & \ref{S2.2.15} \\
{\tt GROUP TYPE}  &                          & optional  & \ref{S2.2.16} \\
{\tt GROUP USES}  &                          & optional  & \ref{S2.2.17} \\
{\tt OBJECT BOUND}&                          & optional  & \ref{S2.2.18} \\
{\tt ENDATA}      &                          & mandatory & \ref{S2.2.2}  \\
\hline
\ectable{\label{Fig1}Possible indicator card}
}

Indicator cards
must appear in the  order shown,  except that the {\tt
GROUPS} and  {\tt VARIABLES}  sections  may be   interchanged to  allow
specification of the linear  terms by rows or  columns.
The cards {\tt CONSTANTS}, {\tt  RHS'}, {\tt RHS},  {\tt RANGES}, {\tt
BOUNDS}, {\tt START  POINT},
{\tt QUADRATIC}, {\tt HESSIAN}, {\tt QUADS}, {\tt QUADOBJ}, {\tt QSECTION},
{\tt ELEMENT  TYPE}, {\tt  ELEMENT USES},
{\tt  GROUP  TYPE},   {\tt GROUP  USES}   and {\tt  OBJECT  BOUND} are
optional.

The data cards are divided into six fields.
The content of each field varies with each type of data card
as
described in Section~\ref{S2.2}.  Those in fields~1, 2, 3 and 5 must
always be left justified within the field.  Field 1, which appears in
columns 2 and 3, may contain a {\em code}
(that is, a one or two
character string which defines the expected contents of the remaining
fields on the card),
fields 2, 3 and 5 may hold {\em names}
and fields
4 and 6 might store {\em numerical values}.
The numerical values are
defined by up to 12 characters which may include a decimal point and
an optional sign (a positive number is assumed unless a -- is given).
The value may be followed by a decimal exponent, written as an {\tt E}
or {\tt D},
followed by a signed or unsigned one or two digit integer;
the first blank
after the {\tt E} or {\tt D} terminates the field.
The names of variables, nonlinear elements
or groups
may be  up to ten characters  long.  These names may include
integer indices
(see Section~\ref{S2.1.1}).

Any card with  the  character * in  column   1 is a comment card;
the remaining contents of the card are ignored. Such a  card may appear
anywhere  in the data  file.  In addition,  completely blank
cards are
ignored when scanning the input file and may thus be used to space the
data.  Finally, the presence of a \$
as the first character in fields 3 or 5 of  a data card
indicates  that the  content  of  the remaining part  of the card is a
comment and will be ignored.

\subsubsection{\label{S2.1.1}Names}

One of the  positive features of the  MPS standard  is the  ability to
give  meaningful  names to problem constraints
and  variables.
As  our proposal  is intended to  be  MPS
compatible,   we too have  this  option.    However,  one of  the less
convenient features of  the MPS
standard  is the  cumbersome way that
repetitious  structure
is handled.   In particular,  the  name of each
variable
and  constraint
must be  defined  on a  separate  line,  and
structure within constraints  is effectively  ignored when setting  up
the  constraint matrix.   We  consider it important  to overcome  this
deficiency of the MPS
standard when formulating large-scale  examples.  One way  is to allow
variable, group
and  nonlinear element names
to have  indices and  to
have syntactic devices which enable  the user to define  many items at
once.

Unless otherwise indicated, we  allow any name which   uses up  to ten
valid characters. A {\em valid} character is any ASCII character
whose decimal  code lies in  the range  32  to  126 (binary 0100000 to
11111110, hex  20 to 7E) (see, for  instance \cite{DictComp83}).  This
includes lower and upper case roman alphabetic characters, the digits 0
to 9,  the blank
character   and  other mathematical  and  grammatical
symbols.  A name can be one of the following:
\begin{enumerate}
\item
a  scalar name
of the form
{\tt \pounds\pounds\pounds\pounds\pounds\pounds\pounds\pounds\pounds\pounds}
where each {\tt \pounds} is a valid character type excepting that the
first {\tt \pounds}  cannot be a {\tt \$}. A  completely blank
string is
also not allowed.  Futhermore, the strings {\tt 'SCALE'}, {\tt 'MARKER'},
{\tt 'DEFAULT'}, {\tt 'INTEGER'} and {\tt 'ZERO-ONE'}
are reserved for special operations.
\item
an array name  of  the form {\tt name(index)}, where  {\tt index} is a
list of integer index names,
{\tt name} is a list  of valid characters
(the first character may  not be a {\tt \$})  and the maximum possible
size of the {\em  expanded name} does  not exceed  ten characters. The
list of  index names  must be of the  form  {\tt list1, list2, list3},
where {\tt list1}, {\tt  list2} and {\tt  list3} are  predefined index
(parameter)
names (see   Section~\ref{S2.2.3},  below) and all  three
indices are optional. The indices are only allowed to  take on integer
values. Commas are  only required  as separators; the  presence  of an
open bracket ``{\tt (}'' announces  a list of  indices and  a  close bracket
``{\tt )}''  terminates the list. An array  {\em name} is {\em  expanded} as
{\tt namenumber1, number2, number3}, where {\tt numberi,  i = 1, 2, 3}
is the integer value allocated to the index {\tt listi} at the time of
use.
\end{enumerate}

As an example, the expanded form of the array name {\tt X(I,J,K)} when
{\tt I}, {\tt J} and {\tt K} have the  values 3, 4 and  6 respectively
would be {\tt X3,4,6}, while it would take the form {\tt  X-6,0,3} if
{\tt I}, {\tt J} and {\tt K} have the values -6, 0 and 3 respectively.
However, {\tt X(I,J,K)} could not be expanded if  {\tt I}, {\tt J} and
{\tt K} were each allowed to be as large as 100 as, for instance, {\tt
X100,100,100}    is over ten characters  long   and thus  not  a valid
expanded name.

An array
item may be referred to by either its array  name
(so long as  the index lists have  been specified) or by its  expanded
name.
Thus, if {\tt I}, {\tt J} and {\tt K} have been specified as 2,
7 and 9 respectively, {\tt X(I,J,K)} and {\tt X2,7,9} are identical.

If  two separators  (opening  or   closing brackets and  commas)   are
adjacent  in an array  name,
the intervening index   is deemed not  to
exist and is  ignored when the  name is expanded.
Thus, the expanded name of {\tt Y()}  is just {\tt Y},  while  that of
{\tt   Z(I,,K)} is {\tt  Z3,4}  if {\tt I}   is 3 and  {\tt K}  is  4.
Furthermore, any  name which   does not  include  the characters ``{\tt
(}'', ``{\tt )}'' or ``{\tt ,}'' may  be used as an array  name
and is its own expanded name.
Thus the name {\tt X} may be a scalar or array
name whereas {\tt W(} and {\tt V,} can only be scalar names.

We defer the definition of integer indices until Section~\ref{S2.2.3}.

Note that blanks
are considered to be significant characters.  Thus if
$_{-}$ denotes a blank, the names $_{-}${\tt x}  and {\tt x}$_{-}$ are
different. It  is recommended that all  names are  left-shifted within
their  relevant data fields
to avoid possible user-instigated name recognition errors.

\subsubsection{\label{S2.1.2}Fortran Names}

A  notable exception  to the above are Fortran  names. A {\em Fortran}
name takes the form of a sequence of one to six upper case  letters or
digits, the first  of which  must not be a  digit. These names are used in
Sections~\ref{S2.2.14}--\ref{S2.2.17}, \ref{S3.2.1}--\ref{S3.2.3} and
\ref{S4.2.1}.

\subsubsection{\label{S2.1.3}Numerical Values}

The  definition of   a specific problem  normally requires  the use of
numerical (integer or real) data values. Such values  can be specified
in  two  ways.  Firstly,  the values  may  simply occur  as integer or
floating-point numbers in data fields
4 and 6. Secondly, values may be
allocated to named  parameters,
known as  integer or real  parameters,
and a value subsequently used by reference to a particular integer or
real parameter name.  This second method may only  be used to allocate
values   on  certain cards;
when  this  facility  is used,  the first
character in field 1 on the  relevant  data card
will  be an {\tt X} or a {\tt Z}.
This latter approach is particularly useful  when a value is
to be used repeatedly or if a value is to be  changed within a do-loop
(see Section~\ref{S2.2.4}).

We defer the definition of integer and real parameters
until Section~\ref{S2.2.3}.

\subsubsection{\label{S2.1.4}An Example}

Before we  give the complete  syntax for   an  SDIF file,  we give  an
illustrative  example. In  order   to exhibit as  many constructs   as
possible,   we consider    how  we  might    encode  the example    in
Section~\ref{S1.6}.   We   urge the reader to   study this section  in
detail.   As  always, there  are  many possible  ways of specifying  a
particular problem; we give one  in Figures~\ref{Fig2} and \ref{Fig3},
pages~\pageref{Fig2} and \pageref{Fig3}.
The horizontal and vertical lines  are merely included to indicate the
extent of data fields. The  actual  widths of the fields
are given at the top  of the figure,  and the  column numbers given at
its foot.

The SDIF file naturally  divides into two  parts.   In the first part,
lines 2  to 39 of the figure, we specify  information regarding linear
functions used in the example.
In the second part,  lines 40 to 93, we specify nonlinear  information.
The first part  is merely an extension  of  the MPS
input format; the second part is new.

The file must always start with a {\tt NAME}
card, on which a name (in this case {\tt EG3}) for the example  may be
given (line 1), and must end with an {\tt ENDATA}
card (line  93).
A comment is inserted  at the end  of line 1  as to the  source of the
example.  The character {\tt \$} identifies  the remainder of the line
as a comment; the comment is ignored when interpreting the input file.

We next specify  names of parameters
which will occur   frequently in
spe\-cifying the example  (lines 2 to 5).  In our case the integer and
real parameters
{\tt 1} and {\tt ONE} are given along with  {\tt N}, a
problem dimension  --- here  {\tt N} is set to  100,  but it would  be
trivial to change  the example  in  6  to allow  variables   $ x_1
,\ldots, x_n$ for any $n$.  We make a comment  to this effect on
line 4; any card with the character {\tt *} in column  1  is a  comment card
and its content is ignored when interpreting the input file.

We now name the problem variables
and groups
(in our example objective function and constraints)
used. The groups may  be specified before or
after  the variables. We  choose here to name  the   groups first.
The  objective  function
will  be known as   {\tt OBJ} (line  7); the
character {\tt  N} in  field 1 specifies  that  this is  an  objective
function    group.
The  inequality constraints    \req{1.6.2}
and \req{1.6.3} are  named ${\tt  CONLE1},\ldots,  {\tt CONLE99}$  and
${\tt  CONGE1},\ldots,  {\tt  CONGE100}$  respectively.  Rather   than
specify them individually, a do-loop
is used to make an array
definition.  Thus the constraints
${\tt CONLE1,\ldots, CONLE99}$  are defined {\em en masse}
on lines 9  to  11 with  the do-loop  index  {\tt I}  running from the
previously defined  value $1$ to the  value  {\tt  NM1}.   The integer
parameter, {\tt  NM1},
is defined on line 8 to be the sum of {\tt  N} and the value
${\tt -1}$ and in our case will  be 99.   The characters  {\tt XL}
in field 1 of line 10 indicate that an  array
definition is being made
(the  ${\tt  X}$)    and  that the   groups
are less-than-or-equal-to
constraints
(the {\tt L}).  The do-loop introduced on  line 9 with the
characters {\tt DO}
in  field 1 is  terminated on  line 11  with  the
characters {\tt  ND}
in   its    first field.  In  a   similar  way,  the constraints ${\tt
CONGE1,\ldots, CONGE99}$ are defined all  together on lines 12 to  14;
that these constraints involve  bounds
on  both sides is taken care of
by considering  them to  be greater-than-or-equal-to constraints
({\tt XG})
on  line  13 and later  specifying the additional upper bounds in
the {\tt RANGES}
section  (lines 26 to  29).  Finally,
the equality constraint \req{1.6.4}
is to be  called {\tt CONEQ} (line 15); the character   {\tt E}
in   field  1 specifies that this  is  an equality constraint group.

Having named the groups,
we  next name the  problem variables.
At the
same time,  we include the  coefficients  of  all  the linear elements
used.  The variables are named  ${\tt X1},\ldots, {\tt X100}$ and {\tt
Y}; an array
declaration is made for the former set on  lines 17 to 19
and {\tt Y} is defined on line 20. The character {\tt X} in field 1 of
line 18 indicates that an array definition is used. Only the objective
function
\req{1.6.1}, inequality  constraint
\req{1.6.2} and equality
constraint  groups
\req{1.6.4}  contain  linear  elements.
As well as
introducing  {\tt Y}, line  20 also specifies  that the linear element
associated with group {\tt  OBJ} (field  3) involves variable  {\tt Y},
and that {\tt Y} 's coefficient in the linear element
is  {1.0} (field 4).  A do-loop
is now used in lines 21 to 23  to  show that the linear
elements  for constraints \req{1.6.2} also use  the  variable {\tt Y}.
It is assumed  that unless a variable is  explicitly identified with a
linear element,  the element is  independent of that variable.   Thus,
although  \req{1.6.4} uses  a linear element,  the element is constant
and need not be specified in the {\tt VARIABLES}
section.

The only remaining part of the linear elements
which must be specified
is the constant term. Again, only nonzero constants need be given. For
our  example, only  the  equality constraint
group
\req{1.6.4} has a
nonzero constant term and  this data is specified on  lines 24 and 25.
The string  {\tt C1} in  field 2 of  line 25 is  the   name given to a
specific set of constants. In general,  more  than one set of constants
may be specified in  the SDIF file and the  relevant one selected in a
postprocessing stage. Here, of course, we only have one set.

As   we have seen, the  inequality   constraint groups \req{1.6.3} are
bounded from above as well as from  below. In the {\tt RANGES}
section (lines 26 to  30)  we  specify  these  upper bounds (or  range
constraints  as  they are  sometimes  known).
The  numerical  values
$\half$ are specified for each bound  for the relevant groups
in an array
definition on line 28; the string {\tt R1} in field 2 is once
again a name
given to a specific set of range values as it is possible
to define more than one set in the {\tt RANGES} section.

We now turn to the simple  bounds
\req{1.6.5} which are  specified in
lines 30 to 36 of the example.   All problem variables  are assumed to
have  lower  bounds  of zero and   no upper  bounds  unless  otherwise
specified.  All but one  of the variables  for our  example have lower
bounds of $-1$. We  thus change the default
value for the value of the lower bound on line 31 - the set of bounds
is named {\tt BND1}.  The  character {\tt L}
specifies that  it is the lower  bound
default that  is to be changed.
The string {\tt 'DEFAULT'}
in field 3 indicates that the default is  being changed.  The variable
$x_i$ is given an upper bound
of $i$. We encode  that in a do-loop on
lines  32 to  35 of  the figure.  The  do-loop  index {\tt I}   is  an
integer.  We change its current value to a real on line 33  and assign
that value  as the upper bound
on  line 34.  The  character {\tt Z} in
field 1 of this line indicates that an  array
definition is being made
and that the data is taken from a parameter in field 5 (as  opposed to
a specified numerical   value in field   4) and the character  {\tt U}
specifies that the upper bound
value is to  be assigned.  The variable
{\tt y} is unbounded  or,  as   it  is   often known,  free.
This  is specified on line 36, the  string {\tt FR}
in field  1 indicating that {\tt Y} is free.

The final ``linear'' piece of information given is an estimate  of the
solution to the problem (if known) or  at  least a set of  values from
which to start a minimization  algorithm. This information is given on
lines  37  to  39. For  our problem,   we  choose  the  values  $x_i =
\half,  1  \leq  i  \leq  100$ and  $y   = 0$.  Unless otherwise
specified, all starting values take a default of zero.
We change that
default on line 38 to $\half$ --- the set of starting values are
named  {\tt START1} ---  and then specify the individual  value for the
variable {\tt Y} on line 39.

We now specify the nonlinear information. 
%{\bf New} 
Firstly, we recall that
there is a quadratic objective group,
$x_1^2 + 2 x_1^{ } x_{100}^{ } \equiv \half( 4 x_1 x_1 +
 2 x_1 x_{100} + 2 x_{100} x_1 )$. We need to specify the nonzero coefficients
of the terms $x_j x_k$, and in our cases these are
$h_{1,1} = 4$ and $h_{1,100} = h_{100,1} = 2$.
The rule that we adopt is
that there is no need to supply both nonzeros
$h_{j,k}$ and $h_{k,j}$ since they are the same, and
that one (whichever is unimportant) suffices. Thus $h_{1,1} = 4$
and we (arbitrarily) choose to give $h_{1,100} = 2$.
In the {\tt QUADRATIC} section on lines 39a to 39b, we indicate that
the quadratic objective has two terms involving
${x_1}$; the coefficient 4 is given for the
${x_1^2 \equiv x_1^{ } x_1^{ }}$ term, while that for
the ${x_1 x_{100}}$ term is assigned the value 2.

Next, we saw in Section~\ref{S1.6}
that there are four element types
for  the problem, being of  the form
(i) $(v_1 - v_2 ) v_3$, (ii) $p_1 v_1 v_2$, (iii) $\sin v_1$  and (iv)
$(v_1 + v_2  )^2$.  In  the {\tt ELEMENT  TYPE}
section on lines 40 to 48, we record details  of these types.  We name
the four  types (i)--(iv) {\tt 3PROD,  2PROD, SINE}  and  {\tt SQUARE}
respectively.   For  {\tt 3PROD},  we define  the elemental  variables
(lines 41  and 42) to be {\tt  V1, V2}  and {\tt V3}  and the internal
variables (line 43) to be {\tt U1} and  {\tt U2}.  Elemental variables
may be defined, two to a line, on lines for which field 1 is {\tt EV}.
Internal variables,
on the other hand, are defined on lines  with {\tt
IV}  in field 1.  Similar definitions  are  made for {\tt 2PROD} (line
44), {\tt SINE} (line 46)  and {\tt  SQUARE} (line  47). The type {\tt
2PROD} also makes use of a parameter $p_1$. This is named {\tt  P1}
on line 45 for which field 1 reads {\tt EP}.

Having   specified the  element  types,
we  next   specify individual
nonlinear elements
in the {\tt  ELEMENT USES}
section.  As we have seen, the objective function
group uses  a single
nonlinear element of type
{\tt 3PROD}. We name this particular element
{\tt OBJ1}.  On line 50,  the character {\tt  T}  in field 1 indicates
that the {\tt OBJ1} is of type  {\tt 3PROD}. The assignment of problem
to  elemental variables
is made on  lines  51 to 53. Problem variables
{\tt X1} and {\tt X2} are assigned to elemental variables {\tt V1} and
{\tt V3}; the assignment  is  indicated by the   character {\tt V}  in
field 1.  In order to assign $x_{100}$ (or in general $x_n$) to $v_2$,
we assign the array
entry {\tt X(N)}  to {\tt  V2}. Notice  that as an
array element is being used, this must be specially flagged  ({\tt ZV}
in field 1) as otherwise the wrong variable (called {\tt  X(N)} rather
than {\tt X100}, which is  the expanded form of  {\tt X(N)}) would be
assigned.   There are two   nonlinear  elements
for   each inequality
constraint group
\req{1.6.2}, each being of  the same type  {\tt 2PROD}. We  name these
elements
${\tt   CLEA1,\ldots,  CLEA99}$  and    ${\tt  CLEB1,\ldots,
CLEB99}$. The assignments are made on lines 54 to 67 within a do-loop.
On lines 56 and 60  the elements are named
and their  types assigned.

As array
assignments are being used,  field 1 for  both lines contains
the string {\tt XT}.
The elemental variables
are then associated with
problem variables
on lines 57--58 and 61--62 respectively. Again array
assignments are used and field 1 contains the string {\tt ZV}.
Notice  that   on  line 58 $v_2$   is  assigned  the  problem variable
$x_{i+1}$, where the  index {\tt IP1} is   defined as  the sum of  the
index {\tt I}  and  the integer  value 1  on line  55.  It remains  to
assign values    for the parameter $p_1$  for   each element.
This is
straightforward for the elements  ${\tt  CLEA1,\ldots, CLEA99}$ as the
required value is always 1 and the assignment is made on  line 59 on a
card
with first field {\tt XP}.
The  remaining elements  have varying parameter
values ${1 +  2/i}$.
This value is  calculated on lines  63 to 65 and  assigned on line 66.
Line 63 assigns  {\tt REALI} to have  the  floating point value of the
index {\tt I}.  This  new value is then divided  into the value  2  on
line 64 and the value assigned to {\tt ONE}  is added to the resulting
value on  the final line.  Thus  the parameter
{\tt 2OVAI+1} holds the
required value of $p_1$ and the  array
assignment is  made on line 66.
On this line the string {\tt ZP} indicates that an array
assignment is
being made, taking its value from the parameter  {\tt 2OVAI+1} in field
4 (the {\tt Z}) and that the elemental parameter
{\tt  P1} in field 3
is to  be assigned (the  {\tt  P}).  The  definition of the  nonlinear
elements for the remaining constraint
groups is straightforward. The inequality constraints
\req{1.6.3} each
use a single element, named
${\tt CGE1,\ldots,  CGE100}$, of type {\tt
SINE}  and the appropriate  array
assignments are made on  lines 68 to
70.  Finally, the equality constraint
\req{1.6.4} is  named {\tt CEQ1}
and typed {\tt SQUARE} with appropriate elemental variable
assignments on lines 72 to 74.

We next need to specify  the nontrivial group  types.
This is done in
the  {\tt  GROUP   TYPE}
section  on   lines   75   to 77.  We  saw in
Section~\ref{S1.6}    that  a    single nontrivial   group,  $p_1
\alpha^2$, is required. On  line 76, the  name {\tt PSQUARE} is given
for the type and the group type variable ${\tt \alpha}$ is named ${\tt
ALPHA.}$ The string {\tt GV}
pin field 1 indicates that a type  and its
variable are to be defined. On the following line field 1  is {\tt GP}
and this is used to announce that the group type parameter
$p_1$ is named {\tt P1}.

Finally, we need to allocate nonlinear elements
to groups
and specify
what type the resulting groups are to  be. This takes place within the
{\tt GROUP USES}
section which runs from line 78 to 90.  The objective
function group is nontrivial and its type is announced on line 79. The
group uses  the single nonlinear element
{\tt OBJ1} specified on line
80    and the  group-type   parameter
$p_1$  is   set   to the  value
$\half$ on  the next line. The  characters {\tt T},  {\tt E} and
{\tt  P}
in the  first  fields  of these three  cards
announce their
purposes.  The inequality groups  \req{1.6.2} each use   two nonlinear
elements,
but the groups themselves are trivial
(and  thus their types
do not have to be made  explicit). The assignment of  the  elements to
each group is made in an  array
definition on lines 82  to 84; line 83
is flagged  as assigning elements to a  group with the string {\tt XE}
in field 1. The  second set of  inequality constraints
\req{1.6.3} use
the nontrivial group type {\tt PSQUARE} with parameter  value 1.
Each group  uses a  single  nonlinear  element
and  the   appropriate array
assignments  are contained   on lines 85    to 89. Lastly  the trivial
equality  constraint   group  \req{1.6.4}  is assigned the   nonlinear
element {\tt CEQ1} on line 90.

The definition of the problem is now complete. However, it often helps
the intended minimization program if known  lower and  upper bounds on
the possible values  of the objective function
can  be given.   For  our example,  the objective function \req{1.6.1}
cannot be smaller than zero. This  data is  specified on lines  91 and
92.  The string {\tt LO}
in field  1 of line 92 indicates  that a lower bound
is known for the
value of \req{1.6.1}.  The string {\tt OBOUND} in field~2 of this line
is a name given to this known bound. The value of  the lower bound now
follows in field~4.  No upper bound need be specified as  the function
is initially assumed to lie between plus and minus infinity.

{\renewcommand{\arraystretch}{0.8}
{\small {\tt
\bcftable{r|@{}c@{}|@{}l@{}|@{}l@{}|@{}l@{}|@{}l@{}|@{}l@{}|@{}l@{}|@{}l@{}|}
\multicolumn{1}{@{}c@{}}{~}&
\multicolumn{1}{@{}c@{}}{~}&
\multicolumn{1}{@{}c@{}}{$<$$>$~}&
\multicolumn{1}{@{}c@{}}{$<$---10---$>$}&
\multicolumn{1}{@{}c@{}}{$<$---10---$>$}&
\multicolumn{1}{@{}c@{}}{$<$----12----$>$~~~}&
\multicolumn{1}{@{}c@{}}{$<$---10---$>$}&
\multicolumn{1}{@{}c@{}}{$<$----12----$>$}\\
\multicolumn{1}{@{}c@{}}{\sz line}&
\multicolumn{1}{@{}c@{}}{}&
\multicolumn{1}{@{}c@{}}{\rm F.1}&
\multicolumn{1}{@{}c@{}}{\rm Field 2}&
\multicolumn{1}{@{}c@{}}{\rm Field 3}&
\multicolumn{1}{@{}c@{}}{\rm Field 4}&
\multicolumn{1}{@{}c@{}}{\rm Field 5}&
\multicolumn{1}{@{}c@{}}{\rm Field 6}\\
\cline{2-8}
{\sz 1} &\multicolumn{3}{@{}l@{}|@{}}{NAME}&EG3&
&\multicolumn{2}{@{}l@{}|@{}}{\$ The example of \S 1.6 }\\
{\sz 2} &&IE &1         &         &1             &         &  \\
{\sz 3} &&IE &N         &         &100           &         &  \\
{\sz 4} &\multicolumn{7}{@{}l@{}|@{}}{*Variants of \S 1.6 obtained by
choice of N on previous card}\\
{\sz 5} &&RE &ONE       &         &1.0           &         &  \\
{\sz 6} &\multicolumn{3}{@{}l@{}|@{}}{GROUPS}&&&&\\
{\sz 7} &&N  &OBJ       &         &              &         &   \\
{\sz 8} &&IA &NM1       &N        &-1            &         &   \\
{\sz 9} &&DO &I         &1        &              &NM1      &   \\
{\sz 10}&&XL &CONLE(I)  &         &              &         &   \\
{\sz 11}&&ND &          &         &              &         &   \\
{\sz 12}&&DO &I         &1        &              &N        &   \\
{\sz 13}&&XG &CONGE(I)  &         &              &         &   \\
{\sz 14}&&ND &          &         &              &         &   \\
{\sz 15}&&E  &CONEQ     &         &              &         &   \\
{\sz 16}&\multicolumn{3}{@{}l@{}|@{}}{VARIABLES}&&&&\\
{\sz 17}&&DO &I         &1        &              &N        &   \\
{\sz 18}&&X  &X(I)      &         &              &         &   \\
{\sz 19}&&ND &          &         &              &         &   \\
{\sz 20}&&   &Y         &OBJ      &1.0           &         &   \\
{\sz 21}&&DO &I         &1        &              &NM1      &   \\
{\sz 22}&&X  &Y         &CONLE(I) &1.0           &         &   \\
{\sz 23}&&ND &          &         &              &         &   \\
{\sz 24}&\multicolumn{3}{@{}l@{}|@{}}{CONSTANTS}&&&&\\
{\sz 25}&&   &C1        &CONEQ    &1.0           &         &   \\
{\sz 26}&\multicolumn{3}{@{}l@{}|@{}}{RANGES}&&&&\\
{\sz 27}&&DO &I         &1        &              &NM1      &   \\
{\sz 28}&&X  &R1        &CONGE(I) &0.5           &         &   \\
{\sz 29}&&ND &          &         &              &         &   \\
{\sz 30}&\multicolumn{3}{@{}l@{}|@{}}{BOUNDS}&&&&\\
{\sz 31}&&LO &BND1      &'DEFAULT'&-1.0          &         &   \\
{\sz 32}&&DO &I         &1        &              &N        &   \\
{\sz 33}&&RI &REALI     &I        &              &         &   \\
{\sz 34}&&ZU &BND1      &X(I)     &              &REALI    &   \\
{\sz 35}&&ND &          &         &              &         &   \\
{\sz 36}&&FR &BND1      &Y        &              &         &   \\
{\sz 37}&\multicolumn{3}{@{}l@{}|@{}}{START POINT}&&&&\\
{\sz 38}&&   &START1    &'DEFAULT'&0.5           &         &   \\
{\sz 39}&&   &START1    &Y        &0.0           &         &   \\
{\sz 39a}&\multicolumn{3}{@{}l@{}|@{}}{QUADRATIC}&&&&\\
{\sz 39b}&&  &X1        &X1       &4.0           &X100     &2.0 \\
{\sz 40}&\multicolumn{3}{@{}l@{}|@{}}{ELEMENT TYPE}&&&&\\
{\sz 41}&&EV &3PROD     &V1       &              &V2       &   \\
{\sz 42}&&EV &3PROD     &V3       &              &         &   \\
{\sz 43}&&IV &3PROD     &U1       &              &U2       &   \\
{\sz 44}&&EV &2PROD     &V1       &              &V2       &   \\
{\sz 45}&&EP &2PROD     &P1       &              &         &   \\
{\sz 46}&&EV &SINE      &V1       &              &         &   \\
{\sz 47}&&EV &SQUARE    &V1       &              &V2       &   \\
{\sz 48}&&IV &SQUARE    &U1       &              &         &   \\
\cline{2-8}
\multicolumn{1}{@{}c@{}}{~}&
\multicolumn{1}{@{}c@{}}{$\uparrow$}&
\multicolumn{1}{@{}c@{}}{$\uparrow\uparrow\;$~}&
\multicolumn{1}{@{}c@{}}{$\uparrow\;$~~~~$\uparrow\;$~~$\;\uparrow$}&
\multicolumn{1}{@{}c@{}}{$\uparrow\;$~~~~~~~$\;\uparrow$}&
\multicolumn{1}{@{}l@{}}{$\uparrow\;$~~~~~~~~~$\;\uparrow$}&
\multicolumn{1}{@{}c@{}}{$\uparrow\;$~~~~~~~$\;\uparrow$}&
\multicolumn{1}{@{}c@{}}{$\uparrow\;$~~~~~~~~~$\;\uparrow$}\\
\multicolumn{1}{@{}c@{}}{~}&
\multicolumn{1}{@{}c@{}}{\sz 1}&
\multicolumn{1}{@{}c@{}}{{\sz 2~3~}~}&
\multicolumn{1}{@{}c@{}}{{\sz 5~}~~~~{\sz 10}~~{\sz 14}}&
\multicolumn{1}{@{}c@{}}{{\sz 15}~~~~~~~{\sz 24}}&
\multicolumn{1}{@{}l@{}}{{\sz 25}~~~~~~~~~{\sz 36}}&
\multicolumn{1}{@{}c@{}}{{\sz 40}~~~~~~~{\sz 49}}&
\multicolumn{1}{@{}c@{}}{{\sz 50}~~~~~~~~~{\sz 61}}\\
\ecftable{\label{Fig2}SDIF file (part 1) for the example of
Section~\protect\ref{S1.6}}
}}}

\clearpage

{\renewcommand{\arraystretch}{0.8}
{\small {\tt
\bcftable{r|@{}c@{}|@{}l@{}|@{}l@{}|@{}l@{}|@{}l@{}|@{}l@{}|@{}l@{}|@{}l@{}|}
\multicolumn{1}{@{}c@{}}{~}&
\multicolumn{1}{@{}c@{}}{~}&
\multicolumn{1}{@{}c@{}}{$<$$>$~}&
\multicolumn{1}{@{}c@{}}{$<$---10---$>$}&
\multicolumn{1}{@{}c@{}}{$<$---10---$>$}&
\multicolumn{1}{@{}c@{}}{$<$----12----$>$~~~}&
\multicolumn{1}{@{}c@{}}{$<$---10---$>$}&
\multicolumn{1}{@{}c@{}}{$<$----12----$>$}\\
\multicolumn{1}{@{}c@{}}{\sz line}&
\multicolumn{1}{@{}c@{}}{}&
\multicolumn{1}{@{}c@{}}{\rm F.1}&
\multicolumn{1}{@{}c@{}}{\rm Field 2}&
\multicolumn{1}{@{}c@{}}{\rm Field 3}&
\multicolumn{1}{@{}c@{}}{\rm Field 4}&
\multicolumn{1}{@{}c@{}}{\rm Field 5}&
\multicolumn{1}{@{}c@{}}{\rm Field 6}\\
\cline{2-8}
{\sz 49}&\multicolumn{3}{@{}l@{}|@{}}{ELEMENT USES}&&&&\\
{\sz 50}&&T  &OBJ1      &3PROD    &              &         &   \\
{\sz 51}&&V  &OBJ1      &V1       &              &X1       &   \\
{\sz 52}&&ZV &OBJ1      &V2       &              &X(N)     &   \\
{\sz 53}&&V  &OBJ1      &V3       &              &X2       &   \\
{\sz 54}&&DO &I         &1        &              &NM1      &   \\
{\sz 55}&&IA &IP1       &I        &1             &         &   \\
{\sz 56}&&XT &CLEA(I)   &2PROD    &              &         &   \\
{\sz 57}&&ZV &CLEA(I)   &V1       &              &X(1)     &   \\
{\sz 58}&&ZV &CLEA(I)   &V2       &              &X(IP1)   &   \\
{\sz 59}&&XP &CLEA(I)   &P1       &1.0           &         &   \\
{\sz 60}&&XT &CLEB(I)   &2PROD    &              &         &   \\
{\sz 61}&&ZV &CLEB(I)   &V1       &              &X(I)     &   \\
{\sz 62}&&ZV &CLEB(I)   &V2       &              &X(N)     &   \\
{\sz 63}&&RI &REALI     &I        &              &         &   \\
{\sz 64}&&RD &2OVERI    &REALI    &2.0           &         &   \\
{\sz 65}&&R+ &2OVAI+1   &2OVERI   &              &ONE      &   \\
{\sz 66}&&ZP &CLEB(I)   &P1       &              &2OVAI+1  &   \\
{\sz 67}&&ND &          &         &              &         &   \\
{\sz 68}&&DO &I         &1        &              &N        &   \\
{\sz 69}&&XT &CGE(I)    &SINE     &              &         &   \\
{\sz 70}&&ZV &CGE(I)    &V1       &              &X(I)     &   \\
{\sz 71}&&ND &          &         &              &         &   \\
{\sz 72}&&T  &CEQ1      &SQUARE   &              &         &   \\
{\sz 73}&&V  &CEQ1      &V1       &              &X1       &   \\
{\sz 74}&&ZV &CEQ1      &V2       &              &X(N)     &   \\
{\sz 75}&\multicolumn{3}{@{}l@{}|@{}}{GROUP TYPE}&&&&\\
{\sz 76}&&GV &PSQUARE   &ALPHA    &              &         &   \\
{\sz 77}&&GP &PSQUARE   &P1       &              &         &   \\
{\sz 78}&\multicolumn{3}{@{}l@{}|@{}}{GROUP USES}&&&&\\
{\sz 79}&&T  &OBJ       &PSQUARE  &              &         &   \\
{\sz 80}&&E  &OBJ       &OBJ1     &              &         &   \\
{\sz 81}&&P  &OBJ       &P1       &0.5           &         &   \\
{\sz 82}&&DO &I         &1        &              &NM1      &   \\
{\sz 83}&&XE &CONLE(I)  &CLEA(I)  &              &CLEB(I)  &   \\
{\sz 84}&&ND &          &         &              &         &   \\
{\sz 85}&&DO &I         &1        &              &N        &   \\
{\sz 86}&&XT &CONGE(I)  &PSQUARE  &              &         &   \\
{\sz 87}&&XE &CONGE(I)  &CGE(I)   &              &         &   \\
{\sz 88}&&XP &CONGE(I)  &P1       &1.0           &         &   \\
{\sz 89}&&ND &          &         &              &         &   \\
{\sz 90}&&E  &CONEQ     &CEQ1     &              &         &   \\
{\sz 91}&\multicolumn{3}{@{}l@{}|@{}}{OBJECT BOUND}&&&&\\
{\sz 92}&&LO &OBOUND    &         &0.0           &         &   \\
{\sz 93}&\multicolumn{3}{@{}l@{}|@{}}{ENDATA}&&&&\\
\cline{2-8}
\multicolumn{1}{@{}c@{}}{~}&
\multicolumn{1}{@{}c@{}}{$\uparrow$}&
\multicolumn{1}{@{}c@{}}{$\uparrow\uparrow\;$~}&
\multicolumn{1}{@{}c@{}}{$\uparrow\;$~~~~$\uparrow\;$~~~$\uparrow$}&
\multicolumn{1}{@{}c@{}}{$\uparrow\;$~~~~~~~$\;\uparrow$}&
\multicolumn{1}{@{}l@{}}{$\uparrow\;$~~~~~~~~~$\;\uparrow$}&
\multicolumn{1}{@{}c@{}}{$\uparrow\;$~~~~~~~$\;\uparrow$}&
\multicolumn{1}{@{}c@{}}{$\uparrow\;$~~~~~~~~~$\;\uparrow$}\\
\multicolumn{1}{@{}c@{}}{~}&
\multicolumn{1}{@{}c@{}}{\sz 1}&
\multicolumn{1}{@{}c@{}}{{\sz 2~3~}~}&
\multicolumn{1}{@{}c@{}}{{\sz 5~}~~~~{\sz 10}~~{\sz 14}}&
\multicolumn{1}{@{}c@{}}{{\sz 15}~~~~~~~{\sz 24}}&
\multicolumn{1}{@{}l@{}}{{\sz 25}~~~~~~~~~{\sz 36}}&
\multicolumn{1}{@{}c@{}}{{\sz 40}~~~~~~~{\sz 49}}&
\multicolumn{1}{@{}c@{}}{{\sz 50}~~~~~~~~~{\sz 61}}\\
\ecftable{\label{Fig3}SDIF file (part 2) for the example
         of Section~\protect\ref{S1.6}}
}}}

\subsection{\label{S2.2}Indicator and Data Cards}

 We now give details of the indicator cards and the  data  cards which
follow them.

\subsubsection{\label{S2.2.1}The {\tt NAME} Indicator Card}

The {\tt  NAME}  indicator card
is used  to announce the  start of the
input data for a particular  problem. The  user may specify a name for
the problem; this name is entered on the indicator card in field~3 and
may be at  most 8 characters long. The  syntax for the {\tt NAME} card
is given in Figure~\ref{F2.2.1}.

{\tt
\bcftable{|@{}l@{}|@{}l@{}|@{}l@{}|}
\multicolumn{1}{@{}c@{}}{~~~~~~~~~~~~~~}&
\multicolumn{1}{@{}c@{}}{$<$---8--$>$~~}&
\multicolumn{1}{@{}c@{}}{~~~~~~~~~~~~~~~~~~~~~~~~~~~~~~~~~~~~~}\\
\multicolumn{1}{@{}c@{}}{~}&
\multicolumn{1}{@{}c@{}}{\rm Field 3}&
\multicolumn{1}{@{}c@{}}{~}\\
\hline
NAME & prob-nam & \\
\hline
\multicolumn{1}{@{}c@{}}{~}&
\multicolumn{1}{@{}c@{}}{$\uparrow\;$~~~~~~~$\;\uparrow$}&
\multicolumn{1}{@{}c@{}}{~}\\
\multicolumn{1}{@{}c@{}}{~}&
\multicolumn{1}{@{}c@{}}{{\sz 15}~~~~~~~{\sz 24}}&
\multicolumn{1}{@{}c@{}}{~}
\ecftable{\label{F2.2.1}The indicator card {\tt NAME}}
}

\subsubsection{\label{S2.2.2}The {\tt ENDATA} Indicator Card}

The {\tt ENDATA}
indicator card simply announces the end of the input data.
The data for a particular problem, in the form of indicator and
data cards, must lie between a {\tt NAME} and an {\tt ENDATA} card.
The syntax for the {\tt ENDATA} card
is given in Figure~\ref{F2.2.2}.

{\tt
\bcftable{|@{}l@{}|}
\multicolumn{1}{@{}c@{}}{~~~~~~~~~~~~~~~~~~~~~~~~~~~~~~~~~~~~~~~~~~~~~~~
~~~~~~~~~~~~}\\
\hline
ENDATA \\
\hline
\ecftable{\label{F2.2.2}The indicator card {\tt ENDATA}}
}

\subsubsection{\label{S2.2.3}Integer and Real Parameters}

We shall use the  word {\em parameter}  to mean the name given  to any
quantity  which is associated with a  specified   numerical value. The
numerical value will be known  as the {\em  parameter} value.  Integer
and  real values may  be associated  with parameters  in two ways. The
easiest  way  is simply to set  a  parameter to a  specified parameter
value, or to obtain a parameter from a previously defined parameter by
simple  arithmetic   operations (addition, subtraction, multiplication
and division).  The second way is  to have a parameter value specified
in a do-loop,
or to obtain a parameter from one specified in a do-loop
(see Section~\ref{S2.2.4} below).

The syntax for associating a parameter
with a specific value is given in Figure~\ref{Fig4.3}.

The two character  string in data  field~1 (F.1) specifies  the way in
which  the parameter value is  to be assigned.  If  the first of these
characters   is a {\tt   I}, the  assigned  value  is  an integer; the
parameter will be referred  to as an  {\em integer parameter}  or {\em
integer index.} Alternatively, if the first of  these characters is an
{\tt R} or an {\tt A}, the assigned value  is a real and the parameter
will be called a {\em real parameter}.

{\renewcommand{\arraystretch}{0.8}
{\small {\tt
\bcftable{|@{}c@{}|@{}l@{}|@{}l@{}|@{}l@{}|@{}l@{}|@{}l@{}|@{}l@{}|}
\multicolumn{1}{@{}c@{}}{~~}&
\multicolumn{1}{@{}c@{}}{$<$$>$~}&
\multicolumn{1}{@{}c@{}}{$<$---10---$>$}&
\multicolumn{1}{@{}c@{}}{$<$---10---$>$}&
\multicolumn{1}{@{}c@{}}{$<$----12----$>$~~~}&
\multicolumn{1}{@{}c@{}}{$<$---10---$>$}&
\multicolumn{1}{@{}c@{}}{~~~~~~~~~~~~~}\\
\multicolumn{1}{@{}c@{}}{~}&
\multicolumn{1}{@{}c@{}}{\rm F.1}&
\multicolumn{1}{@{}c@{}}{\rm Field 2}&
\multicolumn{1}{@{}c@{}}{\rm Field 3}&
\multicolumn{1}{@{}c@{}}{\rm Field 4}&
\multicolumn{1}{@{}c@{}}{\rm Field 5}&
\multicolumn{1}{@{}c@{}}{~}\\
\hline
&IE &int-p-name  &             &numerical-vl &                & \\
&IR &int-p-name  &rl--p-name&              &                & \\
&IA &int-p-name  &int-p-name &numerical-vl &                & \\
&IS &int-p-name  &int-p-name &numerical-vl &                & \\
&IM &int-p-name  &int-p-name &numerical-vl &                & \\
&ID &int-p-name  &int-p-name &numerical-vl &                & \\
&I= &int-p-name  &int-p-name &              &                & \\
&I+ &int-p-name  &int-p-name &              &int-p-name    & \\
&I- &int-p-name  &int-p-name &              &int-p-name    & \\
&I* &int-p-name  &int-p-name &              &int-p-name    & \\
&I/ &int-p-name  &int-p-name &              &int-p-name    & \\
&RE &rl--p-name &             &numerical-vl &                & \\
&RI &rl--p-name &int-p-name &              &                & \\
&RA &rl--p-name &rl--p-name&numerical-vl &                & \\
&RS &rl--p-name &rl--p-name&numerical-vl &                & \\
&RM &rl--p-name &rl--p-name&numerical-vl &                & \\
&RD &rl--p-name &rl--p-name&numerical-vl &                & \\
&RF &rl--p-name &funct-name  &numerical-vl &                & \\
&R= &rl--p-name &rl--p-name&              &                & \\
&R+ &rl--p-name &rl--p-name&              &rl--p-name   & \\
&R- &rl--p-name &rl--p-name&              &rl--p-name   & \\
&R* &rl--p-name &rl--p-name&              &rl--p-name   & \\
&R/ &rl--p-name &rl--p-name&              &rl--p-name   & \\
&R( &rl--p-name &funct-name  &              &rl--p-name   & \\
&AE &r-p-a-name &             &numerical-vl &                & \\
&AI &r-p-a-name &int-p-name &              &                & \\
&AA &r-p-a-name &r-p-a-name&numerical-vl &                & \\
&AS &r-p-a-name &r-p-a-name&numerical-vl &                & \\
&AM &r-p-a-name &r-p-a-name&numerical-vl &                & \\
&AD &r-p-a-name &r-p-a-name&numerical-vl &                & \\
&AF &r-p-a-name &funct-name  &numerical-vl &                & \\
&A= &r-p-a-name &r-p-a-name&              &                & \\
&A+ &r-p-a-name &r-p-a-name&              &r-p-a-name   & \\
&A- &r-p-a-name &r-p-a-name&              &r-p-a-name   & \\
&A* &r-p-a-name &r-p-a-name&              &r-p-a-name   & \\
&A/ &r-p-a-name &r-p-a-name&              &r-p-a-name   & \\
&A( &r-p-a-name &funct-name  &              &r-p-a-name   & \\
\hline
\multicolumn{1}{@{}c@{}}{~}&
\multicolumn{1}{@{}c@{}}{$\uparrow\uparrow\;$~}&
\multicolumn{1}{@{}c@{}}{$\uparrow\;$~~~~~~~$\;\uparrow$}&
\multicolumn{1}{@{}c@{}}{$\uparrow\;$~~~~~~~$\;\uparrow$}&
\multicolumn{1}{@{}l@{}}{$\uparrow\;$~~~~~~~~~~$\uparrow$}&
\multicolumn{1}{@{}c@{}}{$\uparrow\;$~~~~~~~$\;\uparrow$}&
\multicolumn{1}{@{}c@{}}{~}\\
\multicolumn{1}{@{}c@{}}{~}&
\multicolumn{1}{@{}c@{}}{{\sz 2~3~}~}&
\multicolumn{1}{@{}c@{}}{{\sz 5~}~~~~~~~{\sz 14}}&
\multicolumn{1}{@{}c@{}}{{\sz 15}~~~~~~~{\sz 24}}&
\multicolumn{1}{@{}l@{}}{{\sz 25}~~~~~~~~~~{\sz 36}}&
\multicolumn{1}{@{}c@{}}{{\sz 40}~~~~~~~{\sz 49}}&
\multicolumn{1}{@{}c@{}}{~}\\
\ecftable{\label{Fig4.3}Possible cards for specifying parameter values}
}}}

If  the string is {\tt  IE},
the integer parameter
{\tt int-p-name}
named   in field~2  is   to be  given the  integer  value specified in
field~4.  The parameter  may  be  up to  ten characters long,  and the
integer value can occupy up to twelve positions.

If the string  is  {\tt IR},
the  integer parameter
value  named  in
field~2 is to be assigned the value of the nearest  integer (closer to
zero) to the value of the real parameter {\tt rl--p-name} specified
in field~3.  The parameter appearing in field~3 must have already been
assigned a value.

If the string is  {\tt IA},
the integer parameter  named in field~2 is
to be formed by adding the value  of  the parameter {\tt int-p-name}
referred to in field~3 to the integer value  specified in field~4. The
parameter appearing  in   field~3 must have  already   been assigned a
value.

If the string is {\tt  IS},
the  integer parameter named in field~2 is
to  be formed    by subtracting  the value    of    the parameter {\tt
int-p-name} referred to in field~3 from  the integer value specified
in field~4. The parameter appearing in field~3 must  have already been
assigned a value.

If the string is {\tt IM},
the value of the integer parameter named in
field~2  is to be obtained  by multiplying the value already specified
for  the   parameter  in  field~3 by   the integer  value specified in
field~4. Once  again, the parameter appearing  in  field~3  must have
already been assigned a value.

If the string is {\tt ID},
the value of the integer parameter named in
field~2 is to be obtained  by dividing the  integer value specified in
field~4 by the value already specified  for the  parameter in field~3.
Once again, the parameter appearing in field~3  must have already been
assigned a value.

If the string is {\tt I=},
the value of the integer parameter named in
field~2 is to be set to the  integer value specified for the parameter
in field~3.  The parameter appearing in field~3 must have already been
assigned a value.

If the string is {\tt I+},
the value of the integer parameter named in
field~2  is  to be  calculated by  adding   the values of the  integer
parameters  {\tt  int-p-name}  referred  to in  fields~3 and  5. The
parameters appearing in fields~3 and 5 must have already been assigned
values.

If the string is {\tt I-},
the value of the integer parameter named in
field~2 is to  be calculated by  subtracting the value  of the integer
parameters {\tt int-p-name}  referred  to in field~5  from   that in
field~3. The parameters appearing in fields~3 and 5 must  have already
been assigned values.

If the string is {\tt I*},
the value of the integer parameter named in
field~2 is to be formed as the product of the values already specified
for the integer parameters in fields~3 and 5. The parameters appearing
in fields~3 and 5 must have already been assigned values.

Finally, if the string is {\tt I/},
the value of the integer parameter
named in field~2 is  to be formed by dividing  the value specified for
the  integer parameters in field~3  by that specified for the  integer
parameters   in field~5.  Once    again, the  parameters  appearing in
fields~3 and 5 must have already been assigned values.

Note that, as an array  name
can only  be a  maximum of 10  characters
long, any integer parameter which is to be the  index of an  array can
only  be at most seven  characters  in   length.  Furthermore, such  a
parameter name may not include the characters ``{\tt (}'', ``{\tt )}''
or ``{\tt ,}''.

If the string is {\tt RE},
the real parameter {\tt rl--p-name} named
in field~2  is to be  given the real value  specified  in field~4. The
parameter  may be up  to ten characters long, and  the real  value can
occupy up to twelve positions.

If the string is {\tt RI},
the real  parameter value named  in field~2
is to be assigned the equivalent floating point  value of  the integer
parameter  {\tt int-p-name} specified    in field~3.  The  parameter
appearing in field~3 must have already been assigned a value.

If the string is {\tt RA},
the value  of  the real parameter named in
field~2 is to be formed by adding the value of the real parameter {\tt
rl--p-name} referred to in field~3 to  the real  value specified in
field~4. The parameter  appearing in field~3  must  have  already been
assigned a value.

If the string is  {\tt RS},
the value  of the  real parameter named in
field~2 is to be formed by subtracting the value of the real parameter
{\tt rl--p-name}  referred  to  in  field~3   from the real   value
specified in field~4. The  parameter  appearing in field~3  must  have
already been assigned a value.

If the string is {\tt RM},
the value of the parameter named in field~2
is to  be formed  by multiplying   the  value specified  for  the real
parameter in  field~3  by the real  value  specified in field~4.  Once
again, the  parameter  appearing in field~3   must  have already  been
assigned a value.

If the string is {\tt RD},
the value of the parameter named in field~2
is to be formed by dividing the real value specified in field~4 by the
value  specified  for  the real parameter   in  field~3. The parameter
appearing in field 3 must have already been assigned a value.

If the string is {\tt RF},
the value of the parameter named in field~2
is  to be  formed  by evaluating the function named  in field~3 at the
real value specified in field~4.   The  function {\tt funct-name} ---
and its mathematical  equivalent  $f(x)$ --- may  be one of: {\tt ABS}
($f(x)  = |x|$), {\tt  SQRT} ($f(x) = \sqrt{x}$),  {\tt EXP}  ($f(x) =
e^x$), {\tt LOG} ($f(x) = \log_{e} x$), {\tt LOG10} ($f(x) = \log_{10}
x$), {\tt SIN} ($f(x) = \sin x$),  {\tt COS}  ($f(x) = \cos  x$), {\tt
TAN} ($f(x) =  \tan x$),  {\tt  ARCSIN} ($f(x) = \sin^{-1}  x$),  {\tt
ARCCOS} ($f(x) = \cos^{-1} x$), {\tt  ARCTAN} ($f(x)  =  \tan^{-1} x$),
{\tt HYPSIN} ($f(x) = \sinh  x$), {\tt  HYPCOS}  ($f(x) = \cosh x$) or
{\tt HYPTAN} ($f(x) = \tanh x$).
Certain of the functions may only be
evaluated for arguments  lying within restricted ranges.  The argument
for {\tt SQRT}  must be  non-negative, those for   {\tt LOG} and  {\tt
LOG10} must be strictly positive, and those for {\tt ARCSIN}  and {\tt
ARCCOS} must be no larger than one in absolute value.

If the string is {\tt R=},
the parameter
value named  in field~2 is to
be  assigned the  value of   the  real parameter  {\tt  rl--p-name}
referred to in field~3. The parameter appearing  in field 3  must have
already been assigned a value.

If the string is {\tt R+},
the parameter  value named in field~2 is to
be  formed as the   sum of the   values  of the  real  parameters {\tt
rl--p-name} referred to in fields~3 and 5. The parameters appearing
in fields~3 and 5 must have already been assigned values.

If the string is {\tt R-},
the parameter value named in field~2  is to
be    formed by subtracting    the value of  the real   parameter {\tt
rl--p-name} referred to in field~5 from the value of that referred
to in  field~3. The parameters appearing  in fields~3 and  5 must have
already been assigned values.

If the string is  {\tt  R*},
the value  of the real parameter named in
field~2 is to be formed as the product of the values already specified
for the real parameters in fields~3  and 5. Once again, the parameters
appearing in fields~3 and 5 must have already been assigned values.

If the string is {\tt R/},
the parameter value named  in field~2 is to
be   formed  by dividing  the    value  of  the   real parameter  {\tt
rl--p-name} referred to in field~3  by the value of that  referred
to  in field~5. The parameters appearing  in fields~3 and 5 must  have
already been assigned values.

Finally, if the string is {\tt R(},
the value of the  parameter named
in field~2 is to be formed by evaluating the function named in field~3
at the value of  the real parameter  {\tt  rl--p-name} specified in
field~5. The function (and its mathematical equivalent)  may be any of
those named  in the {\tt  RF} paragraph and  the  restrictions on  the
allowed argument ranges given above still apply.

If the first  character  in field~1 is  an  {\tt A}, an array
of real
parameters is to be defined. The  particular type of definition is as
for the {\tt R} cards,
excepting that any name,  {\tt r-p-a-name},
referred  to  in  fields~2,  3  or  5, with  the  exception of integer
parameters named in field~3 of {\tt  AI} cards and functions  named in
the  same  field   of {\tt AF}  and  {\tt A(}  cards, must   be a real
parameter array name
with a valid index.

Parameter assignments may  be made at any  point  within the SDIF file
between the  {\tt NAME}
and {\tt  ENDATA}
indicator cards.   It is anticipated that  parameters
will be used to store values such as the total number of variables and
groups, which are used later in array
definitions, and to allow a user
to enter regular and repetitious data in a straightforward and compact
way.

\subsubsection{\label{S2.2.4}Do-loops}

A do-loop may occur at any point in the {\tt GROUPS}, {\tt VARIABLES},
{\tt CONSTANTS}, {\tt RANGES},  {\tt BOUNDS},
{\tt START POINT}, 
%{\bf New}
{\tt QUADRATIC}, {\tt ELEMENT USES} or
{\tt GROUP USES} sections.
Do-loops are  used to make   array
definitions,  that is,   to  make compact definitions of
several quantities at once. The syntax  required for do-loops is given
in Figure~\ref{F2.2.4}.

{\renewcommand{\arraystretch}{0.8}
{\small {\tt
\bcftable{|@{}c@{}|@{}l@{}|@{}l@{}|@{}l@{}|@{}l@{}|@{}l@{}|@{}l@{}|}
\multicolumn{1}{@{}c@{}}{~~}&
\multicolumn{1}{@{}c@{}}{$<$$>$~}&
\multicolumn{1}{@{}c@{}}{$<$---10---$>$}&
\multicolumn{1}{@{}c@{}}{$<$---10---$>$}&
\multicolumn{1}{@{}c@{}}{~~~~~~~~~~~~~~~~}&
\multicolumn{1}{@{}c@{}}{$<$---10---$>$}&
\multicolumn{1}{@{}c@{}}{~~~~~~~~~~~}\\
\multicolumn{1}{@{}c@{}}{~}&
\multicolumn{1}{@{}c@{}}{F.1}&
\multicolumn{1}{@{}c@{}}{Field 2}&
\multicolumn{1}{@{}c@{}}{Field 3} &
\multicolumn{1}{@{}c@{}}{~}&
\multicolumn{1}{@{}c@{}}{Field 5}&
\multicolumn{1}{@{}c@{}}{~}\\
\hline
&  DO  & int-p-name  &  int-p-name  &         &  int-p-name & \\
&  DI  & int-p-name  &  int-p-name  &         &               & \\
\hline
\multicolumn{1}{|@{}c@{}}{~}&
\multicolumn{1}{@{}c@{}}{~}&
\multicolumn{5}{l@{}|}{\rm one or more array definitions} \\
\hline
&  OD  &  int-p-name &                &         &               & \\
&  ND  &               &                &         &               & \\
\hline
\multicolumn{1}{@{}c@{}}{~}&
\multicolumn{1}{@{}c@{}}{$\uparrow\uparrow\;$~}&
\multicolumn{1}{@{}c@{}}{$\uparrow\;$~~~~~~~$\;\uparrow$}&
\multicolumn{1}{@{}c@{}}{$\uparrow\;$~~~~~~~$\;\uparrow$}&
\multicolumn{1}{@{}c@{}}{~}&
\multicolumn{1}{@{}c@{}}{$\uparrow\;$~~~~~~~$\;\uparrow$}&
\multicolumn{1}{@{}c@{}}{~}\\
\multicolumn{1}{@{}c@{}}{~}&
\multicolumn{1}{@{}c@{}}{{\sz 2~3~}~}&
\multicolumn{1}{@{}c@{}}{{\sz 5~}~~~~~~~$\,${\sz 14}}&
\multicolumn{1}{@{}c@{}}{{\sz 15}~~~~~~~$\,${\sz 24}}&
\multicolumn{1}{@{}c@{}}{~}&
\multicolumn{1}{@{}c@{}}{{\sz 40}~~~~~~~$\,${\sz 49}}&
\multicolumn{1}{@{}c@{}}{~}\\
\ecftable{\label{F2.2.4}Syntax for do-loops}
}}}

The two-character string in data field 1  specifies either  the start or
the end of a do-loop. The start of a loop is indicated  by  the string
{\tt DO.}
In   this   case an integer   parameter
named in field~2  is
defined to take values starting from the integer parameter value given
in field~3 and ending with the last value before the integer parameter
value given  in field~5 has been surpassed.   The parameters  named in
fields~3  and 5 must  have been defined  on previous data  cards.
The
parameter name
defined in field 2 can occupy up to ten locations.   If
the next data card does not have the characters {\tt  DI}
as its first field, the parameter defined on the  {\tt DO}
card, {\tt iloop} say, will take all integer values starting from that
given in field  3, say  {\tt istart}, and  ending on that in field  5,
{\tt iend} say.  If {\tt istart} is larger  than {\tt iend},  the loop
will be skipped.

If the data card
following a {\tt DO}
card has the string  {\tt DI}
in field~1,   the  do-loop parameter
named    in   field~2 is    to be
incremented
by the amount,  {\tt incr} say,  specified for the integer
parameter given in field~3.  Once again, the parameter in field~3 must
have been previously  defined.  The  index {\tt  iloop} will now  take
values
\bdmath
{\tt iloop} = {\tt istart}  + j \cdot {\tt incr}
\edmath
for all non-negative $j$  for which  {\tt   iloop}  lies   between  (and
including) {\tt istart} and {\tt iend}.  If {\tt incr} is negative and
{\tt istart}  is  larger than  {\tt  iend},  the parameter specifies a
decreasing sequence of  values.   If {\tt incr}  is positive  and {\tt
istart} is larger  than {\tt iend},  or  if {\tt incr} is negative and
{\tt istart} is smaller than {\tt iend}, the loop will be skipped.

Once a do-loop has been  started, any array
definitions  which use its
do-loop index specify that the definition is to be made for all values
of the integer parameter
specified in  the loop.   Loops can be nested
up to three deep;  this corresponds to the  maximum number  of allowed
indices in an array
index list.

A do-loop must be terminated. A particular loop can be terminated on a
data card
in which field~1 contains the characters  {\tt OD};
the name of the loop
parameter must appear in field~2. Alternatively, all loops
may be terminated at once using a data  card in which field~1 contains
the characters {\tt ND}.

In   addition, parameter     assignments with the   syntax  given   in
Figure~\ref{Fig4.3} --- that is, cards  whose first  field are {\tt IE},
{\tt IR}, {\tt IA}, {\tt IS}, {\tt IM}, {\tt ID},  {\tt I=}, {\tt I+},
{\tt I-}, {\tt I*}, {\tt I/}, {\tt RE}, {\tt RI},  {\tt RA}, {\tt RS},
{\tt RM}, {\tt RD}, {\tt RF}, {\tt R=}, {\tt R+}, {\tt  R-}, {\tt R*},
{\tt R/}, {\tt R(}, {\tt AE}, {\tt AI},  {\tt AA}, {\tt AS}, {\tt AM},
{\tt AD}, {\tt AF}, {\tt A=},  {\tt A+}, {\tt  A-}, {\tt A*}, {\tt A/}
or {\tt R(} --- may be inserted at any point in a  do-loop; it is only
necessary that a parameter is defined prior to its use.

Note   that  array
definitions  may  occur  both  within  and  outside
do-loops; all that is required  for  a  successful array definition is
that  the  integer indices   used have   defined values  when they are
needed.  The use of do-loops is illustrated in Section~\ref{S2.4}.

\subsubsection{\label{S2.2.5}The Definition of Variables and Groups}

In  the MPS
standard,  the  constraint
matrix,   the matrix of linear elements,
is input by  columns;
firstly the  names of the constraints
are specified  in the {\tt  ROWS}
 section and  then variable names
and the corresponding  matrix coefficients are  set one  at a  time in the
{\tt COLUMNS}
section. While there is some justification for this form
of  matrix  entry  for linear programming
problems --- the principal  solution algorithm  for such problems, the
simplex  method  \cite{Dant63}, is usually  column oriented ---  there
seems no good reason why the coefficients of linear elements
might not also  be input by  rows.
After all, it is  more  natural to think of
specifying the constraints
for a  problem one at a time.  Furthermore,
requiring that a complete row or column  has been specified before the
next may be processed is unnecessarily restrictive.

We thus allow the data to be input in a either a group-wise (row-wise)
or  variable-wise (column-wise) fashion.  In  a group/row-wise scheme,
one or two coefficients and their  variable/column names are specified
for a given group/row; for  a  variable/column-wise scheme, one or two
coefficients and their  group/row  names   are specified for   a given
variable/column.  We do, however, still require that in a group/row-wise
storage scheme, the names of all the  variables/rows {\it which appear
in linear elements}
are  completely specified before  the coefficients
are  input.  Similarly,  in a variable/column-wise storage scheme, the
names of all the groups/rows {\it which have a linear element}
must be
completely specified  before the coefficients   are input. This allows
for some checking of the input data.

If the groups/rows are specified first,  there is no  requirement that
variables/columns are input one at a time (but of course they may be).
When   processing  the  data   file, variable/column  names  should be
inspected to see if they  are new or  where they have appeared before.
Likewise,  if the variables/columns are specified  first,  there is no
requirement that groups/rows are ordered on input.  The coordinates of
new data  values can then  be  stored  as  a linked triple (group/row,
variable/column, value).  Conversion from such a  component-wise input
scheme to a row or column based  storage scheme may be performed very
efficiently  if  desired   (see  \cite[pp30--31]{DuffErisReid86}, and
subroutine MC39 in the Harwell Subroutine Library).

If a variable/column-wise input scheme is to be adopted, the data file
will contain a  {\tt GROUPS/\-ROWS/\-CONSTRAINTS}
indicator  card
and section followed by a  {\tt VARIABLES/COLUMNS}
card and section.  The
allowed  data  cards  are   discussed   in    Section~\ref{S2.2.6} and
Section~\ref{S2.2.7}.   If   a group/row-wise  input scheme  is  to be
adopted, the  data   file   will  contain a  {\tt   VARIABLES/COLUMNS}
indicator   card
and section followed by a {\tt GROUPS/\-ROWS/\-CONSTRAINTS}
card  and section.  The data  cards
for  this  scheme are discussed in
Section~\ref{S2.2.8} and Section~\ref{S2.2.9}.

\subsubsection{\label{S2.2.6}The {\tt GROUPS}, {\tt ROWS} or
{\tt CONSTRAINTS} Data Cards \protect\\
(variable/\-col\-umn-wise)}
The {\tt GROUPS}, {\tt ROWS} and {\tt CONSTRAINTS}
indicator cards
are used interchangeably to  announce the names of the groups
which  make up   the  objective  function
or, for  constrained
problems, the  names  of the  constraints
(or rows, as  they  are often known in linear programming
applications).  The user may give a  scaling factor for the  groups or
constraints.
In addition,   groups which  are  linear combinations   of
previous groups may be specified. The  syntax for the data cards
which follow these indicator cards
is given in Figure~\ref{F2.2.6}.

{\renewcommand{\arraystretch}{0.8}
{\small {\tt
\bcftable{|@{}c@{}|@{}l@{}|@{}l@{}|@{}l@{}|@{}l@{}|@{}l@{}|@{}l@{}|}
\multicolumn{1}{@{}c@{}}{~~}&
\multicolumn{1}{@{}c@{}}{$<$$>$~}&
\multicolumn{1}{@{}c@{}}{$<$---10---$>$}&
\multicolumn{1}{@{}c@{}}{$<$---10---$>$}&
\multicolumn{1}{@{}c@{}}{$<$----12----$>$~~~}&
\multicolumn{1}{@{}c@{}}{$<$---10---$>$}&
\multicolumn{1}{@{}c@{}}{$<$----12----$>$}\\
\multicolumn{1}{@{}c@{}}{}&
\multicolumn{1}{@{}c@{}}{\rm F.1}&
\multicolumn{1}{@{}c@{}}{\rm Field 2}&
\multicolumn{1}{@{}c@{}}{\rm Field 3}&
\multicolumn{1}{@{}c@{}}{\rm Field 4}&
\multicolumn{1}{@{}c@{}}{\rm Field 5}&
\multicolumn{1}{@{}c@{}}{\rm Field 6}\\
\hline
\multicolumn{7}{|@{}l@{}|}{GROUPS {\rm or}}  \\
\multicolumn{7}{|@{}l@{}|}{ROWS {\rm or}}  \\
\multicolumn{7}{|@{}l@{}|}{CONSTRAINTS}  \\
\hline
& N &group-name&\$\$\$\$\$\$\$\$\$\$&numerical-vl &         & \\
& G &group-name&\$\$\$\$\$\$\$\$\$\$&numerical-vl &         & \\
& L &group-name&\$\$\$\$\$\$\$\$\$\$&numerical-vl &         & \\
& E &group-name&\$\$\$\$\$\$\$\$\$\$&numerical-vl &         & \\
& XN&group-name&\$\$\$\$\$\$\$\$\$\$&numerical-vl &         & \\
& XG&group-name&\$\$\$\$\$\$\$\$\$\$&numerical-vl &         & \\
& XL&group-name&\$\$\$\$\$\$\$\$\$\$&numerical-vl &         & \\
& XE&group-name&\$\$\$\$\$\$\$\$\$\$&numerical-vl &         & \\
& ZN&group-name&\$\$\$\$\$\$\$\$\$\$&              &r-p-a-name& \\
& ZG&group-name&\$\$\$\$\$\$\$\$\$\$&              &r-p-a-name& \\
& ZL&group-name&\$\$\$\$\$\$\$\$\$\$&              &r-p-a-name& \\
& ZE&group-name&\$\$\$\$\$\$\$\$\$\$&              &r-p-a-name& \\
& DN&group-name&\$\$\$\$\$\$\$\$\$\$&numerical-vl
                &\$\$\$\$\$\$\$\$\$\$&numerical-vl \\
& DG&group-name&\$\$\$\$\$\$\$\$\$\$&numerical-vl
                &\$\$\$\$\$\$\$\$\$\$&numerical-vl \\
& DL&group-name&\$\$\$\$\$\$\$\$\$\$&numerical-vl
                &\$\$\$\$\$\$\$\$\$\$&numerical-vl \\
& DE&group-name&\$\$\$\$\$\$\$\$\$\$&numerical-vl
                &\$\$\$\$\$\$\$\$\$\$&numerical-vl \\
\hline
\multicolumn{1}{@{}c@{}}{~}&
\multicolumn{1}{@{}c@{}}{$\uparrow\uparrow\;$~}&
\multicolumn{1}{@{}c@{}}{$\uparrow\;$~~~~~~~$\;\uparrow$}&
\multicolumn{1}{@{}c@{}}{$\uparrow\;$~~~~~~~$\;\uparrow$}&
\multicolumn{1}{@{}l@{}}{$\uparrow\;$~~~~~~~~~$\;\uparrow$}&
\multicolumn{1}{@{}c@{}}{$\uparrow\;$~~~~~~~$\;\uparrow$}&
\multicolumn{1}{@{}c@{}}{$\uparrow\;$~~~~~~~~~$\;\uparrow$}\\
\multicolumn{1}{@{}c@{}}{~}&
\multicolumn{1}{@{}c@{}}{{\sz 2~3~}~}&
\multicolumn{1}{@{}c@{}}{{\sz 5~}~~~~~~~{\sz 14}}&
\multicolumn{1}{@{}c@{}}{{\sz 15}~~~~~~~{\sz 24}}&
\multicolumn{1}{@{}l@{}}{{\sz 25}~~~~~~~~~{\sz 36}}&
\multicolumn{1}{@{}c@{}}{{\sz 40}~~~~~~~{\sz 49}}&
\multicolumn{1}{@{}c@{}}{{\sz 50}~~~~~~~~~{\sz 61}}\\
\ecftable{\label{F2.2.6}Possible data cards for {\tt GROUPS, ROWS} or
{\tt CONSTRAINTS} \protect\\ (column-wise)}
}}}

The one- or two-character string in data field~1  specifies the type of
group, row or constraint
to  be input.  Possible values for the first
character are:
\begin{description}
\item[\tt N :]
the  group is  to be specially marked   (for constrained problems, the
group/\-row is an objective function group/\-row).
\item[\tt G :]
the group  is to use  an extra  ``artificial'' variable; this variable
will only occur in this particular group, will be non-negative and its
value will be subtracted   from  the group function.  For  constrained
problems,  this is  equivalent  to   requiring   the constraint/row be
non-negative; the  extra   variable is then  a   surplus variable  and
whether it is used  explicitly (considered as  a  problem variable) or
implicitly will  depend  upon  the optimization technique  to be used.
Thus, if the  problem variables are $x$,  and the  $k$-th group has  a
linear element
$a_k^T x - b_k$, the linear element that will be passed
to the optimization procedure could be $a_k^T x - y_k - b_k$, for some
non-negative variable $y_k$.
\item[\tt L :]
the  group is to use an  extra ``artificial''  variable; this variable
will only occur in this particular group, will be non-negative and its
value will be added to the group  function.  For constrained problems,
this is  equivalent to requiring  the constraint/row
be non-positive; the extra variable is then a slack
variable and may be used explicitly or  implicitly by the optimization
procedure.   Thus, if the  linear element
is  as  specified above, the
linear element that will be passed to the optimization procedure could
be $a_k^T x + y_k - b_k$, for some non-negative variable $y_k$.
\item[\tt E :]
the    group   is   a  normal   one  (for   constrained problems,  the
row/constraint is an equality),
\item[{\tt X} {\rm and} {\tt Z} :]
an  array
of  groups  are  to  be  defined at   once.  When the  first
character is an {\tt X} or {\tt Z}, the second character may be one of
{\tt N}, {\tt G}, {\tt L} or {\tt E}.   The  resulting array
of groups
then each has the  characteristics of an {\tt N},  {\tt G}, {\tt L} or
{\tt E} group as just described.
\item[\tt D :]
the group  is to be  formed as  a linear   combination of two previous
groups. When  the  first character is  a {\tt D}, the second character
may be one of {\tt N}, {\tt  G},  {\tt L} or  {\tt  E}.  The resulting
group then has the characteristics of an {\tt N}, {\tt  G}, {\tt L} or
{\tt E} group as just described.
\end{description}

The string {\tt  group-name}
in data  field~2 gives the  name of  the group (or row or constraint)
under consideration.  This name may be up
to ten characters long, excepting  that the name  {\tt `SCALE'}
is not
allowed.  For  {\tt X} data  cards,
the expanded  array
name must be valid  and the integer  indices must have been defined in
a parameter assignment (see Section~\ref{S2.2.3}).

The string  $\$\$\$\$\$\$\$\$\$$  in  data field~3  may be blank;
this
happens when field 2 is  merely announcing the name of  a group. If it
is not blank, it is used for two purposes.
\begin{itemize}
\item
It may be used to announce that all the entries (if any) in the linear
element
for the group  under  consideration are to be scaled,  that is
{\em divided}  by a constant  scale factor;  in this case field~3 will
contain the string {\tt `SCALE'}.
If the first  character  in field~1 is a {\tt  Z}, the  string in data
field~5 gives the name of a previously defined  real parameter
and the
numerical value associated with this parameter gives the scale factor.
Otherwise, the string {\tt numerical-vl}, occupying up to 12 locations
in data  field~4, contains the scale factor.   Fields 5 and 6  are not
then used.
\item
If the first character in field~1 is a  {\tt D}, the  current group is
to be formed as a linear combination of the groups mentioned in fields
3 and 5; the multiplication factors are then recorded  in fields 4 and
6 respectively. Thus we will have
\[
\mbox{group in field~2} = \mbox{group in field~3} * \mbox{field~4} +
                          \mbox{group in field~5} * \mbox{field~6}.
\]
In this  case, the names of the  groups  in  fields~3 and 5  must have
already been defined.  The multiplication factors may occupy  up to 12
locations in fields~4 and 6.
\end{itemize}

\subsubsection[The {\tt VARIABLES} or {\tt COLUMNS} Data Cards
 (Variable/Column-Wise)]{\label{S2.2.7}The {\tt VARIABLES} or {\tt COLUMNS} Data
 Cards  \protect\\
(Variable/Column-Wise)}

The  {\tt VARIABLES}  or   {\tt COLUMNS}  indicator   cards
are used interchangeably to announce the (problem) variables
for  the  minimization.  In  addition,   the  entries  for  the linear
elements
are input here.  The user may also give  a scaling factor for
the entries  in  any column.
The syntax  for data  following    this indicator card
is given   in Figure~\ref{F2.2.7}.

{\renewcommand{\arraystretch}{0.8}
{\small {\tt
\bcftable{|@{}c@{}|@{}l@{}|@{}l@{}|@{}l@{}|@{}l@{}|@{}l@{}|@{}l@{}|}
\multicolumn{1}{@{}c@{}}{~~}&
\multicolumn{1}{@{}c@{}}{$<$$>$~}&
\multicolumn{1}{@{}c@{}}{$<$---10---$>$}&
\multicolumn{1}{@{}c@{}}{$<$---10---$>$}&
\multicolumn{1}{@{}c@{}}{$<$----12----$>$~~~}&
\multicolumn{1}{@{}c@{}}{$<$---10---$>$}&
\multicolumn{1}{@{}c@{}}{$<$----12----$>$}\\
\multicolumn{1}{@{}c@{}}{}&
\multicolumn{1}{@{}c@{}}{\rm F.1}&
\multicolumn{1}{@{}c@{}}{\rm Field 2}&
\multicolumn{1}{@{}c@{}}{\rm Field 3}&
\multicolumn{1}{@{}c@{}}{\rm Field 4}&
\multicolumn{1}{@{}c@{}}{\rm Field 5}&
\multicolumn{1}{@{}c@{}}{\rm Field 6}\\
\hline
\multicolumn{7}{|@{}l@{}|}{VARIABLES {\rm or}}  \\
\multicolumn{7}{|@{}l@{}|}{COLUMNS}  \\
\hline
&   &varbl-name &\$\$\$\$\$\$\$\$\$\$&numerical-vl
                 &\$\$\$\$\$\$\$\$\$\$&numerical-vl  \\
& X &varbl-name &\$\$\$\$\$\$\$\$\$\$&numerical-vl
                 &\$\$\$\$\$\$\$\$\$\$&numerical-vl  \\
& Z &varbl-name &\$\$\$\$\$\$\$\$\$\$&              & r-p-a-name & \\
\hline
\multicolumn{1}{@{}c@{}}{~}&
\multicolumn{1}{@{}c@{}}{$\uparrow\uparrow\;$~}&
\multicolumn{1}{@{}c@{}}{$\uparrow\;$~~~~~~~$\;\uparrow$}&
\multicolumn{1}{@{}c@{}}{$\uparrow\;$~~~~~~~$\;\uparrow$}&
\multicolumn{1}{@{}l@{}}{$\uparrow\;$~~~~~~~~~$\;\uparrow$}&
\multicolumn{1}{@{}c@{}}{$\uparrow\;$~~~~~~~$\;\uparrow$}&
\multicolumn{1}{@{}c@{}}{$\uparrow\;$~~~~~~~~~$\;\uparrow$}\\
\multicolumn{1}{@{}c@{}}{~}&
\multicolumn{1}{@{}c@{}}{{\sz 2~3~}~}&
\multicolumn{1}{@{}c@{}}{{\sz 5~}~~~~~~~{\sz 14}}&
\multicolumn{1}{@{}c@{}}{{\sz 15}~~~~~~~{\sz 24}}&
\multicolumn{1}{@{}l@{}}{{\sz 25}~~~~~~~~~{\sz 36}}&
\multicolumn{1}{@{}c@{}}{{\sz 40}~~~~~~~{\sz 49}}&
\multicolumn{1}{@{}c@{}}{{\sz 50}~~~~~~~~~{\sz 61}}\\
\ecftable{\label{F2.2.7}Possible data cards for {\tt VARIABLES} or
{\tt COLUMNS} (column-wise)}
}}}

The string  {\tt varbl-name} in data field~2  gives  the name of  the
variable (or column)
under consideration.  This name  may be up to ten
characters long excepting that the name {\tt  `SCALE'}
is not allowed.
If data field~1 holds the character  {\tt  X} or {\tt Z},  an array
of variables is to be defined. In this case, the  expanded array name
of the variables (or  columns) must be  valid and the  integer indices
must  have   been    defined   in   a  parameter     assignment   (see
Section~\ref{S2.2.3}).

The  string {\tt  \$\$\$\$\$\$\$\$\$\$} in  data field~3 is   used for
five purposes.
\begin{itemize}
\item
If the string  is empty, the  card
is   just  defining the name
of  a problem variable.
\item
It may be  used  to specify  that  the  variable mentioned  in field~2
occurs in the linear element
for the group given  in field~3. In this
case, the string in field~3 must have been defined in the {\tt GROUPS}
section.  If an array definition is being made, the string  in field~3
must be an array name.
\item
It may be used to announce that all the entries in the linear elements
for the  variable under consideration  are to be  scaled;
in this case field~3 will contain the string {\tt `SCALE'}.
\item
It may be  used  to specify  that  the  variable(s) mentioned  in field~2
is(are) only allowed to take integer values. In this case field~3 will
contain the string {\tt 'INTEGER'}.
\item
It may be  used  to specify  that  the  variable(s) mentioned  in field~2
is(are) only allowed to take the values 0 or 1. In this case field~3 will
contain the string {\tt 'ZERO-ONE'}.
\end{itemize}

A numerical value, whose purpose depends on the string in the previous
field,   is  now specified.   On  {\tt Z}   cards,
the value  is that
previously associated with  the  real parameter {\tt r-p-a-name} in
field~5.  On     other cards, the   actual   numerical   value    {\tt
numerical-vl} may occupy up to 12 characters in data field~4.

If field~3 indicates that an entry for the linear  element
for a group
is to be defined, the specified numerical  value gives the coefficient
of that entry.   If, on the other  hand, field~3  indicates   that all
entries for the variable  in  field~2 are to  be scaled, the specified
value gives the scale factor,
that is the  factor by  which each entry is to be divided.

On non {\tt Z} cards,
the strings in fields~5  and  6 are optional and
are used exactly as for strings~3 and 4 to define further entries or a
scale factor.

\subsubsection[The {\tt VARIABLES} or {\tt COLUMNS} Data Cards
 (Group/Row-Wise)]{\label{S2.2.8}The {\tt VARIABLES} or {\tt COLUMNS} Data Cards
 \protect\\
(Group/Row-Wise)}

The   {\tt VARIABLES} or  {\tt  COLUMNS}  indicator cards
are used interchangeably to announce the (problem) variables
for the minimization.  The user may also give a scaling factor for the
entries in the column.
The  syntax for  data  following  this  indicator  card is  given   in
Figure~\ref{F2.2.8}.

{\renewcommand{\arraystretch}{0.8}
{\small {\tt
\bcftable{|@{}c@{}|@{}l@{}|@{}l@{}|@{}l@{}|@{}l@{}|@{}l@{}|@{}l@{}|}
\multicolumn{1}{@{}c@{}}{~~}&
\multicolumn{1}{@{}c@{}}{$<$$>$~}&
\multicolumn{1}{@{}c@{}}{$<$---10---$>$}&
\multicolumn{1}{@{}c@{}}{$<$---10---$>$}&
\multicolumn{1}{@{}c@{}}{$<$----12----$>$~~~}&
\multicolumn{1}{@{}c@{}}{$<$---10---$>$}&
\multicolumn{1}{@{}c@{}}{$<$----12----$>$}\\
\multicolumn{1}{@{}c@{}}{}&
\multicolumn{1}{@{}c@{}}{\rm F.1}&
\multicolumn{1}{@{}c@{}}{\rm Field 2}&
\multicolumn{1}{@{}c@{}}{\rm Field 3}&
\multicolumn{1}{@{}c@{}}{\rm Field 4}&
\multicolumn{1}{@{}c@{}}{\rm Field 5}&
\multicolumn{1}{@{}c@{}}{\rm Field 6}\\
\hline
\multicolumn{7}{|@{}l@{}|}{VARIABLES {\rm or}} \\
\multicolumn{7}{|@{}l@{}|}{COLUMNS}  \\
\hline
&  & varbl-name &\$\$\$\$\$\$\$\$\$\$&numerical-vl&              & \\
&X & varbl-name &\$\$\$\$\$\$\$\$\$\$&numerical-vl&              & \\
&Z & varbl-name &\$\$\$\$\$\$\$\$\$\$&             &r-p-a-name & \\
\hline
\multicolumn{1}{@{}c@{}}{~}&
\multicolumn{1}{@{}c@{}}{$\uparrow\uparrow\;$~}&
\multicolumn{1}{@{}c@{}}{$\uparrow\;$~~~~~~~$\;\uparrow$}&
\multicolumn{1}{@{}c@{}}{$\uparrow\;$~~~~~~~$\;\uparrow$}&
\multicolumn{1}{@{}l@{}}{$\uparrow\;$~~~~~~~~~$\;\uparrow$}&
\multicolumn{1}{@{}c@{}}{$\uparrow\;$~~~~~~~$\;\uparrow$}&
\multicolumn{1}{@{}c@{}}{$\uparrow\;$~~~~~~~~~$\;\uparrow$}\\
\multicolumn{1}{@{}c@{}}{~}&
\multicolumn{1}{@{}c@{}}{{\sz 2~3~}~}&
\multicolumn{1}{@{}c@{}}{{\sz 5~}~~~~~~~{\sz 14}}&
\multicolumn{1}{@{}c@{}}{{\sz 15}~~~~~~~{\sz 24}}&
\multicolumn{1}{@{}l@{}}{{\sz 25}~~~~~~~~~{\sz 36}}&
\multicolumn{1}{@{}c@{}}{{\sz 40}~~~~~~~{\sz 49}}&
\multicolumn{1}{@{}c@{}}{{\sz 50}~~~~~~~~~{\sz 61}}\\
\ecftable{\label{F2.2.8}Possible data cards for {\tt VARIABLES} or
{\tt COLUMNS} (row-wise)}
}}}

The string {\tt varbl-name} in  data  field~2 gives the name of the
variable (or column)
under consideration.  This name  may be up to ten
characters long excepting that the name  {\tt `SCALE'}
is not allowed.
If  data field~1  holds  the character {\tt  X}  or {\tt  Z}, an array
definition is to be made. In this case, the expanded array name
of the variables (or columns) must  be  valid  and the integer indices
must    have  been  defined    in   a     parameter   assignment  (see
Section~\ref{S2.2.3}).

The string {\tt \$\$\$\$\$\$\$\$\$\$} in  data field~3 is used for four
purposes.
\begin{itemize}
\item
If  the string is  empty,  the card is  just  defining  the  name
of a problem variable.
Such  a card
must be inserted  for all  variables that only  appear in
nonlinear elements.

\item
It may be used to announce that all the entries in the linear elements
for the variable  under consideration are  to be  scaled.
On {\tt Z} cards,
the numerical value of this scale factor, the amount
by  which each entry is to  be divided, is  that previously associated
with the real parameter {\tt r-p-a-name}  given in field~5.   On other
cards,
the actual scale factor {\tt  numerical-vl}  occupies up to 12
characters in data field~4.
\item
It may be  used  to specify  that  the  variable(s) mentioned  in field~2
is(are) only allowed to take integer values. In this case field~3 will
contain the string {\tt 'INTEGER'}.
\item
It may be  used  to specify  that  the  variable(s) mentioned  in field~2
is(are) only allowed to take the values 0 or 1. In this case field~3 will
contain the string {\tt 'ZERO-ONE'}.
\end{itemize}

\subsubsection[The {\tt GROUPS}, {\tt ROWS} or {\tt CONSTRAINTS} Data Cards
 (Group/Row-Wise)]{\label{S2.2.9}The {\tt GROUPS}, {\tt ROWS} or {\tt
CONSTRAINTS} Data Cards  \protect\\ (Group/Row-Wise)}

The {\tt GROUPS}, {\tt ROWS} and {\tt CONSTRAINTS} indicator cards
are used interchangeably to announce the names of the groups
which make up
the objective function
and, for constrained
problems, the names of the
constraints (or rows,  as  they are often  known  in linear programming
applications).
In  addition, the entries  for the linear  elements
are input here.  The user may give a scaling factor  for the groups or
constraints.
Furthermore, groups  which  are linear  combinations  of
previous groups may be specified. The syntax  for the data cards
which follow these indicator cards
is given in Figure~\ref{F2.2.9}.

{\renewcommand{\arraystretch}{0.8}
{\small {\tt
\bcftable{|@{}c@{}|@{}l@{}|@{}l@{}|@{}l@{}|@{}l@{}|@{}l@{}|@{}l@{}|}
\multicolumn{1}{@{}c@{}}{~~}&
\multicolumn{1}{@{}c@{}}{$<$$>$~}&
\multicolumn{1}{@{}c@{}}{$<$---10---$>$}&
\multicolumn{1}{@{}c@{}}{$<$---10---$>$}&
\multicolumn{1}{@{}c@{}}{$<$----12----$>$~~~}&
\multicolumn{1}{@{}c@{}}{$<$---10---$>$}&
\multicolumn{1}{@{}c@{}}{$<$----12----$>$}\\
\multicolumn{1}{@{}c@{}}{}&
\multicolumn{1}{@{}c@{}}{\rm F.1}&
\multicolumn{1}{@{}c@{}}{\rm Field 2}&
\multicolumn{1}{@{}c@{}}{\rm Field 3}&
\multicolumn{1}{@{}c@{}}{\rm Field 4}&
\multicolumn{1}{@{}c@{}}{\rm Field 5}&
\multicolumn{1}{@{}c@{}}{\rm Field 6}\\
\hline
\multicolumn{7}{|@{}l@{}|}{GROUPS {\rm or}} \\
\multicolumn{7}{|@{}l@{}|}{ROWS {\rm or}} \\
\multicolumn{7}{|@{}l@{}|}{CONSTRAINTS} \\
\hline
&N &group-name &\$\$\$\$\$\$\$\$\$\$ &numerical-vl
                &\$\$\$\$\$\$\$\$\$\$ &numerical-vl \\
&G &group-name &\$\$\$\$\$\$\$\$\$\$ &numerical-vl
                &\$\$\$\$\$\$\$\$\$\$ &numerical-vl \\
&L &group-name &\$\$\$\$\$\$\$\$\$\$ &numerical-vl
                &\$\$\$\$\$\$\$\$\$\$ &numerical-vl \\
&E &group-name &\$\$\$\$\$\$\$\$\$\$ &numerical-vl
                &\$\$\$\$\$\$\$\$\$\$ &numerical-vl \\
&XN&group-name &\$\$\$\$\$\$\$\$\$\$ &numerical-vl
                &\$\$\$\$\$\$\$\$\$\$ &numerical-vl \\
&XG&group-name &\$\$\$\$\$\$\$\$\$\$ &numerical-vl
                &\$\$\$\$\$\$\$\$\$\$ &numerical-vl \\
&XL&group-name &\$\$\$\$\$\$\$\$\$\$ &numerical-vl
                &\$\$\$\$\$\$\$\$\$\$ &numerical-vl \\
&XE&group-name &\$\$\$\$\$\$\$\$\$\$ &numerical-vl
                &\$\$\$\$\$\$\$\$\$\$ &numerical-vl \\
&ZN&group-name &\$\$\$\$\$\$\$\$\$\$ &              &r-p-a-name &  \\
&ZG&group-name &\$\$\$\$\$\$\$\$\$\$ &              &r-p-a-name &  \\
&ZL&group-name &\$\$\$\$\$\$\$\$\$\$ &              &r-p-a-name &  \\
&ZE&group-name &\$\$\$\$\$\$\$\$\$\$ &              &r-p-a-name &  \\
&DN&group-name &\$\$\$\$\$\$\$\$\$\$ &numerical-vl
                &\$\$\$\$\$\$\$\$\$\$ &numerical-vl \\
&DG&group-name &\$\$\$\$\$\$\$\$\$\$ &numerical-vl
                &\$\$\$\$\$\$\$\$\$\$ &numerical-vl \\
&DL&group-name &\$\$\$\$\$\$\$\$\$\$ &numerical-vl
                &\$\$\$\$\$\$\$\$\$\$ &numerical-vl \\
&DE&group-name &\$\$\$\$\$\$\$\$\$\$ &numerical-vl
                &\$\$\$\$\$\$\$\$\$\$ &numerical-vl \\
\hline
\multicolumn{1}{@{}c@{}}{~}&
\multicolumn{1}{@{}c@{}}{$\uparrow \uparrow\;$~}&
\multicolumn{1}{@{}c@{}}{$\uparrow\;$~~~~~~~$\;\uparrow$}&
\multicolumn{1}{@{}c@{}}{$\uparrow\;$~~~~~~~$\;\uparrow$}&
\multicolumn{1}{@{}l@{}}{$\uparrow\;$~~~~~~~~~$\;\uparrow$}&
\multicolumn{1}{@{}c@{}}{$\uparrow\;$~~~~~~~$\;\uparrow$}&
\multicolumn{1}{@{}c@{}}{$\uparrow\;$~~~~~~~~~$\;\uparrow$}\\
\multicolumn{1}{@{}c@{}}{~}&
\multicolumn{1}{@{}c@{}}{{\sz 2~3~}~}&
\multicolumn{1}{@{}c@{}}{{\sz 5~}~~~~~~~{\sz 14}}&
\multicolumn{1}{@{}c@{}}{{\sz 15}~~~~~~~{\sz 24}}&
\multicolumn{1}{@{}l@{}}{{\sz 25}~~~~~~~~~{\sz 36}}&
\multicolumn{1}{@{}c@{}}{{\sz 40}~~~~~~~{\sz 49}}&
\multicolumn{1}{@{}c@{}}{{\sz 50}~~~~~~~~~{\sz 61}}\\
\ecftable{\label{F2.2.9}Possible data cards for {\tt GROUPS}, {\tt ROWS},
or {\tt CONSTRAINTS} \protect\\ (row-wise)}
}}}

The one- or two-character string in data field~1  specifies the type of
group, row  or constraint
to  be   input.  Possible values  for the  first  character and  their
interpretations are exactly as in Section~\ref{S2.2.6}.

The  string {\tt group-name}  in data field~2  gives the name of  the
group (or row or constraint)
under consideration.  This name   may be up   to ten  characters  long
excepting that the name {\tt `SCALE'}
is not allowed.  For {\tt X} and
{\tt  Z} data  cards,
the expanded array name
must  be valid and  the  integer indices must have  been defined  in a
parameter  assignment (see  Section~\ref{S2.2.3}).  The kind  of group
({\tt N}, {\tt L}, {\tt G} or {\tt E})
will  be  taken  to be that   which  is defined  on the    {\em first}
occurrence of  a data  card
for that group.
Subsequent contradictory information will be ignored.

The string {\tt \$\$\$\$\$\$\$\$\$\$}  in  data  field~3 is  used  for
three purposes.
\begin{itemize}
\item
It may be used to specify  that  the group  mentioned in field~2 has a
linear element
involving the variable given in field~3.  In this case,
the string in field~3 must  have been defined  in the {\tt  VARIABLES}
section.  If an array definition is being made, the string  in field~3
must be an array name.
The numerical  value of the coefficient of the
linear term corresponding  to the variable must  now be specified.  On
{\tt Z} cards,
the value  is that previously  associated with the real
parameter
{\tt r-p-a-name}  given in field~5.   On other cards, the
actual numerical  value {\tt  numerical-vl}  may occupy  up   to   12
characters in data field~4.

\item
It may be used to announce that 
the group function under consideration is to be scaled;  
%all the entries (if any) in the linear
%element for  the  group under consideration are to  be scaled; 
in this case field~3  will contain the   string  {\tt `SCALE'}.
The numerical value of the scale factor,  that is the  factor by which
the group is to be divided, is now specified exactly as above.

In these first two cases, fields 5 and 6 may be used to define further
coefficients or a scale factor for non {\tt Z} cards.

\item
If the first character in field~1 is a D,  the current  group is to be
formed as a linear combination of the groups mentioned in fields 3 and
5;  the multiplication  factors are  then recorded in  fields  4 and 6
respectively. Thus we will have
\[
\mbox{group in field~2} = \mbox{group in field~3} * \mbox{field~4} +
                          \mbox{group in field~5} * \mbox{field~6}.
\]
In this case,  the names of the groups  in  fields~3  and 5  must have
already been defined.  The multiplication factors may occupy up  to 12
locations in fields~4 and 6.
\end{itemize}

\subsubsection{\label{S2.2.10}The {\tt CONSTANTS, RHS} or {\tt RHS'}
Data Cards}

The  {\tt CONSTANTS, RHS}  or   {\tt RHS'}  indicator  cards
are  used
interchangeably to announce the definition of a vector of the constant
terms $b_i$ (in the constrained  case, the right-hand-sides) for  each
linear element.
The syntax for data following this indicator  card is
given in Figure~\ref{F2.2.10}.

{\renewcommand{\arraystretch}{0.8}
{\small {\tt
\bcftable{|@{}c@{}|@{}l@{}|@{}l@{}|@{}l@{}|@{}l@{}|@{}l@{}|@{}l@{}|}
\multicolumn{1}{@{}c@{}}{~~}&
\multicolumn{1}{@{}c@{}}{$<$$>$~}&
\multicolumn{1}{@{}c@{}}{$<$---10---$>$}&
\multicolumn{1}{@{}c@{}}{$<$---10---$>$}&
\multicolumn{1}{@{}c@{}}{$<$----12----$>$~~~}&
\multicolumn{1}{@{}c@{}}{$<$---10---$>$}&
\multicolumn{1}{@{}c@{}}{$<$----12----$>$}\\
\multicolumn{1}{@{}c@{}}{}&
\multicolumn{1}{@{}c@{}}{\rm F.1}&
\multicolumn{1}{@{}c@{}}{\rm Field 2}&
\multicolumn{1}{@{}c@{}}{\rm Field 3}&
\multicolumn{1}{@{}c@{}}{\rm Field 4}&
\multicolumn{1}{@{}c@{}}{\rm Field 5}&
\multicolumn{1}{@{}c@{}}{\rm Field 6}\\
\hline
\multicolumn{7}{|@{}l@{}|}{CONSTANTS {\rm or}} \\
\multicolumn{7}{|@{}l@{}|}{RHS {\rm or}} \\
\multicolumn{7}{|@{}l@{}|}{RHS'} \\
\hline
& &rhs--name &\$\$\$\$\$\$\$\$\$\$ &numerical-vl
               &group-name          &numerical-vl \\
&X&rhs--name &\$\$\$\$\$\$\$\$\$\$ &numerical-vl
               &group-name          &numerical-vl \\
&Z&rhs--name &\$\$\$\$\$\$\$\$\$\$ &              &r-p-a-name & \\
\hline
\multicolumn{1}{@{}c@{}}{~}&
\multicolumn{1}{@{}c@{}}{$\uparrow\uparrow\;$~}&
\multicolumn{1}{@{}c@{}}{$\uparrow\;$~~~~~~~$\;\uparrow$}&
\multicolumn{1}{@{}c@{}}{$\uparrow\;$~~~~~~~$\;\uparrow$}&
\multicolumn{1}{@{}l@{}}{$\uparrow\;$~~~~~~~~~$\;\uparrow$}&
\multicolumn{1}{@{}c@{}}{$\uparrow\;$~~~~~~~$\;\uparrow$}&
\multicolumn{1}{@{}c@{}}{$\uparrow\;$~~~~~~~~~$\;\uparrow$}\\
\multicolumn{1}{@{}c@{}}{~}&
\multicolumn{1}{@{}c@{}}{{\sz 2~3~}~}&
\multicolumn{1}{@{}c@{}}{{\sz 5~}~~~~~~~{\sz 14}}&
\multicolumn{1}{@{}c@{}}{{\sz 15}~~~~~~~{\sz 24}}&
\multicolumn{1}{@{}l@{}}{{\sz 25}~~~~~~~~~{\sz 36}}&
\multicolumn{1}{@{}c@{}}{{\sz 40}~~~~~~~{\sz 49}}&
\multicolumn{1}{@{}c@{}}{{\sz 50}~~~~~~~~~{\sz 61}}\\
\ecftable{\label{F2.2.10}Possible data cards for {\tt CONSTANTS,
RHS} or {\tt RHS'}}
}}}

The string {\tt  rhs--name} in data field~2  gives the name of the
vector of group constants/ right-hand-sides.  This  name may  be up to
ten characters long.  More than one  vector of group  constants may be
defined.

The strings {\tt \$\$\$\$\$\$\$\$\$\$} is used for two purposes.
\begin{itemize}
\item
It may  be used to  assign a default  value to all  the constants in a
particular vector.
In this case field~3  will contain the string {\tt 'DEFAULT'}.
\item
It  may   contain the name  of  a  group/row/constraint
for  which the
constant term/right-hand-side is to be specified.  Such a  string must
have been defined in the {\tt GROUPS} section.
\end{itemize}

The string  {\tt numerical-vl} in data  field~4 and  (optionally) 6
now  contains the numerical value  of the constant/right-hand-side and
may occupy up to 12 locations.

Constants for an array
of groups may also be defined on cards
in which field~1 contains the character {\tt X} or {\tt Z}.
On such cards, the expanded array name
in field~3 and (as an option on
{\tt X} cards) 5 must be valid and the integer  indices must have been
defined in a parameter assignment (see Section~\ref{S2.2.3}).  On {\tt
Z} cards, the numerical  value of the constant/right-hand-side is that
previously associated with the real parameter array,
{\tt r-p-a-name},
given in field~5.   On {\tt X} cards,
the actual numerical value  {\tt  numerical-vl}  may  occupy up  to 12
characters in data fields~4 and (optionally) 6.

Any constants not  specified  take a default  value. The default value
for the components of each vector is  initially zero.
This default may
be changed using  a card
whose third field  contains  the  string {\tt 'DEFAULT'}
as mentioned above.  On such a card,  the default value for
the vector in field~2 is given in field~4 whenever field~1 is blank
or contains the character {\tt X}.
If field~1 contains the character  {\tt Z},
the default  value  is that associated  with the real  parameter,
{\tt
rl--p-name}, named in   field~5.  The default value  applies   to each
constant  not explicitly  specified; if the  default is to be changed,
the change must be made on the first card
naming a  particular vector of constants.

\subsubsection{\label{S2.2.11}The {\tt RANGES} Data Cards}

The {\tt RANGES} indicator card
is used to announce the definition of a vector of additional bounds
on  the artificial variables introduced
in the {\tt GROUPS}
section (in the constrained case, this corresponds
to saying that  specified inequality  constraints/rows
have both lower and upper bounds).  The syntax for data following this
indicator card
is given in Figure~\ref{F2.2.11}.

{\renewcommand{\arraystretch}{0.8}
{\small {\tt
\bcftable{|@{}c@{}|@{}l@{}|@{}l@{}|@{}l@{}|@{}l@{}|@{}l@{}|@{}l@{}|}
\multicolumn{1}{@{}c@{}}{~~}&
\multicolumn{1}{@{}c@{}}{$<$$>$~}&
\multicolumn{1}{@{}c@{}}{$<$---10---$>$}&
\multicolumn{1}{@{}c@{}}{$<$---10---$>$}&
\multicolumn{1}{@{}c@{}}{$<$----12----$>$~~~}&
\multicolumn{1}{@{}c@{}}{$<$---10---$>$}&
\multicolumn{1}{@{}c@{}}{$<$----12----$>$}\\
\multicolumn{1}{@{}c@{}}{}&
\multicolumn{1}{@{}c@{}}{\rm F.1}&
\multicolumn{1}{@{}c@{}}{\rm Field 2}&
\multicolumn{1}{@{}c@{}}{\rm Field 3}&
\multicolumn{1}{@{}c@{}}{\rm Field 4}&
\multicolumn{1}{@{}c@{}}{\rm Field 5}&
\multicolumn{1}{@{}c@{}}{\rm Field 6}\\
\hline
\multicolumn{7}{|@{}l@{}|}{{\tt RANGES}} \\
\hline
& &range-name &\$\$\$\$\$\$\$\$\$\$ &numerical-vl
               &group-name          &numerical-vl \\
&X&range-name &\$\$\$\$\$\$\$\$\$\$ &numerical-vl
               &group-name          &numerical-vl \\
&Z&range-name &\$\$\$\$\$\$\$\$\$\$ &              &r-p-a-name & \\
\hline
\multicolumn{1}{@{}c@{}}{~}&
\multicolumn{1}{@{}c@{}}{$\uparrow\uparrow\;$~}&
\multicolumn{1}{@{}c@{}}{$\uparrow\;$~~~~~~~$\;\uparrow$}&
\multicolumn{1}{@{}c@{}}{$\uparrow\;$~~~~~~~$\;\uparrow$}&
\multicolumn{1}{@{}l@{}}{$\uparrow\;$~~~~~~~~~$\;\uparrow$}&
\multicolumn{1}{@{}c@{}}{$\uparrow\;$~~~~~~~$\;\uparrow$}&
\multicolumn{1}{@{}c@{}}{$\uparrow\;$~~~~~~~~~$\;\uparrow$}\\
\multicolumn{1}{@{}c@{}}{~}&
\multicolumn{1}{@{}c@{}}{{\sz 2~3~}~}&
\multicolumn{1}{@{}c@{}}{{\sz 5~}~~~~~~~{\sz 14}}&
\multicolumn{1}{@{}c@{}}{{\sz 15}~~~~~~~{\sz 24}}&
\multicolumn{1}{@{}l@{}}{{\sz 25}~~~~~~~~~{\sz 36}}&
\multicolumn{1}{@{}c@{}}{{\sz 40}~~~~~~~{\sz 49}}&
\multicolumn{1}{@{}c@{}}{{\sz 50}~~~~~~~~~$\,${\sz 61}}\\
\ecftable{\label{F2.2.11}Possible data cards for {\tt RANGES}}
}}}

The string {\tt  range-name} in data  field~2 gives the  name of  the
vector of range values.
This name may be  up to ten characters long.
More than one vector of range values may be defined.

The string {\tt \$\$\$\$\$\$\$\$\$\$} is used for two purposes.
\begin{itemize}
\item
It may be used to assign a default value to all the  range values
in a
particular vector. In this case field~3 will contain  the  string {\tt
'DEFAULT'}.

\item
It may contain the name of a group/row/constraint
for  which the range
value is to be specified.  Such a string must have been defined in the
{\tt GROUPS}
section.
\end{itemize}

In addition, the  (optional) string {\tt  group-name} in data field~5
may also define the name of a group/row/constraint
for which the range value is to be specified.

The string {\tt numerical-vl} in  data field~4 and (optionally) 6 now
contains the  numerical  value of the  relevant   range value and  may
occupy up to 12 locations.
Only groups initially specified with a {\tt G} or {\tt L} in columns~1
or 2 of  field~1 in  the {\tt  GROUPS}
section  use range  values
and therefore   only  these groups may  be specified.

Range values for an array
of groups may  also be  defined on cards
on which field~1 is the character {\tt X} or {\tt  Z}.  On such cards,
the expanded array name
in field~3 and (as an option  on  {\tt X} cards) 5
must be  valid  and the integer indices  must  have been defined  in a
parameter assignment (see Section~\ref{S2.2.3}).   On {\tt Z}  cards,
the range value
is that previously associated with the real parameter,
{\tt  r-p-a-name}, given in field~5.   On {\tt X} cards, the actual
numerical value {\tt numerical-vl} may occupy up to 12  characters in
data    fields~4 and (optionally)     6.  Using  the    terminology of
Section~\ref{S2.2.6}, the extra bound
is taken to imply the inequality
$0 \leq  y_k  \leq |\mbox{field~4 or  6}|$ on the  artificial variable
$y_k$.

Any component in  a range vector  not specified takes a default value.
The default   value for the components  of  each vector   is initially
infinite. This default may be changed  using  a card
whose third field
contains the string  {\tt 'DEFAULT'} as mentioned above.   On  such  a
card, the default value for the vector in field~2  is given in field~4
whenever field~1  is blank
or contains the   character {\tt  X}.  If
field~1  contains the  character  {\tt Z}, the default
value  is that
associated with the   real  parameter,
{\tt  rl--p-name}, named  in
field~5.  The default value applies to each range value
not explicitly
specified.  If the default is to be  changed, the  change must be made
on the first card
naming a particular vector of range values.

\subsubsection{\label{S2.2.12}The {\tt BOUNDS} Data Cards}

The {\tt BOUNDS}
indicator card is  used to announce  a vector of data
giving  lower and upper bounds on  the  unknown variables.
The syntax   for  data following  this  indicator   card  is given  in
Figure~\ref{F2.2.12}.

{\renewcommand{\arraystretch}{0.8}
{\small {\tt
\bcftable{|@{}c@{}|@{}l@{}|@{}l@{}|@{}l@{}|@{}l@{}|@{}l@{}|@{}l@{}|}
\multicolumn{1}{@{}c@{}}{~~}&
\multicolumn{1}{@{}c@{}}{$<$$>$~}&
\multicolumn{1}{@{}c@{}}{$<$---10---$>$}&
\multicolumn{1}{@{}c@{}}{$<$---10---$>$}&
\multicolumn{1}{@{}c@{}}{$<$----12----$>$~~~}&
\multicolumn{1}{@{}c@{}}{$<$---10---$>$}&
\multicolumn{1}{@{}c@{}}{$<$----12----$>$}\\
\multicolumn{1}{@{}c@{}}{}&
\multicolumn{1}{@{}c@{}}{\rm F.1}&
\multicolumn{1}{@{}c@{}}{\rm Field 2}&
\multicolumn{1}{@{}c@{}}{\rm Field 3}&
\multicolumn{1}{@{}c@{}}{\rm Field 4}&
\multicolumn{1}{@{}c@{}}{\rm Field 5}&
\multicolumn{1}{@{}c@{}}{~}\\
\hline
\multicolumn{7}{|@{}l@{}|}{\tt BOUNDS} \\
\hline
&LO &bound-name &\$\$\$\$\$\$\$\$\$\$ &numerical-vl &              &  \\
&UP &bound-name &\$\$\$\$\$\$\$\$\$\$ &numerical-vl &              &  \\
&FX &bound-name &\$\$\$\$\$\$\$\$\$\$ &numerical-vl &              &  \\
&FR &bound-name &\$\$\$\$\$\$\$\$\$\$ &              &              &  \\
&MI &bound-name &\$\$\$\$\$\$\$\$\$\$ &              &              &  \\
&PL &bound-name &\$\$\$\$\$\$\$\$\$\$ &              &              &  \\
&XL &bound-name &\$\$\$\$\$\$\$\$\$\$ &numerical-vl &              &  \\
&XU &bound-name &\$\$\$\$\$\$\$\$\$\$ &numerical-vl &              &  \\
&XX &bound-name &\$\$\$\$\$\$\$\$\$\$ &numerical-vl &              &  \\
&XR &bound-name &\$\$\$\$\$\$\$\$\$\$ &              &              &  \\
&XM &bound-name &\$\$\$\$\$\$\$\$\$\$ &              &              &  \\
&XP &bound-name &\$\$\$\$\$\$\$\$\$\$ &              &              &  \\
&ZL &bound-name &\$\$\$\$\$\$\$\$\$\$ &              &r-p-a-name &  \\
&ZU &bound-name &\$\$\$\$\$\$\$\$\$\$ &              &r-p-a-name &  \\
&ZX &bound-name &\$\$\$\$\$\$\$\$\$\$ &              &r-p-a-name &  \\
\hline
\multicolumn{1}{@{}c@{}}{~}&
\multicolumn{1}{@{}c@{}}{$\uparrow\uparrow\;$~}&
\multicolumn{1}{@{}c@{}}{$\uparrow\;$~~~~~~~$\;\uparrow$}&
\multicolumn{1}{@{}c@{}}{$\uparrow\;$~~~~~~~$\;\uparrow$}&
\multicolumn{1}{@{}l@{}}{$\uparrow\;$~~~~~~~~~$\;\uparrow$}&
\multicolumn{1}{@{}c@{}}{$\uparrow\;$~~~~~~~$\;\uparrow$}&
\multicolumn{1}{@{}c@{}}{~}\\
\multicolumn{1}{@{}c@{}}{~}&
\multicolumn{1}{@{}c@{}}{{\sz 2~3~}~}&
\multicolumn{1}{@{}c@{}}{{\sz 5~}~~~~~~~{\sz 14}}&
\multicolumn{1}{@{}c@{}}{{\sz 15}~~~~~~~{\sz 24}}&
\multicolumn{1}{@{}l@{}}{{\sz 25}~~~~~~~~~{\sz 36}}&
\multicolumn{1}{@{}c@{}}{{\sz 40}~~~~~~~$\,${\sz 49}}&
\multicolumn{1}{@{}c@{}}{~}\\
\ecftable{\label{F2.2.12}Possible data cards for {\tt BOUNDS}}
}}}

The two-character string in data  field~1 specifies  the type of bound
to  be input.   Possible values are: {\tt LO},  {\tt XL} or {\tt ZL},
in the case of a lower bound,
{\tt UP}, {\tt  XU}, {\tt ZU},in the case of an upper  bound,
{\tt FX}, {\tt XX}, {\tt ZX},in the case of a fixed variable,
i.e., the lower and  upper bounds are equal,  {\tt  FR} or {\tt  XR} if the
variable is free,
i.e., the lower and upper bounds are infinite, {\tt MI}
or {\tt XM}, if there is no lower  bound, and {\tt PL} or  {\tt XP}, if there
is no  upper bound.
The  string  {\tt bound-name} in  data  field~2 gives  the name of the
bound vector   under consideration.    This  name may  be up   to  ten
characters long.   Several different bound  vectors may be  defined in
the {\tt BOUNDS} section.

The string {\tt \$\$\$\$\$\$\$\$\$\$} is used for two purposes.
\begin{itemize}
\item
It may contain the name of a variable/column for  which  a bound is to
be  specified.  This name may be  up to ten   characters long and must
refer to a variable defined in the {\tt VARIABLE} data.

If the card  is of type  {\tt LO}, {\tt UP},  {\tt FX}, {\tt FR}, {\tt
MI}, or {\tt PL,}
the string in data field~3 specifies  to which variable
the bound is applied.  If the card is of type {\tt XL}, {\tt ZL}, {\tt
XU}, {\tt ZU},  {\tt XX}, {\tt ZX}, {\tt  XR}, {\tt XM} or  {\tt  XP},
this string specifies an array
of variables which  are to be bounded.
On such cards,
the expanded array  name
of this string must  be valid
and the  integer  indices  must  have  been  defined   in  a parameter
assignment (see Section~\ref{S2.2.3}).

For bounds of type {\tt LO}, {\tt UP}, {\tt FX}, {\tt XL}, {\tt XU} or
{\tt XX},
the numerical value of the bound or array
of bounds is given
as the string {\tt numerical-vl}, using at most 12 characters, in data
field~4.  For  bounds of  type {\tt ZL},  {\tt ZU}  or  {\tt  ZX}, the
numerical value of the  array
of bounds is that previously associated
with the real  parameter array {\tt  r-p-a-name} specified in field~5.
When both lower and upper bounds
on a variable are required, they must
be  specified  on separate   cards.
Possible combinations    are {\tt
LO}--{\tt UP}, {\tt LO}--{\tt PL},  {\tt MI}--{\tt UP}, {\tt XL}--{\tt
XU}, {\tt XL}--{\tt XP}, {\tt XM}--{\tt XU},  {\tt ZL}--{\tt XU}, {\tt
XL}--{\tt ZU},   {\tt  ZL}--{\tt  ZU},  {\tt  ZL}--{\tt  XP}  and {\tt
XM}--{\tt ZU}.

\item
It may be used to assign a default value to all the lower and/or upper
bounds
in a particular vector.  In this case  field~3 will contain the
string {\tt 'DEFAULT'}.
Each bound  vector is  given default lower  and upper  bounds on every
variable.  The value of the default lower bound is  initially zero and
the upper bound is initially  infinite.  These default values may  be
changed.  A new default value for the  vector in field~2 is then given
in field~4 whenever field~1 is {\tt LO}, {\tt UP}, {\tt FX}, {\tt FR},
{\tt MI},  {\tt PL}
or starts with  the  character {\tt X}.  If  field~1  starts with  the
character {\tt Z}, the default value is that associated with  the real
parameter,
{\tt  rl--p-name},  named   in field~5.  The   appropriate
default value applies to each bound
not explicitly specified.  If the
default is to be changed, the change must  be  made on  the first card
naming a particular vector of bounds.

If default lower and  upper bounds
of zero and  infinity, respectively,
are in effect and a   card
with  {\tt MI} or   {\tt XM}
in field~1  is
encountered, the  relevant lower bound
is  changed to  minus  infinity
{\em and} the upper bound becomes zero.  If  the  same defaults are in
effect and an upper bound
of zero is specified on a {\tt UP}, {\tt XU}
or {\tt ZU}
card,
the relevant  lower  bound becomes minus  infinity.
These two features are necessary for MPS compatibility.
\end{itemize}

\subsubsection{\label{S2.2.13}The {\tt START POINT} Data Cards}

The {\tt START POINT}
indicator  card
is used to announce a vector of initial estimates of the values of the
unknown variables and, in the case of problems with general
constraints, Lagrange multipliers.
The   Lagrangian
function     associated    with
\req{objective}--\req{inequality_constraints} is the function
\begin{eqnarray*}
l(x,\lambda) & = &
\sum_{i \in I_{0}}
        g_i\left(  \sum_{j \in J_i} w_{i,j} f_j (x_j) + a_i^T x - b_i \right)\\
& & \ms \ms \ms \ms + \sum_{i \in I_E \cup I_I}
\lambda_i g_i\left( \sum_{j \in J_i} w_{i,j} f_j (x_j) + a_i^T x - b_i \right)
\end{eqnarray*}
where the scalars $\lambda_i$ are known as Lagrange multipliers.
Good  estimates  of these  parameters  can   sometimes be  useful  for
optimization procedures  (see, for example,    \cite{GillMurrWrig81}).
The  syntax  for  data  following   this  indicator card
is  given in Figure~\ref{F2.2.13}.

{\renewcommand{\arraystretch}{0.8}
{\small {\tt
\bcftable{|@{}c@{}|@{}l@{}|@{}l@{}|@{}l@{}|@{}l@{}|@{}l@{}|@{}l@{}|}
\multicolumn{1}{@{}c@{}}{~~}&
\multicolumn{1}{@{}c@{}}{$<$$>$~}&
\multicolumn{1}{@{}c@{}}{$<$---10---$>$}&
\multicolumn{1}{@{}c@{}}{$<$---10---$>$}&
\multicolumn{1}{@{}c@{}}{$<$----12----$>$~~~}&
\multicolumn{1}{@{}c@{}}{$<$---10---$>$}&
\multicolumn{1}{@{}c@{}}{$<$----12----$>$}\\
\multicolumn{1}{@{}c@{}}{}&
\multicolumn{1}{@{}c@{}}{\rm F.1}&
\multicolumn{1}{@{}c@{}}{\rm Field 2}&
\multicolumn{1}{@{}c@{}}{\rm Field 3}&
\multicolumn{1}{@{}c@{}}{\rm Field 4}&
\multicolumn{1}{@{}c@{}}{\rm Field 5}&
\multicolumn{1}{@{}c@{}}{\rm Field 6}\\
\hline
\multicolumn{7}{|@{}l@{}|}{\tt START POINT} \\
\hline
&V &start-name &\$\$\$\$\$\$\$\$\$\$ &numerical-vl
                &varbl-name          &numerical-vl  \\
&XV&start-name &\$\$\$\$\$\$\$\$\$\$ &numerical-vl
                &varbl-name          &numerical-vl  \\
&ZV&start-name &\$\$\$\$\$\$\$\$\$\$ &              &rl--p-name & \\
&M &start-name &\$\$\$\$\$\$\$\$\$\$ &numerical-vl
                &multp-name          &numerical-vl \\
&XM&start-name &\$\$\$\$\$\$\$\$\$\$ &numerical-vl
                &multp-name          &numerical-vl \\
&ZM&start-name &\$\$\$\$\$\$\$\$\$\$ &              &rl--p-name & \\
&  &start-name &\$\$\$\$\$\$\$\$\$\$ &numerical-vl
                &varbl-name          &numerical-vl \\
&  &start-name &\$\$\$\$\$\$\$\$\$\$ &numerical-vl
                &multp-name          &numerical-vl \\
&X &start-name &\$\$\$\$\$\$\$\$\$\$ &numerical-vl
                &varbl-name          &numerical-vl \\
&X &start-name &\$\$\$\$\$\$\$\$\$\$ &numerical-vl
                &multp-name          &numerical-vl \\
&Z &start-name &\$\$\$\$\$\$\$\$\$\$ &              &rl--p-name & \\
\hline
\multicolumn{1}{@{}c@{}}{~}&
\multicolumn{1}{@{}c@{}}{$\uparrow\uparrow\;$~}&
\multicolumn{1}{@{}c@{}}{$\uparrow\;$~~~~~~~$\;\uparrow$}&
\multicolumn{1}{@{}c@{}}{$\uparrow\;$~~~~~~~$\;\uparrow$}&
\multicolumn{1}{@{}l@{}}{$\uparrow\;$~~~~~~~~~$\;\uparrow$}&
\multicolumn{1}{@{}c@{}}{$\uparrow\;$~~~~~~~$\;\uparrow$}&
\multicolumn{1}{@{}c@{}}{$\uparrow\;$~~~~~~~~~$\;\uparrow$}\\
\multicolumn{1}{@{}c@{}}{~}&
\multicolumn{1}{@{}c@{}}{{\sz 2~3~}~}&
\multicolumn{1}{@{}c@{}}{{\sz 5~}~~~~~~~{\sz 14}}&
\multicolumn{1}{@{}c@{}}{{\sz 15}~~~~~~~{\sz 24}}&
\multicolumn{1}{@{}l@{}}{{\sz 25}~~~~~~~~~{\sz 36}}&
\multicolumn{1}{@{}c@{}}{{\sz 40}~~~~~~~{\sz 49}}&
\multicolumn{1}{@{}c@{}}{{\sz 50}~~~~~~~~~{\sz 61}$\,$}\\
\ecftable{\label{F2.2.13}Possible data cards for {\tt START POINT}}
}}}

The {\tt  V}, {\tt  XV}  and  {\tt ZV} cards
are used to  define the
starting value for variables.  In any  of these cards, the string {\tt
start-name} in data field~2 gives the  name of a  starting  vector and
may be up to ten characters long.  Several different  starting vectors
may be defined in the {\tt START POINT}
section.  The string {\tt \$\$\$\$\$\$\$\$\$\$} in field~3 is used for
two purposes.

\begin{itemize}
\item
It must contain the name of a variable defined in the  {\tt VARIABLES}
section, when a starting value is to be assigned to that variable.  If
field~1 does  not   contain  {\tt    ZV}, the  optional string    {\tt
varbl-name}  in data field 5  may also  contain the  name  of  such a
variable whose starting value is to be assigned.

\item
It may be used  to assign a default  value to all  the starting values
for variables   in a  particular vector.  In  this  case  field~3 must
contain the string {\tt 'DEFAULT'}.

Each starting point   vector    is  given a  default
value  for  every
variable.  This value is initially  zero, but it may  be changed.  The
appropriate  default value applies  to each variable   not  explicitly
specified.
\end{itemize}

The   {\tt M}, {\tt  XM}  and {\tt ZM}   cards
are used  to define the starting value for Lagrange multipliers.
In any of those  cards, the
string {\tt start-name} in data field~2 gives the name of  a starting
vector  and  may be  up  to ten  characters  long.  Several  different
starting vectors may be defined in the {\tt START POINT} section.  The
string {\tt \$\$\$\$\$\$\$\$\$\$} in field~3 is used for two purposes.
\begin{itemize}
\item
It must contain the name of a group defined in the  {\tt GROUPS} section
which is not an objective function group,
when a starting  value is to
be assigned to the corresponding Lagrange multiplier.  If field~1 does
not contain {\tt ZV},  the optional string  {\tt multp-name} in  data
field~5 may also contain the name of such a multiplier  whose starting
value is to be assigned.

\item
It may be  used to assign  a default value to all  the starting values
for Lagrange multipliers in a particular vector.  In this case field~3
must contain the string {\tt 'DEFAULT'}.

Each starting point vector is given a default value for every Lagrange
multiplier.  This value is initially zero, but it may be changed.  The
appropriate default value applies  to  each multiplier not  explicitly
specified.
\end{itemize}

The effect of a card
whose field~1  is blank
or contains  {\tt X} or
{\tt Z} is similar to that of {\tt  V}, {\tt  XV}, {\tt  ZV}, {\tt M},
{\tt XM} and {\tt ZM} cards, except that
\begin{itemize}
\item
variables and Lagrange multipliers  may be mixed up  on   cards  whose
field~3 does not contain {\tt `DEFAULT'},
\item
default values
are assigned to both variables and Lagrange multipliers on card
whose field~3 contains {\tt `DEFAULT'}.  The default value for
both variables and multipliers is initially zero.
\end{itemize}

If  the defaults
are  to be  changed, the change  must be  made on the
first card naming a particular starting point vector.

Starting values for an array
of  variables or Lagrange multipliers may
only be defined on cards whose field~1 begins with  the character {\tt
X} or {\tt Z}; on {\tt X} cards two arrays may be  defined on a single
card.  On such cards, the expanded array  name
in field~3 (and field~5
for {\tt  X} cards)
must be  valid  and the integer indices  must have
been defined in a parameter assignment (see Section~\ref{S2.2.3}).

It remains to specify the numerical value of the default or individual
starting point  as appropriate.
On cards   whose field~1  starts with
{\tt  Z},  the value  is that   previously  associated  with the  real
parameter
{\tt rl--p-name}  or   array
of real  parameters  {\tt
rl--p-name} (respectively)  given in field~5.   On other cards, the
numerical value   is  (or values  are) specified using  up   to twelve
characters  in the string(s)  {\tt numerical-vl} in data field~4 (and
if required field~6).

%{\bf New}
\subsubsection{\label{S2.2.13a}The
{\tt QUADRATIC}, {\tt HESSIAN}, {\tt QUADS}, {\tt QUADOBJ} or {\tt QSECTION}
 Data Cards}

The {\tt QUADRACTIC}, {\tt HESSIAN},
{\tt QUADS}\footnote{This indicator card is included for compatibility with the
proposed MPS format extension of Poncele\'{o}n \cite{Ponc:1990}.},
{\tt QUADOBJ}\footnote{This indicator card was given by Maros and Meszaros \cite{MaroMesa:1999},
for their compatible proposed MPS extension.}
and
{\tt QSECTION}\footnote{This indicator card is included for compatibility
with the OSL \cite{OSLQP:1998} extensions to the MPS format.}
indicator  cards are used interchangeably
to announce any nonzero coefficients ${h_{j,k}}$ in the quadratic
objective group ${\half \sum_{j=1}^n \sum_{k=1}^n h_{j,k} x_j x_k }$.
Only one of each pair ${(h_{j,k}, h_{k,j})}$, ${(j \neq k)}$,
of ``off-diagonal'' terms should be given,
but which is unimportant. Any repeated coefficients will be summed.
The  syntax  for  data  following   these indicator cards
is  given in Figure~\ref{F2.2.13a}.

{\renewcommand{\arraystretch}{0.8}
{\small {\tt
\bcftable{|@{}c@{}|@{}l@{}|@{}l@{}|@{}l@{}|@{}l@{}|@{}l@{}|@{}l@{}|}
\multicolumn{1}{@{}c@{}}{~~}&
\multicolumn{1}{@{}c@{}}{$<$$>$~}&
\multicolumn{1}{@{}c@{}}{$<$---10---$>$}&
\multicolumn{1}{@{}c@{}}{$<$---10---$>$}&
\multicolumn{1}{@{}c@{}}{$<$----12----$>$~~~}&
\multicolumn{1}{@{}c@{}}{$<$---10---$>$}&
\multicolumn{1}{@{}c@{}}{$<$----12----$>$}\\
\multicolumn{1}{@{}c@{}}{}&
\multicolumn{1}{@{}c@{}}{\rm F.1}&
\multicolumn{1}{@{}c@{}}{\rm Field 2}&
\multicolumn{1}{@{}c@{}}{\rm Field 3}&
\multicolumn{1}{@{}c@{}}{\rm Field 4}&
\multicolumn{1}{@{}c@{}}{\rm Field 5}&
\multicolumn{1}{@{}c@{}}{\rm Field 6}\\
\hline
\multicolumn{7}{|@{}l@{}|}{QUADRATIC {\rm or}}  \\
\multicolumn{7}{|@{}l@{}|}{HESSIAN {\rm or}}  \\
\multicolumn{7}{|@{}l@{}|}{QUADS {\rm or}}  \\
\multicolumn{7}{|@{}l@{}|}{QUADOBJ {\rm or}}  \\
\multicolumn{7}{|@{}l@{}|}{QSECTION}  \\
\hline
& &varbl-name &varbl-name &numerical-vl &varbl-name &numerical-vl \\
&X&varbl-name &varbl-name &numerical-vl &varbl-name &numerical-vl \\
&Z&varbl-name &varbl-name &             &r-p-a-name & \\
\hline
\multicolumn{1}{@{}c@{}}{~}&
\multicolumn{1}{@{}c@{}}{$\uparrow\uparrow\;$~}&
\multicolumn{1}{@{}c@{}}{$\uparrow\;$~~~~~~~$\;\uparrow$}&
\multicolumn{1}{@{}c@{}}{$\uparrow\;$~~~~~~~$\;\uparrow$}&
\multicolumn{1}{@{}l@{}}{$\uparrow\;$~~~~~~~~~$\;\uparrow$}&
\multicolumn{1}{@{}c@{}}{$\uparrow\;$~~~~~~~$\;\uparrow$}&
\multicolumn{1}{@{}c@{}}{$\uparrow\;$~~~~~~~~~$\;\uparrow$}\\
\multicolumn{1}{@{}c@{}}{~}&
\multicolumn{1}{@{}c@{}}{{\sz 2~3~}~}&
\multicolumn{1}{@{}c@{}}{{\sz 5~}~~~~~~~{\sz 14}}&
\multicolumn{1}{@{}c@{}}{{\sz 15}~~~~~~~{\sz 24}}&
\multicolumn{1}{@{}l@{}}{{\sz 25}~~~~~~~~~{\sz 36}}&
\multicolumn{1}{@{}c@{}}{{\sz 40}~~~~~~~{\sz 49}}&
\multicolumn{1}{@{}c@{}}{{\sz 50}~~~~~~~~~$\,${\sz 61}}\\
\ecftable{\label{F2.2.13a}Possible data cards for
{\tt HESSIAN}, {\tt QUADS}, {\tt QUADOBJ} or {\tt QSECTION}}
}}}

The strings {\tt varbl-name} in data fields 2 and 3 (and optionally 2 and 5
for those cards whose field 1 does not contain {\tt Z})
give the names of pairs of problem variables ${x_j}$ and
${x_k}$ for which ${h_{j,k}}$ is nonzero.
All problem variables must have been previously set in the
{\tt VARIABLES/COLUMNS} section. Additionally, on a {\tt Z} card,
the name of the variable  must be  an element  of an   array
of variables, with a valid name and index, while on a {\tt X} card,
the name may be either a scalar or an array name.

On cards whose field 1 is either empty or contains the character
{\tt X}, the strings {\tt numerical-vl} in data fields~4 and (optionally) 6
contain the associated numerical values of the coefficients
${h_{j,k}}$. On cards for which field 1 contains the character
{\tt Z}, the string {\tt r-p-a-name} in data field~5 gives a real
parameter array name.  This name must have been previously defined and
its associated value then gives the numerical value of the parameter.


\subsubsection{\label{S2.2.14}The {\tt ELEMENT TYPE} Data Cards}

The {\tt ELEMENT TYPE}
indicator card
is used to announce the data for the different types of
nonlinear elements
to be used. The names of  the elemental
and, optionally, internal variables  and parameters
for  each  element
type are specified   in this  section.   The   syntax  for data  cards
following the indicator card is given in Figure~\ref{F2.2.14}.

{\renewcommand{\arraystretch}{0.8}
{\small {\tt
\bcftable{|@{}c@{}|@{}l@{}|@{}l@{}|@{}l@{}|@{}l@{}|@{}l@{}|@{}l@{}|}
\multicolumn{1}{@{}c@{}}{~~}&
\multicolumn{1}{@{}c@{}}{$<$$>$~}&
\multicolumn{1}{@{}c@{}}{$<$---10---$>$}&
\multicolumn{1}{@{}c@{}}{$<$--6-$>$~~~~~~}&
\multicolumn{1}{@{}c@{}}{~~~~~~~~~~~~~~}&
\multicolumn{1}{@{}c@{}}{$<$--6-$>$~~~~}&
\multicolumn{1}{@{}c@{}}{~~~~~~~~~~~~}\\
\multicolumn{1}{@{}c@{}}{}&
\multicolumn{1}{@{}c@{}}{\rm F.1}&
\multicolumn{1}{@{}c@{}}{\rm Field 2}&
\multicolumn{1}{@{}c@{}}{\rm Field 3}&
\multicolumn{1}{@{}c@{}}{~}&
\multicolumn{1}{@{}c@{}}{\rm Field 5}&
\multicolumn{1}{@{}c@{}}{~}\\
\hline
\multicolumn{7}{|@{}l@{}|}{\tt ELEMENT TYPE} \\
\hline
& {\tt EV} & {\tt etype-name} & {\tt ev-nam} & \mbox{} & {\tt ev-nam} &
\mbox{} \\
& {\tt IV} & {\tt etype-name} & {\tt iv-nam} &  & {\tt iv-nam} & \\
& {\tt EP} & {\tt etype-name} & {\tt ep-nam} &  &  {\tt ep-nam} & \\
\hline
\multicolumn{1}{@{}c@{}}{~}&
\multicolumn{1}{@{}c@{}}{$\uparrow\uparrow\;$~}&
\multicolumn{1}{@{}c@{}}{$\uparrow\;$~~~~~~~$\;\uparrow$}&
\multicolumn{1}{@{}c@{}}{$\uparrow\;$~~~~$\uparrow$~~~~~~}&
\multicolumn{1}{@{}c@{}}{~}&
\multicolumn{1}{@{}c@{}}{$\uparrow\;$~~~~$\uparrow$~~~~}&
\multicolumn{1}{@{}c@{}}{~}\\
\multicolumn{1}{@{}c@{}}{~}&
\multicolumn{1}{@{}c@{}}{{\sz 2~3~}~}&
\multicolumn{1}{@{}c@{}}{{\sz 5~}~~~~~~~{\sz 14}}&
\multicolumn{1}{@{}l@{}}{{\sz 15}~~~~{\sz 20}}&
\multicolumn{1}{@{}c@{}}{~}&
\multicolumn{1}{@{}l@{}}{{\sz 40}~~~~{\sz 45}}&
\multicolumn{1}{@{}c@{}}{~}\\
\ecftable{\label{F2.2.14}Possible data cards for {\tt ELEMENT TYPE}}
}}}

The string in field~1  may be  one of {\tt  EV}, {\tt IV} or {\tt EP}.
This indicates whether the  names of  elemental variables
({\tt EV}), internal variables
({\tt IV})  or elemental parameters
({\tt EP}) are to be specified on the given data card.
If no  cards with the string
{\tt  IV}  in field~1   are found for  a particular  element type, the
element is assumed to have no useful internal  variables;
the internal
variables are then  allocated the  same names
as the elemental
ones.  Likewise, if no cards
with the  string  {\tt EP} in  field~1 are found
for a particular element type,
the element is assumed not to depend on parameter values.

The string  {\tt etype-name} in  data field~2 gives  the  name of the
element  type
under   consideration.  This name    may be   up to ten
characters long.  The data for a particular element must  be specified
on consecutive data cards.

The strings in data  fields 3 and  (optionally)  5  give the  names of
elemental  variables
(field~1  $=$   {\tt   EV}),
internal  variables
(field~1 $=$ {\tt  IV})
or parameters
(field~1  $=$ {\tt EP})
for  the element  type
specified in  field~2.   These strings must  be
valid  Fortran  names (see  Section~\ref{S2.1.2}).  The  names  of the
variables for  different element types  may be the  same; the names of
the elemental variables, internal   variables and  parameters
(if the latter two are given) for a specific element  type must all be
different.

\subsubsection{\label{S2.2.15}The {\tt ELEMENT USES} Data Cards}

The {\tt ELEMENT USES}
indicator card
is used to specify the names and types of the nonlinear
element functions.
The element types may be selected from among those
defined in the {\tt ELEMENT TYPE}
section.
Associations are made between the problem variables
and the elemental variables
for  the  elements used and  parameter
values are assigned.
The syntax  for data  following   this indicator   card
is given   in Figure~\ref{F2.2.15}.

{\renewcommand{\arraystretch}{0.8}
{\small {\tt
\bcftable{|@{}c@{}|@{}l@{}|@{}l@{}|@{}l@{}|@{}l@{}|@{}l@{}|@{}l@{}|}
\multicolumn{1}{@{}c@{}}{~~}&
\multicolumn{1}{@{}c@{}}{$<$$>$~}&
\multicolumn{1}{@{}c@{}}{$<$---10---$>$}&
\multicolumn{1}{@{}c@{}}{$<$--6-$>$~~~~}&
\multicolumn{1}{@{}c@{}}{$<$----12----$>$}&
\multicolumn{1}{@{}c@{}}{$<$--6-$>$~~~~}&
\multicolumn{1}{@{}c@{}}{$<$----12----$>$}\\
\multicolumn{1}{@{}c@{}}{}&
\multicolumn{1}{@{}c@{}}{\rm F.1}&
\multicolumn{1}{@{}c@{}}{\rm Field 2}&
\multicolumn{1}{@{}c@{}}{\rm Field 3}&
\multicolumn{1}{@{}c@{}}{\rm Field 4}&
\multicolumn{1}{@{}c@{}}{\rm Field 5}&
\multicolumn{1}{@{}c@{}}{\rm Field 6}\\
\hline
\multicolumn{7}{|@{}l@{}|}{\tt ELEMENT USES} \\
\hline
&T &\$\$\$\$\$\$\$\$\$\$ &etype-nam &              & &  \\
&XT&\$\$\$\$\$\$\$\$\$\$ &etype-nam &              & &  \\
&V &elmnt-name          &ev-nam    &              &varbl-name  &  \\
&ZV&elmnt-name          &ev-nam    &              &varbl-name  &  \\
&P &elmnt-name          &ep-nam    &numerical-vl
                         &ep-nam    &numerical-vl \\
&XP&elmnt-name          &ep-nam    &numerical-vl
                         &ep-nam    &numerical-vl \\
&ZP&elmnt-name          &ep-nam    &              &r-p-a-nam & \\
\hline
\multicolumn{1}{@{}c@{}}{~}&
\multicolumn{1}{@{}c@{}}{$\uparrow\uparrow\;$~}&
\multicolumn{1}{@{}c@{}}{$\uparrow\;$~~~~~~~$\;\uparrow$}&
\multicolumn{1}{@{}c@{}}{$\uparrow$~~~~$\uparrow$~~~~$\uparrow$}&
\multicolumn{1}{@{}l@{}}{$\uparrow$~~~~~~~~~~$\uparrow$~~~}&
\multicolumn{1}{@{}c@{}}{$\uparrow\;$~~~~$\uparrow$~~~$\;\uparrow$}&
\multicolumn{1}{@{}c@{}}{$\uparrow\;$~~~~~~~~~$\;\uparrow$}\\
\multicolumn{1}{@{}c@{}}{~}&
\multicolumn{1}{@{}c@{}}{{\sz 2~3~}~}&
\multicolumn{1}{@{}c@{}}{{\sz 5~}~~~~~~~{\sz 14}}&
\multicolumn{1}{@{}c@{}}{{\sz 15}~~~{\sz 20}~~~{\sz 24}}&
\multicolumn{1}{@{}l@{}}{{\sz 25}~~~~~~~~~{\sz 36}~~~}&
\multicolumn{1}{@{}c@{}}{{\sz 40}~~~~{\sz 45}~~{\sz 49}}&
\multicolumn{1}{@{}c@{}}{{\sz 50}~~~~~~~~~{\sz 61}$\,$}\\
\ecftable{\label{F2.2.15}Possible data cards for {\tt ELEMENT USES}}
}}}

There are three sorts of data cards
in the {\tt ELEMENT USES}
section.  For cards of the second  and  third  kinds, the  string {\tt
elmnt-name} in data field~2 gives the name, or an array
of names, of a
nonlinear element function.
This  name  may be up  to  ten characters
long and each nonlinear element name must  be unique.  On  array
cards
(those prefixed  by {\tt X}  or {\tt Z}), the  expanded element  array
name in field~2 must be  valid and the integer  indices must have been
defined in a parameter assignment (see Section~\ref{S2.2.3}).

The first kind of cards, identified by the characters {\tt  T} or {\tt
XT}
in field~1, give the name, or an array
of names, of an element and its
type.  On such cards,
the string {\tt \$\$\$\$\$\$\$\$\$\$} in field~2
is used for two purposes.
\begin{itemize}
\item
It may contain the name, or an array of
names, of a  nonlinear element
function whose  type is to be defined.   This  name may  be  up to ten
characters  long and each nonlinear element  name
must be unique.  On
{\tt  XT}
cards  the expanded element  array name
in  field~2 must be
valid and the integer  indices must have  been  defined in a parameter
assignment (see Section~\ref{S2.2.3}).

\item
It may be used to assign a  default type
to all the  nonlinear element
functions.   In   this case field~2    must contain   the string  {\tt
`DEFAULT'}.  Any element not explicitly  typed
is assumed to belong to
a default type.  If a default is to be used, it must be specified on a
card
and such a card may only appear before the first  {\tt T} or {\tt
XT}    card in the {\tt  ELEMENT   USES}
section.   The  string   {\tt
etype-name} in data field~3 gives the name of the  element
type to be
used.   This  name may be up  to  ten characters   long and must  have
appeared in the {\tt ELEMENT TYPE} section.
\end{itemize}

The second kind of data card, identified by the  characters {\tt V} or
{\tt  ZV}
in  field~1, is used  to  assign problem variables
to   the elemental  variables
appropriate for the  element type.   On this data
card,
the string  {\tt ev-nam} in data  field~3 gives the name of one
of the elemental variables for the given element type.
This name must
have been set in the {\tt ELEMENT TYPE}
section and be a valid Fortran
name (see Section~\ref{S2.1.2}).  The string {\tt varbl-name} in data
field~5 then gives the name of the problem variable
that is  to be assigned to  the specified elemental variable.
The name   of  this   variable   may  have  been    set in  the   {\tt
VARIABLES/COLUMNS}
section  or may be  a new  variable (often known  as  a {\em nonlinear
variable}) introduced here and can  be up to ten  characters long.  On
a {\tt ZV}
card,
the name of the variable  must be  an element  of an   array
of variables, with a valid name and index.

The last kind of data card,
identified by the characters {\tt P}, {\tt
XP} or {\tt ZP}
in field~1, is used  to assign numerical values to the
parameters
for the  element functions   ({\tt P}) or  array
of element
functions ({\tt XP} and {\tt ZP}). On this data card,
the string {\tt ep-nam} in data field  3 (and, for  {\tt P} and {\tt
XP}  cards, optionally 5) must give  the name of a parameter.
This  name must have  been set  in the {\tt ELEMENT TYPE}  section and be a
valid Fortran  name,
see Section~\ref{S2.1.2}.  On  {\tt   P} and {\tt XP}
cards,  the strings  {\tt      numerical-vl} in  data  fields~4    and
(optionally) 6 contain  the  numerical value  of the parameter.  These
values may each occupy up to 12 locations within their field.  On {\tt
ZP}  cards,
the  string {\tt r-p-a-name}  in data field~5 gives a real
parameter array name.
This  name must have  been  previously  defined  and its
associated value then gives the numerical value of the parameter.

\subsubsection{\label{S2.2.16}The {\tt GROUP TYPE} Data Cards}

The {\tt GROUP TYPE}
indicator card
is used to announce  the data for
the different types of nontrivial  groups which are to  be used.   The
names of the group-type variable and, optionally, of  group parameters
for each group type  are  specified in this  section.  The  syntax for
data   cards
following    the   indicator    card
is given in Figure~\ref{S2.2.16}.

{\renewcommand{\arraystretch}{0.8}
{\small {\tt
\bcftable{|@{}c@{}|@{}l@{}|@{}l@{}|@{}l@{}|@{}l@{}|@{}l@{}|@{}l@{}|}
\multicolumn{1}{@{}c@{}}{~~}&
\multicolumn{1}{@{}c@{}}{$<$$>$~}&
\multicolumn{1}{@{}c@{}}{$<$---10---$>$}&
\multicolumn{1}{@{}c@{}}{$<$--6-$>$~~~~~~}&
\multicolumn{1}{@{}c@{}}{~~~~~~~~~~~~~~}&
\multicolumn{1}{@{}c@{}}{$<$--6-$>$~~~~}&
\multicolumn{1}{@{}c@{}}{~~~~~~~~~~~~}\\
\multicolumn{1}{@{}c@{}}{}&
\multicolumn{1}{@{}c@{}}{\rm F.1}&
\multicolumn{1}{@{}c@{}}{\rm Field 2}&
\multicolumn{1}{@{}c@{}}{\rm Field 3}&
\multicolumn{1}{@{}c@{}}{~}&
\multicolumn{1}{@{}c@{}}{\rm Field 5}&
\multicolumn{1}{@{}c@{}}{~}\\
\hline
\multicolumn{7}{|@{}l@{}|}{\tt GROUP TYPE} \\
\hline
& GV &gtype-name &gv-nam &          &        &  \\
& GP &gtype-name &gp-nam &          &gp-nam & \\
\hline
\multicolumn{1}{@{}c@{}}{~}&
\multicolumn{1}{@{}c@{}}{$\uparrow\uparrow\;$~}&
\multicolumn{1}{@{}c@{}}{$\uparrow\;$~~~~~~~$\;\uparrow$}&
\multicolumn{1}{@{}l@{}}{$\uparrow\;$~~~~$\uparrow$}&
\multicolumn{1}{@{}c@{}}{~}&
\multicolumn{1}{@{}l@{}}{$\uparrow\;$~~~~$\uparrow$}&
\multicolumn{1}{@{}c@{}}{~}\\
\multicolumn{1}{@{}c@{}}{~}&
\multicolumn{1}{@{}c@{}}{{\sz 2~3~}~}&
\multicolumn{1}{@{}c@{}}{{\sz 5~}~~~~~~~{\sz 14}}&
\multicolumn{1}{@{}l@{}}{{\sz 15}~~~~{\sz 20}}&
\multicolumn{1}{@{}c@{}}{~}&
\multicolumn{1}{@{}l@{}}{{\sz 40}~~~~{\sz 45}}&
\multicolumn{1}{@{}c@{}}{~}\\
\ecftable{\label{F2.2.16}Possible data cards for {\tt GROUP TYPE}}
}}}

The  string  in field~1 may  be either {\tt  GV}   or {\tt  GP}.
This indicates whether the name of a group-type variable ({\tt GV}) or
one or more group parameters
({\tt GP})  are to  be specified on  the
given data   card.
The  data  for a particular  group type  must be
specified on consecutive data  cards.  The string {\tt  gtype-name} in
data field~2 gives the name  of a nontrivial group type  and may be up
to  ten characters long.  If data  field~1 holds {\tt  GV}, the string
{\tt gv-nam} in  data field~3  then gives the  name of the  group-type
variable for this group type.  This string  may be  up to 6 characters
long.   This   string    must    be a valid    Fortran
name (see Section~\ref{S2.1.2}).    The  names of  the   variables for
different group types may be the same.  Alternatively, if data field~1
holds {\tt  GP},
the strings  {\tt  gp-nam} in  data  fields  3   and
(optionally) 5 give the names of parameters
for the group type.  These
strings must again  be valid Fortran
names.  The  names of parameters
for different group types
may be the same; the names of the group-type
variable and  parameters
(if the latter appear)  for a specific group type must all be different.

\subsubsection{\label{S2.2.17}The {\tt GROUP USES} Data Cards}

The {\tt GROUP USES}
indicator card
is  used to announce which  of  the nonlinear  elements
appear in each  group and the type   of group function  involved.  The
group types may be selected from among those defined in the {\tt GROUP
TYPE}
section of the data while the elements
may be  selected from among the
types defined in the {\tt ELEMENT USES}
section.  In addition, group    parameter
values are   assigned.   The
syntax for   data  following this indicator  card
is  given  in Figure~\ref{F2.2.17}.

{\renewcommand{\arraystretch}{0.8}
{\small {\tt
\bcftable{|@{}c@{}|@{}l@{}|@{}l@{}|@{}l@{}|@{}l@{}|@{}l@{}|@{}l@{}|}
\multicolumn{1}{@{}c@{}}{~~}&
\multicolumn{1}{@{}c@{}}{$<$$>$~}&
\multicolumn{1}{@{}c@{}}{$<$---10---$>$}&
\multicolumn{1}{@{}c@{}}{$<$--6-$>$~~~~}&
\multicolumn{1}{@{}c@{}}{$<$----12----$>$}&
\multicolumn{1}{@{}c@{}}{$<$--6-$>$~~~~}&
\multicolumn{1}{@{}c@{}}{$<$----12----$>$}\\
\multicolumn{1}{@{}c@{}}{}&
\multicolumn{1}{@{}c@{}}{\rm F.1}&
\multicolumn{1}{@{}c@{}}{\rm Field 2}&
\multicolumn{1}{@{}c@{}}{\rm Field 3}&
\multicolumn{1}{@{}c@{}}{\rm Field 4}&
\multicolumn{1}{@{}c@{}}{\rm Field 5}&
\multicolumn{1}{@{}c@{}}{\rm Field 6}\\
\hline
\multicolumn{7}{|@{}l@{}|}{\tt GROUP USES} \\
\hline
&T &group-name &gtype-nam &              &              &  \\
&XT&group-name &gtype-nam &              &              &  \\
&E &group-name &elmnt-nam &blank/num-vl &elmnt-name   &blank/num-vl \\
&XE&group-name &elmnt-nam &blank/num-vl &elmnt-name   &blank/num-vl \\
&ZE&group-name &elmnt-nam &              &r-p-a-name &  \\
&P &group-name &gp-nam    &numerical-vl &gp-nam       &numerical-vl \\
&XP&group-name &gp-nam    &numerical-vl &gp-nam       &numerical-vl \\
&ZP&group-name &gp-nam    &              &r-p-a-name &  \\
\hline
\multicolumn{1}{@{}c@{}}{~}&
\multicolumn{1}{@{}c@{}}{$\uparrow\uparrow\;$~}&
\multicolumn{1}{@{}c@{}}{$\uparrow\;$~~~~~~~$\;\uparrow$}&
\multicolumn{1}{@{}c@{}}{$\uparrow\;$~~~~$\uparrow$~~~$\;\uparrow$}&
\multicolumn{1}{@{}l@{}}{$\uparrow$~~~~~~~~~~$\uparrow$}~~~&
\multicolumn{1}{@{}c@{}}{$\uparrow\;$~~~~$\uparrow\;$~~~$\uparrow$}&
\multicolumn{1}{@{}l@{}}{$\uparrow$~~~~~~~~~~$\uparrow$}\\
\multicolumn{1}{@{}c@{}}{~}&
\multicolumn{1}{@{}c@{}}{{\sz 2~3~}~}&
\multicolumn{1}{@{}c@{}}{{\sz 5~}~~~~~~~{\sz 14}}&
\multicolumn{1}{@{}c@{}}{{\sz 15}~~~~{\sz 20}~~{\sz 24}}&
\multicolumn{1}{@{}l@{}}{{\sz 25}~~~~~~~~~$\,${\sz 36}~~~}&
\multicolumn{1}{@{}c@{}}{{\sz 40}~~~$\,${\sz 45}~~$\,${\sz 49}}&
\multicolumn{1}{@{}c@{}}{{\sz 50}~~~~~~~~~$\,${\sz 61}}\\
\ecftable{\label{F2.2.17}Possible data cards for {\tt GROUP USES}}
}}}

There are three sorts of data  cards
in the  {\tt GROUP USES}
section.  For  cards of the second  and  third  kinds, the string {\tt
group-name} in data field~2 gives the  name, or an  array
of names, of the group(s) (or row(s) or constraint(s))
under  consideration.  The name
may be up
to ten  characters  long   and  must   have been defined   in the {\tt
GROUPS/ROWS/CONSTRAINTS} section.

On array
cards
(those prefixed by  {\tt  X} or {\tt Z}), the expanded element array name
in field~2  must be  valid and the integer indices
must  have   been     defined  in    a   parameter  assignment    (see
Section~\ref{S2.2.3}).

The first kind of cards,
identified by the  characters {\tt T} or {\tt XT}
in  field~1,  give the name,  or  an array
of   names, of a  group
function  and   its   type.
On   such cards,   the    string  {\tt
\$\$\$\$\$\$\$\$\$\$} in field~2 is used for two purposes.
\begin{itemize}
\item
It may contain  the name,  or an array of names,  of a  group function
whose type is to be defined.   The name  may be  up  to ten characters
long and  must have been  defined in the {\tt GROUPS/ROWS/CONSTRAINTS}
section.  On {\tt XT}
cards
the expanded element array name
in field~2
must  be valid and  the integer  indices  must have been  defined in a
parameter assignment (see Section~\ref{S2.2.3}).

\item
It may be used  to assign a  default type  to all the group functions.
In this case field~2 must contain the string  {\tt `DEFAULT'}.  Such a
card
may only appear before the first {\tt T} or {\tt  XT} card in the
{\tt GROUP USES}
section.  Any  group not explicitly typed  is assumed
to belong to a default type.
The initial default type  is trivial but
the  default may be changed.   The string {\tt gtype-name}  in data
field~3 gives the name of the group type to be used.  This name may be
up to ten characters  long and  must have appeared  in the  {\tt GROUP
TYPE} section.
\end{itemize}

The second kind  of data card,
identified by the characters {\tt  E},
{\tt  XE} or   {\tt ZE}
in  field~1, is  an indication that particular
nonlinear elements a
re to be included in a given group. Optionally the
given elements may be multiplied by specified weights.
On these data cards,
the string {\tt elmnt-name} in data fields 3 (and optionally 5
on {\tt E} and  {\tt XE} cards) hold  the names of nonlinear  elements
to be used.   The names in both  fields  may  be up to ten
characters long and  must have been defined  in the {\tt ELEMENT} {\tt
USES} section.  On  {\tt  XE}  and {\tt ZE}  cards, the  names  of the
nonlinear  elements must  be   components  of an   array
of nonlinear
elements, with a valid name
and index.  The elements are multiplied by
given weights. By  default,
each  weight takes  the  value 1.0.  Only
non-unit weights need to be specified explicitly.  On {\tt E} and {\tt
XE} cards,
the weights are assigned the  numerical values specified in
data fields  4  (and optionally 6). These  values may occupy up  to 12
locations of their specified field.  The default value of 1.0
is taken
whenever these fields are empty.   On {\tt ZE}  cards, the string {\tt
r-p-a-name} in  data field~5 gives  a real parameter   array name.
This name  must have been previously defined  and its associated value
then gives the numerical value of  the weight.  Any  group that is not
named  on an {\tt E} or  {\tt XE} card
is taken  to have no nonlinear elements.

The last kind of data card,
identified by the characters {\tt P}, {\tt XP} or {\tt ZP}
in field~1, is used to  assign numerical values to the
parameters  for  the group
functions ({\tt   P})  or  array
of  group
functions ({\tt XP} and {\tt ZP}). On this data card,
the strings {\tt gp-nam} in data  fields~3 (and, for {\tt P}  and {\tt
XP}  cards, optionally 5)  give the names  of  parameters.
These names
must have been set in the {\tt GROUP TYPE}
section and be valid Fortran  name,
see
Section~\ref{S2.1.2}.  On {\tt P} and {\tt XP} cards, the strings {\tt
numerical-vl} in  data   fields 4 and  (optionally)  6  contain the
numerical value of the parameter.  These values may  each occupy up to
12  locations  of their  field.  On {\tt  ZP} cards, the  string  {\tt
r-p-a-name} in data field~5 gives   a  real  parameter
array  name.
This name must have been  previously defined  and its associated value
then gives the numerical value of the parameter.

The {\tt T} or {\tt XT} card
for a particular group
must appear before its {\tt E}, {\tt XE}, {\tt ZE}, {\tt  P}, {\tt XP}
or {\tt ZP} cards.

\subsubsection{\label{S2.2.18}The {\tt OBJECT BOUND} Data Cards}

The {\tt OBJECT BOUND}
indicator card
is used to announce  known lower
and upper bounds  on  the value  of  the objective   function
for  the
problem.  The syntax for data following this  indicator card
is given in Figure~\ref{F2.2.18}.

{\renewcommand{\arraystretch}{0.8}
{\small {\tt
\bcftable{|@{}c@{}|@{}l@{}|@{}l@{}|@{}l@{}|@{}l@{}|@{}l@{}|@{}l@{}|}
\multicolumn{1}{@{}c@{}}{~~}&
\multicolumn{1}{@{}c@{}}{$<$$>$~}&
\multicolumn{1}{@{}c@{}}{$<$---10---$>$}&
\multicolumn{1}{@{}c@{}}{~~~~~~~~~~}&
\multicolumn{1}{@{}c@{}}{$<$----12----$>$~~~}&
\multicolumn{1}{@{}c@{}}{$<$---10---$>$}&
\multicolumn{1}{@{}c@{}}{~~~~~~~~~~~~~~~}\\
\multicolumn{1}{@{}c@{}}{}&
\multicolumn{1}{@{}c@{}}{\rm F.1}&
\multicolumn{1}{@{}c@{}}{\rm Field 2}&
\multicolumn{1}{@{}c@{}}{~}&
\multicolumn{1}{@{}c@{}}{\rm Field 4}&
\multicolumn{1}{@{}c@{}}{\rm Field 5}&
\multicolumn{1}{@{}c@{}}{~}\\
\hline
\multicolumn{7}{|@{}l@{}|}{\tt OBJECT BOUND} \\
\hline
&LO&obbnd-name & &numerical-vl &           & \\
&UP&obbnd-name & &numerical-vl &           & \\
&XL&obbnd-name & &numerical-vl &           & \\
&XU&obbnd-name & &numerical-vl &           & \\
&ZL&obbnd-name & &             &rl--p-name & \\
&ZU&obbnd-name & &             &rl--p-name & \\
\hline
\multicolumn{1}{@{}c@{}}{~}&
\multicolumn{1}{@{}c@{}}{$\uparrow \uparrow\;$~}&
\multicolumn{1}{@{}c@{}}{$\uparrow\;$~~~~~~~$\;\uparrow$}&
\multicolumn{1}{@{}c@{}}{~}&
\multicolumn{1}{@{}l@{}}{$\uparrow\;$~~~~~~~~~$\;\uparrow$}&
\multicolumn{1}{@{}c@{}}{$\uparrow\;$~~~~~~~$\;\uparrow$}&
\multicolumn{1}{@{}c@{}}{~}\\
\multicolumn{1}{@{}c@{}}{~}&
\multicolumn{1}{@{}c@{}}{{\sz 2~3~}~}&
\multicolumn{1}{@{}c@{}}{{\sz 5~}~~~~~~~{\sz 14}}&
\multicolumn{1}{@{}c@{}}{~}&
\multicolumn{1}{@{}l@{}}{{\sz 25}~~~~~~~~~{\sz 36}}&
\multicolumn{1}{@{}c@{}}{{\sz 40}~~~~~~~{\sz 49}$\,$}&
\multicolumn{1}{@{}c@{}}{~}\\
\ecftable{\label{F2.2.18}Possible data cards for {\tt OBJECT BOUND}}
}}}

The two-character string in data  field~1 specifies  the type of bound
to  be input. Possible values  are: {\tt LO}, {\tt XL}   or {\tt ZL} for a
lower bound,  and {\tt UP}, {\tt XU}  or {\tt ZU}  for an upper bound.
The string {\tt obbnd-name} in data field~2 gives a name to the bounds
under consideration.   This name may  be  up  to ten characters  long.
Several different  known  bounds  on the objective  function
may  be defined in the {\tt OBJECT BOUND} section.

For bounds of type {\tt LO}  or  {\tt  UP},
the   numerical value  of   the  bound  is  given as the  string  {\tt
numerical\--\-vl}  using at most  12 characters in  data field~4.  For
bounds of type {\tt ZL} or {\tt ZU},
the numerical  value of the bound
is  that  previously  associated with  the real  parameter  array
{\tt r-p-a-name} specified in field~5.  When both lower and upper
bounds
on the objective are known, they must be specified on separate cards.

The  objective function
is  assumed by  default to be  unbounded both
below  and above.  The values  for  each named bound set  may  only be
changed on a {\tt LO}, {\tt UP}, {\tt XL}, {\tt XU}, {\tt  ZL} or {\tt
ZU} card.

\subsection{\label{S2.3}Another Example}

In  Section~\ref{S1.4},  we  gave an  example. An SDIF  file for  this
example is given in Figure~\ref{F2.3.1}. The problem is given the name
{\tt DOC}.  The groups
are referred to  as  {\tt GROUP1/2/3}  and  the
variables  are {\tt X1/2/3.} The vector  of bounds
is called {\tt BN1}
and the  two types of  nonlinear   element
are {\tt  ELEMENT1/2}. The
elemental variables
are assigned names beginning with {\tt U}  and the
internal  variables
for the second  nonlinear  element start with {\tt
V}.  The two group types
are  {\tt GTYPE1/2}.  Finally the nonlinear
element in {\tt GROUP2} is given the name  {\tt G2E1},  while those in
{\tt GROUP3} are {\tt G3E1/2}.


\subsection{\label{S2.4}A Further Example}

In  Section~\ref{S1.5}, we gave  a  second  example.    Because of its
repetitious structure,
this example is well suited  to use array names
and do-loops.    An    SDIF  file for this   example   is   given in
Figure~\ref{F2.4.1}. The problem  is given the name {\tt   DOC2}.  The
variables  are  referred to  as {\tt X1},\ldots,  {\tt  X1000} and the
groups
are {\tt G1},  \ldots, {\tt G1000}.    The vector of bounds
is called {\tt BND},
the constants  are {\tt CONST} and  the single nonlinear
element type
is {\tt SQUARE}, with  elemental variable
{\tt V}.  Note
that  the   {\tt  BND} section  is  necessary  since the variables are
unrestricted and we must override the default lower bounds of zero and
upper bounds of infinity.
 The nonlinear elements are
given the names
{\tt E1},\ldots, {\tt E1000}.  Finally, the single group type
is {\tt
SINE} with group-type variable
{\tt ALPHA} and parameter
{\tt P}.

{\renewcommand{\arraystretch}{0.8}
{\small {\tt
\bcftable{|@{}c@{}|@{}l@{}|@{}l@{}|@{}l@{}|@{}l@{}|@{}l@{}|@{}l@{}|}
\multicolumn{1}{@{}c@{}}{~~}&
\multicolumn{1}{@{}c@{}}{$<$$>$~}&
\multicolumn{1}{@{}c@{}}{$<$---10---$>$}&
\multicolumn{1}{@{}c@{}}{$<$---10---$>$}&
\multicolumn{1}{@{}c@{}}{$<$----12----$>$~~~}&
\multicolumn{1}{@{}c@{}}{$<$---10---$>$}&
\multicolumn{1}{@{}c@{}}{$<$----12----$>$}\\
\multicolumn{1}{@{}c@{}}{}&
\multicolumn{1}{@{}c@{}}{\rm F.1}&
\multicolumn{1}{@{}c@{}}{\rm Field 2}&
\multicolumn{1}{@{}c@{}}{\rm Field 3}&
\multicolumn{1}{@{}c@{}}{\rm Field 4}&
\multicolumn{1}{@{}c@{}}{\rm Field 5}&
\multicolumn{1}{@{}c@{}}{\rm Field 6}\\
\hline
\multicolumn{3}{|@{}l@{}|@{}}{NAME}&DOC&&&\\
\multicolumn{3}{|@{}l@{}|@{}}{GROUPS}&&&&\\
&E &GROUP1 &       &         & & \\
&E &GROUP2 &       &         & & \\
&E &GROUP3 &       &         & & \\
\multicolumn{3}{|@{}l@{}|@{}}{VARIABLES}&&&&\\
&  &X1     &GROUP1 & 1.0     & & \\
&  &X2     &GROUP3 & 1.0     & & \\
&  &X3     &       &         & & \\
\multicolumn{3}{|@{}l@{}|@{}}{BOUNDS}&&&&\\
&FR&BN1    &X1     &         & & \\
&LO&BN1    &X2     & -1.0D+0 & & \\
&LO&BN1    &X3     & 1.0D+0  & & \\
&UP&BN1    &X2     & 1.0D+0  & & \\
&UP&BN1    &X3     & 2.0D+0  & & \\
\multicolumn{3}{|@{}l@{}|@{}}{ELEMENT TYPE}&&&&\\
&EV&ETYPE1 &V1     &   &   & \\
&EV&ETYPE1 &V2     &   &   & \\
&EV&ETYPE2 &V1     &   &   & \\
&EV&ETYPE2 &V2     &   &   & \\
&EV&ETYPE2 &V3     &   &   & \\
&IV&ETYPE2 &U1     &   &   & \\
&IV&ETYPE2 &U2     &   &   & \\
\multicolumn{3}{|@{}l@{}|@{}}{ELEMENT USES}&&&&\\
&T &G2E1 &ETYPE1   &   &   & \\
&V &G2E1 &V1       &   &X2 & \\
&V &G2E1 &V2       &   &X3 & \\
&T &G3E1 &ETYPE2   &   &   & \\
&V &G3E1 &V1       &   &X2 & \\
&V &G3E1 &V2       &   &X1 & \\
&V &G3E1 &V3       &   &X3 & \\
&T &G3E2 &ETYPE1   &   &   & \\
&V &G3E2 &V1       &   &X1 & \\
&V &G3E2 &V2       &   &X3 & \\
\multicolumn{3}{|@{}l@{}|@{}}{GROUP TYPE}&&&&\\
&GV&GTYPE1 &ALPHA  & & & \\
&GV&GTYPE2 &ALPHA  & & & \\
\multicolumn{3}{|@{}l@{}|@{}}{GROUP USES}&&&&\\
&T &GROUP1 &GTYPE1 & & & \\
&T &GROUP2 &GTYPE2 & & & \\
&E &GROUP2 &G2E1   & & & \\
&E &GROUP3 &G3E1   & & & \\
&E &GROUP3 &G3E2   & & & \\
\multicolumn{3}{|@{}l@{}|@{}}{ENDATA}&&&&\\
\hline
\multicolumn{1}{@{}c@{}}{~}&
\multicolumn{1}{@{}c@{}}{$\uparrow \uparrow\;$~}&
\multicolumn{1}{@{}c@{}}{$\uparrow\;$~~~~~~~$\;\uparrow$}&
\multicolumn{1}{@{}c@{}}{$\uparrow\;$~~~~~~~$\;\uparrow$}&
\multicolumn{1}{@{}l@{}}{$\uparrow\;$~~~~~~~~~$\;\uparrow$}&
\multicolumn{1}{@{}c@{}}{$\uparrow\;$~~~~~~~$\;\uparrow$}&
\multicolumn{1}{@{}c@{}}{$\uparrow\;$~~~~~~~~~$\;\uparrow$}\\
\multicolumn{1}{@{}c@{}}{~}&
\multicolumn{1}{@{}c@{}}{{\sz 2~3~}~}&
\multicolumn{1}{@{}c@{}}{{\sz 5~}~~~~~~~{\sz 14}}&
\multicolumn{1}{@{}c@{}}{{\sz 15}~~~~~~~{\sz 24}}&
\multicolumn{1}{@{}l@{}}{{\sz 25}~~~~~~~~~{\sz 36}}&
\multicolumn{1}{@{}c@{}}{{\sz 40}~~~~~~~{\sz 49}}&
\multicolumn{1}{@{}c@{}}{{\sz 50}~~~~~~~~~{\sz 61}}\\
\ecftable{\label{F2.3.1}SDIF file for the example of
Section~\protect\ref{S1.4}}
}}}

\clearpage

{\renewcommand{\arraystretch}{0.8}
{\small {\tt
\bcftable{|@{}c@{}|@{}l@{}|@{}l@{}|@{}l@{}|@{}l@{}|@{}l@{}|@{}l@{}|}
\multicolumn{1}{@{}c@{}}{~~}&
\multicolumn{1}{@{}c@{}}{$<$$>$~}&
\multicolumn{1}{@{}c@{}}{$<$---10---$>$}&
\multicolumn{1}{@{}c@{}}{$<$---10---$>$}&
\multicolumn{1}{@{}c@{}}{$<$----12----$>$~~~}&
\multicolumn{1}{@{}c@{}}{$<$---10---$>$}&
\multicolumn{1}{@{}c@{}}{$<$----12----$>$}\\
\multicolumn{1}{@{}c@{}}{}&
\multicolumn{1}{@{}c@{}}{\rm F.1}&
\multicolumn{1}{@{}c@{}}{\rm Field 2}&
\multicolumn{1}{@{}c@{}}{\rm Field 3}&
\multicolumn{1}{@{}c@{}}{\rm Field 4}&
\multicolumn{1}{@{}c@{}}{\rm Field 5}&
\multicolumn{1}{@{}c@{}}{\rm Field 6}\\
\hline
\multicolumn{3}{|@{}l@{}|@{}}{NAME}&DOC2&&&\\
&IE&ONE   &          & 1    &    & \\
&IE&N     &          & 1000 &    & \\
&IA&NM1   &N         & -1   &    & \\
\multicolumn{3}{|@{}l@{}|@{}}{VARIABLES}&&&&\\
&DO&I     &ONE       &      &N   & \\
&  &X     &X(I)      &      &    & \\
&ND&      &          &      &    & \\
\multicolumn{3}{|@{}l@{}|@{}}{GROUPS}&&&&\\
&DO&I     &ONE       &      &NM1 & \\
&XN&G(I)  &X(ONE)    & 1.0  &    & \\
&ND&      &          &      &    & \\
&XN&G(N)  &          &      &    & \\
\multicolumn{3}{|@{}l@{}|@{}}{CONSTANTS}&&&&\\
&  &CONST &`DEFAULT' & 1.0  &    & \\
&X &CONST &G(N)      & 0.0  &    & \\
\multicolumn{3}{|@{}l@{}|@{}}{BOUNDS}&&&&\\
&FR&BND   &`DEFAULT' &      &    & \\
\multicolumn{3}{|@{}l@{}|@{}}{ELEMENT TYPE}&&&&\\
&EV&SQUARE&V         &      &    & \\
\multicolumn{3}{|@{}l@{}|@{}}{ELEMENT USES}&&&&\\
&DO&I     &ONE       &      &N   & \\
&XT&E(I)  &SQUARE    &      &    & \\
&ZV&E(I)  &V         &      &X(I)& \\
&ND&      &          &      &    & \\
\multicolumn{3}{|@{}l@{}|@{}}{GROUP TYPE}&&&&\\
&GV&SINE  &ALPHA     &      &    & \\
&GP&SINE  &P         &      &    & \\
\multicolumn{3}{|@{}l@{}|@{}}{GROUP USES}&&&&\\
&DO&I     &ONE       &      &NM1 & \\
&XT&G(I)  &SINE      &      &    & \\
&XE&G(I)  &E(I)      &      &E(N)& \\
&XP&G(I)  &P         & 1.0  &    & \\
&ND&      &          &      &    & \\
&E &G1000 &E1000     &      &    & \\
&P &G1000 &P         & 0.5  &    & \\
\multicolumn{3}{|@{}l@{}|@{}}{ENDATA}&&&&\\
\hline
\multicolumn{1}{@{}c@{}}{~}&
\multicolumn{1}{@{}c@{}}{$\uparrow \uparrow\;$~}&
\multicolumn{1}{@{}c@{}}{$\uparrow\;$~~~~~~~$\;\uparrow$}&
\multicolumn{1}{@{}c@{}}{$\uparrow\;$~~~~~~~$\;\uparrow$}&
\multicolumn{1}{@{}l@{}}{$\uparrow\;$~~~~~~~~~$\;\uparrow$}&
\multicolumn{1}{@{}c@{}}{$\uparrow\;$~~~~~~~$\;\uparrow$}&
\multicolumn{1}{@{}c@{}}{$\uparrow\;$~~~~~~~~~$\;\uparrow$}\\
\multicolumn{1}{@{}c@{}}{~}&
\multicolumn{1}{@{}c@{}}{{\sz 2~3~}~}&
\multicolumn{1}{@{}c@{}}{{\sz 5~}~~~~~~~{\sz 14}}&
\multicolumn{1}{@{}c@{}}{{\sz 15}~~~~~~~{\sz 24}}&
\multicolumn{1}{@{}l@{}}{{\sz 25}~~~~~~~~~{\sz 36}}&
\multicolumn{1}{@{}c@{}}{{\sz 40}~~~~~~~{\sz 49}}&
\multicolumn{1}{@{}c@{}}{{\sz 50}~~~~~~~~~{\sz 61}}\\
\ecftable{\label{F2.4.1}SDIF file for the example of
Section~\protect\ref{S1.5}}
}}}



\section[The SIF for Nonlinear Elements]{\label{S3}The Standard Input Format for
\protect\\ Nonlinear Elements}
\setcounter{figure}{0}

In  addition  to  the problem data  described in Section~\ref{S2}, the
user might also wish to  specify the nonlinear  element
functions, and their derivatives,
in   a systematic way.    A  particular  nonlinear
element function is defined in terms  of its problem variables
and its type;
both of these quantities are  specified in Section~\ref{S2}. Thus, the
only details   which   remain to  be  specified are the   function and
derivative values of the {\em  element types} and the  transformations
between elemental and internal variables, if any.

In  this section, we present one  approach to  this issue.  As before,
data is specified in a file. The file comprises  an ordered mixture of
indicator  and data  cards;
the latter  allow  function and derivative
definitions in appropriate high-level language statements.

\subsection[Introduction to the Standard Element Type \protect\\ Input
 Format]{\label{S3.1}Introduction to the Standard Element Type \protect\\ Input
 Format}

\subsubsection{\label{S3.1.1}The Values and Derivatives Required}

It is assumed that a nonlinear element type
is  specified in  terms of internal variables
{\bf u},  whose names
are those  given on the {\tt ELEMENT TYPE}
data  cards
in an SDIF file  (if   the element has  no useful internal
variables, the internal  and elemental
variables are  the same and the
internal variables will  have  been named  after  the elementals), see
Section~\ref{S2.2.14}.  An optimization procedure is likely to require
the  values of the   element functions
and possibly their first and second, derivatives.
These derivatives need only be given with respect
to the internal variables.
For if we denote the gradient
and Hessian
matrix of an element function
$f$ with respect to $u$ by
\bdmath
\nabla_u f \tim{and} \nabla_{uu} f
\edmath
respectively, the  gradient and  Hessian
matrices with  respect to the elemental variables
are
\bdmath
W^T \nabla_u f \tim{and} W^T \nabla_{uu} f W,
\edmath
where $W$ is defined by \req{1.2.7}.

We thus need only supply derivatives
with respect to $u$.  Formally, we must define the function value $f$,
possibly the  gradient
vector $\nabla_u   f$  (i.e., the vector  whose  $i$-th
component is the first partial derivative with  respect  to the $i$-th
internal variable) and,  possibly, the Hessian
matrix $\nabla_{uu} f$
(i.e.,   the matrix whose    $i,j$-th entry is    the  second  partial
derivative with respect to  the $i$-th and $j$-th internal variables),
all evaluated at $u$.   We now describe how  to set up  the data for a
given problem.

\subsection{\label{S3.1.2}Indicator Cards}

As  before, the user must prepare  an input  file, the SEIF  (Standard
Element type   Input Format)
file, consisting  of indicator and  data cards.
The former contain a simple keyword to  specify the type  of data that
follows.  Possible indicator cards
are given in Figure~\ref{F3.1}.

{\small
\bcftable{lllc}
\multicolumn{1}{c}{Keyword} &
\multicolumn{1}{c}{Comments} &
\multicolumn{1}{c}{Presence} &
Described in \S \\
\hline
{\tt ELEMENTS}    &  same as {\tt NAME} & mandatory & \ref{S2.2.1} \\
{\tt TEMPORARIES} &                     & optional  & \ref{S3.2.1} \\
{\tt GLOBALS}     &                     & optional  & \ref{S3.2.2} \\
{\tt INDIVIDUALS} &                     & optional  & \ref{S3.2.3} \\
{\tt ENDATA}      &                     & mandatory & \ref{S2.2.2} \\
\hline
\ecftable{\label{F3.1}Possible indicator cards}
}

Indicator cards must appear  in   the order shown.    The  cards
{\tt TEMPORARIES}, {\tt GLOBALS} and {\tt INDIVIDUALS} are optional.

The data cards are of two kinds. The first are like those described in
Section~\ref{S2.1}. The others use four fields,  fields~1, 2 and 3, as
before, and field~7  which starts  in column 25  and is  41 characters
long.  This  last field is  used to hold  arithmetic  expressions.  An
{\em arithmetic expression} is as  defined in the  Fortran
programming language standard (ANSI  X3.9-1978).  We allow  the use of
any of the   language's intrinsic functions  in   such an  expression.
Continuation  of an expression over at   most nineteen  lines  is also
permitted.

\subsection{\label{S3.1.3}An Example}

Before we give the complete  syntax for an SEIF  file, we continue the
illustrative example that we started in Section~\ref{S2.1.4}  and show
how to  specify  an input   file   appropriate  for   the problem   of
Section~\ref{S1.6}.   Once again, there  are   many  possible ways  of
specifying a particular problem;  we give one  in Figure~\ref{F3.1.3}.
The arithmetic expressions given are written in Fortran.

The file  must always start  with an  {\tt ELEMENTS}
card,
on which a name (in this case {\tt EG3}) for  the example may be
given (line 1), and must end with an {\tt ENDATA}
card (line 40).

We  next need to specify the   names and  attributes of  any auxiliary
quantities  and functions that  we intend  to use   in  our high level
description  of the element  functions.
These are needed to  allow for consistency
checks  in  the subsequent high-level language statements
and must always occur in the  {\tt  TEMPORARIES}
section of the input  file.  Lines~3  to  6 indicate that  we shall be
using temporary quantities {\tt SINV1}, {\tt ZERO}, {\tt ONE} and {\tt
TWOP1}, and the character {\tt  R} in the  first field for  these lines
states that  these quantities will  be  associated with floating point
(real)  values. The character  {\tt M}   in field~1 of  Lines~7  and 8
indicates that we may  use the intrinsic (machine)
functions  {\tt SIN}
and {\tt COS}.  These are   of   course Fortran
intrinsic functions appropriate for the high-level language used here.

We now  specify any numerical  values which are to  be used in one  or
more element
descriptions within  the {\tt GLOBALS}
section.  On lines 10  and 11, we  allocate the values 0  and 1 to the
previously defined quantities {\tt ZERO} and {\tt ONE}. Note that such
cards
require the character {\tt A} in field~1 - if an assignment were
to take more than 41 characters  (the width  of  field~7), it could be
continued  on  subsequent lines  for  which the  string   {\tt  A+} is
required in field~1.

{\renewcommand{\arraystretch}{0.8}
{\footnotesize {\tt
\bcftable{r|@{}c@{}|@{}l@{}|@{}l@{}|@{}l@{}|@{}l@{}|@{}l@{}|@{}l@{}|@{}l@{}|}
\multicolumn{1}{@{}c@{}}{~}&
\multicolumn{1}{@{}c@{}}{~}&
\multicolumn{1}{@{}c@{}}{~}&
\multicolumn{1}{@{}c@{}}{~}&
\multicolumn{1}{@{}c@{}}{~}&
\multicolumn{4}{@{}c@{}}{$<$--------------41----{\rm Field~7}--------------$>$}\\
\multicolumn{1}{@{}c@{}}{~}&
\multicolumn{1}{@{}c@{}}{~}&
\multicolumn{1}{@{}c@{}}{$<$$>$~}&
\multicolumn{1}{@{}c@{}}{$<$---10---$>$}&
\multicolumn{1}{@{}c@{}}{$<$---10---$>$}&
\multicolumn{1}{@{}c@{}}{$<$----12----$>$~~~}&
\multicolumn{1}{@{}c@{}}{$<$---10---$>$}&
\multicolumn{1}{@{}c@{}}{$<$----12----$>$}&
\multicolumn{1}{@{}c@{}}{~~~~}\\
\multicolumn{1}{@{}c@{}}{\sz line}&
\multicolumn{1}{@{}c@{}}{}&
\multicolumn{1}{@{}c@{}}{\rm F.1}&
\multicolumn{1}{@{}c@{}}{\rm Field 2}&
\multicolumn{1}{@{}c@{}}{\rm Field 3}&
\multicolumn{1}{@{}c@{}}{\rm Field 4}&
\multicolumn{1}{@{}c@{}}{\rm Field 5}&
\multicolumn{1}{@{}c@{}}{\rm Field 6}&
\multicolumn{1}{@{}c@{}}{~}\\
\cline{2-9}
 1&\multicolumn{3}{@{}l@{}|}{ELEMENTS}&EG3 & & & & \\
 2&\multicolumn{3}{@{}l@{}|}{TEMPORARIES}& & & & & \\
 3&&R &SINV1 &   &     &   &      & \\
 4&&R &ZERO  &   &     &   &      & \\
 5&&R &ONE   &   &     &   &      & \\
 6&&R &TWOP1 &   &     &   &      & \\
 7&&M &SIN   &   &     &   &      & \\
 8&&M &COS   &   &     &   &      & \\
 9&\multicolumn{3}{@{}l@{}|}{GLOBALS}& & & & & \\
10&&A &ZERO  &   & 0.0 &   &      & \\
11&&A &ONE   &   & 1.0 &   &      & \\
12&\multicolumn{3}{@{}l@{}|}{INDIVIDUALS}& & & & & \\
13&&T &3PROD &   &     &   &      & \\
14&&R &U1    &V1 & 1.0 &V2 & -1.0 & \\
15&&R &U2    &V3 & 1.0 &   &      & \\
16&&F &      &   &\multicolumn{4}{@{}l@{}|}{U1*U2}\\
17&&G &U1    &   &\multicolumn{4}{@{}l@{}|}{U2}   \\
18&&G &U2    &   &\multicolumn{4}{@{}l@{}|}{U1}   \\
19&&H &U1    &U1 &\multicolumn{4}{@{}l@{}|}{ZERO} \\
20&&H &U1    &U2 &\multicolumn{4}{@{}l@{}|}{ONE}  \\
21&&H &U2    &U2 &\multicolumn{4}{@{}l@{}|}{ZERO} \\
22&&T &2PROD &   &     &   &      & \\
23&&F &      &   &\multicolumn{4}{@{}l@{}|}{V1*V2}\\
24&&G &V1    &   &\multicolumn{4}{@{}l@{}|}{V2}   \\
25&&G &V2    &   &\multicolumn{4}{@{}l@{}|}{V1}   \\
26&&H &V1    &V1 &\multicolumn{4}{@{}l@{}|}{ZERO} \\
27&&H &V1    &V2 &\multicolumn{4}{@{}l@{}|}{ONE}  \\
28&&H &V2    &V2 &\multicolumn{4}{@{}l@{}|}{ZERO} \\
29&&T &SINE  &   &     &   &      & \\
30&&A &SINV1 &   &\multicolumn{4}{@{}l@{}|}{SIN(V1)} \\
31&&F &      &   &\multicolumn{4}{@{}l@{}|}{SINV1}   \\
32&&G &V1    &   &\multicolumn{4}{@{}l@{}|}{COS(V1)} \\
33&&H &V1    &V1 &\multicolumn{4}{@{}l@{}|}{-SINV1}  \\
34&&T &SQUARE&   &     &   &      & \\
35&&R &U1    &V1 & 1.0 &V2 & 1.0  & \\
36&&A &TWO   &   &\multicolumn{4}{@{}l@{}|}{2.0}   \\
37&&F &      &   &\multicolumn{4}{@{}l@{}|}{U1*U1} \\
38&&G &U1    &   &\multicolumn{4}{@{}l@{}|}{TWO*U1}\\
39&&H &U1    &U1 &\multicolumn{4}{@{}l@{}|}{TWO}   \\
40&\multicolumn{3}{|@{}l@{}|}{ENDATA}& & & &  & \\
\cline{2-9}
\multicolumn{1}{@{}c@{}}{~}&
\multicolumn{1}{@{}c@{}}{$\uparrow$}&
\multicolumn{1}{@{}c@{}}{$\uparrow \uparrow\;$~}&
\multicolumn{1}{@{}c@{}}{$\uparrow\;$~~~~$\uparrow$~~~$\;\uparrow$}&
\multicolumn{1}{@{}c@{}}{$\uparrow\;$~~~~$\uparrow$~~~$\;\uparrow$}&
\multicolumn{1}{@{}l@{}}{$\uparrow\;$~~~~~~~~~$\;\uparrow$}&
\multicolumn{1}{@{}c@{}}{$\uparrow\;$~~~~~~~$\;\uparrow$}&
\multicolumn{1}{@{}c@{}}{$\uparrow\;$~~~~~~~~~$\;\uparrow$}&
\multicolumn{1}{@{}c@{}}{~~~$\;\uparrow$}\\
\multicolumn{1}{@{}c@{}}{~}&
\multicolumn{1}{@{}c@{}}{\sz 1}&
\multicolumn{1}{@{}c@{}}{{\sz 2~3~}~}&
\multicolumn{1}{@{}c@{}}{{\sz 5~}~~~{\sz 10}~~~{\sz 14}}&
\multicolumn{1}{@{}c@{}}{{\sz 15}~~~{\sz 20}~~~{\sz 24}}&
\multicolumn{1}{@{}l@{}}{{\sz 25}~~~~~~~~~{\sz 36}}&
\multicolumn{1}{@{}c@{}}{{\sz 40}~~~~~~~{\sz 49}}&
\multicolumn{1}{@{}c@{}}{{\sz 50}~~~~~~~~~{\sz 61}}&
\multicolumn{1}{@{}c@{}}{~~~{\sz 65}}\\
\ecftable{\label{F3.1.3}SEIF file for the example of
Section~\protect\ref{S1.6}}
}}}

Finally we need to  make  the actual definitions  of the  function and
derivative
values for the element  types  and specify  the  transformations
from elemental to internal variables
if they are used.  Such specifications
occur in the {\tt INDIVIDUALS}
section from lines 12 to 39 of the example.  We  recall that there are
four  element types
{\tt 3PROD},  {\tt  2PROD},  {\tt  SINE} and  {\tt
SQUARE} and  that  their attributes (names of elemental   and internal
variables and parameters)
have been described  in the SDIF file set up
in  Section~\ref{S2.1.4}.  Two  of the element types ({\tt  3PROD} and
{\tt SQUARE}  ) use internal   variables so  we need to   describe the
relevant transformation for those.

On line 13, the presence of the character {\tt T} in field~1 announces
that the data for the  element type
{\tt  3PROD} is to follow. All the
data for this element must be specified before another element type is
considered. On  lines~14 and 15 we  describe  the  transformation from
elemental to internal  variables that is used  for {\tt 3PROD}. Recall
that the  transformation  is  $u_1  = v_1   - v_2$ and  $u_2= v_3$. On
line~14, the first of these transformations is given, namely that {\tt
U1} is to be formed by adding  1.0 times {\tt  V1} to -1.0  times {\tt
V2}.  The second transformation
is given on the following line, namely
that {\tt U2} is formed by  taking 1.0 times {\tt  V3}. Both lines are
marked as defining transformations by the character {\tt R}
in field~1 ---  continuation lines
are   possible  for transformations
involving more than two elemental variables on lines  in which the
string {\tt R+}
appears in the same field.

We now specify the function and derivative values
of the element type
$u_1 u_2$  with respect to its internal variables.
On line~16, the code
{\tt F}
in field~1 indicates that we are setting the value of the element type
to {\tt U1*U2}, the Fortran
expression for multiplying  {\tt U1}  and
{\tt U2}.  On lines~17 and 18, we specify the first derivatives of the
element type
with  respect to its two  internal variables
{\tt U1} and
{\tt U2} - the character {\tt G}
in field~1 indicates that gradient
values are  to be set.  On line~17,
the derivative
with respect to  the variable  {\tt  U1}, specified in
field~2, is taken and expressed as {\tt U2}  in field~7. Similarly, on
line 18, the  derivative with  respect to  the  variable {\tt U2}  (in
field~2), {\tt U1}, is given in field~7.  Finally, on lines~19  to 21,
the second partial derivatives
with respect to both internal variables
are  given.  These  derivatives appear   on  cards
whose  first  field
contains the character {\tt H}.
On line~19,  the second derivative
with respect  to the variables {\tt
U1} (in field~2) and {\tt U1} (in field~3), 0.0, is given  in field~7.
Similarly the second derivative with respect to the variables {\tt U1}
(in  field~2) and {\tt U2} (in  field~3), 1.0,  occurs  in  field~7 of
line~20 and that with respect  to  {\tt U2} (in  field~2) and {\tt U2}
(in field~3), 0.0, is given in field~7 of the following line.

The same principle   is applied     to  the specification   of   range
transformations,
values   and derivatives
for  the  remaining element
types. The type {\tt 2PROD} does not use a transformation  to internal
variables, so derivatives
are  taken  with respect to the   elemental variables
{\tt V1} and  {\tt V2} (or  one  might think of the internal
variables
 being {\tt V1}   and {\tt  V2},  related to  the  elemental
variables through the   identity  transformation).
The values    and
derivatives for  this element type
are given  on lines~22  to  28. The
type {\tt SINE} again does not use special internal  variables and the
required value and  derivatives are given  on lines~29 to  33.   Note,
however, that  the value and   its second derivative with  respect  to
$v_1$ both use  the quantity $\sin  v_1$; for  efficiency, we set  the
auxiliary quantity {\tt  SINV1} to the Fortran
value  {\tt SIN(V1)} on
line~30  and  thereafter refer  to {\tt SINV1}   on lines~31  and  33.
Notice that this  definition of auxiliary  quantities occurs on a line
whose first field contains  the  character {\tt A}.
Finally,  the type  {\tt  SQUARE},  which  uses a transformation from
elemental  to internal  variables  $u_1 =  v_1  + v_2$,  is defined on
lines~34 to 39. Again notice that the value {2.0} occurs in both first
and second derivatives,
so the auxiliary quantity {\tt TWO} is  set on
line~36 to hold this value.

\subsection{\label{S3.2}Data Cards}

The {\tt ELEMENTS}
and {\tt ENDATA}
indicator cards
perform the same function as the cards {\tt NAME}
and {\tt ENDATA}
in Section~\ref{S2.2.1} and \ref{S2.2.2}.   The problem name
specified in field~3 on the {\tt ELEMENTS}
card  must be the  same as that given
in the same field on the {\tt NAME}
card of the SDIF file.

\subsubsection{\label{S3.2.1}The {\tt TEMPORARIES} Data Card}

When specifying  the function  and  derivative
values of  a  nonlinear element,
it  often happens that an expression  occurs more  than once.
It  is  then convenient to define an  auxiliary  parameter
to have the
value of  the  common  expression and  henceforth   to refer  to   the
auxiliary parameter.  For  instance, a  nonlinear  element  of the two
internal variables $u_1$ and $u_2$ might  be $u_1 e^{u_2}.$ (The names
of the  internal  variables
have  already  been specified in  the {\tt ELEMENT TYPE}
section of the SDIF and are known as {\em  reserved} parameters.)
Its gradient
vector (vector  of first partial derivatives)
has  components
$e^{u_2}$ and $u_1 e^{u_2}$.  If  we define the auxiliary parameter
$w = e^{u_2}$, the derivatives
are then $w$ and $u_1 w$.

{\renewcommand{\arraystretch}{0.8}
{\small {\tt
\bcftable{|@{}c@{}|@{}l@{}|@{}l@{}|@{}l@{}|}
\multicolumn{1}{@{}c@{}}{~~}&
\multicolumn{1}{@{}c@{}}{$<$$>$~}&
\multicolumn{1}{@{}c@{}}{$<$--6-$>$~~~~}&
\multicolumn{1}{@{}c@{}}{~~~~~~~~~~~~~~~~~~~~~~~~~~~~~~~~~~~~~~~~~~~~~~~~~~~}\\
\multicolumn{1}{@{}c@{}}{~}&
\multicolumn{1}{@{}c@{}}{\rm F.1}&
\multicolumn{1}{@{}c@{}}{\rm Field 2}&
\multicolumn{1}{@{}c@{}}{~}\\
\hline
\multicolumn{4}{|@{}l@{}|}{\tt TEMPORARIES} \\
\hline
&I &p-name  & \\
&R &p-name  & \\
&L &p-name  & \\
&M &p-name  &\\
&F &p-name  &\\
\hline
\multicolumn{1}{@{}c@{}}{~}&
\multicolumn{1}{@{}c@{}}{$\uparrow \uparrow\;$~}&
\multicolumn{1}{@{}l@{}}{$\uparrow\;$~~~$\;\uparrow$}&
\multicolumn{1}{@{}c@{}}{~}\\
\multicolumn{1}{@{}c@{}}{~}&
\multicolumn{1}{@{}c@{}}{{\sz 2~3~}~}&
\multicolumn{1}{@{}l@{}}{{\sz 5~}~~~{\sz 10}}&
\multicolumn{1}{@{}c@{}}{~}\\
\ecftable{\label{F3.2.1}Possible data cards for {\tt TEMPORARIES}}
}}}

The {\tt TEMPORARIES}
indicator card
is  used   to announce  the   names of  any  auxiliary
parameters
which    are to be   used in   defining  the function  and
derivative
values of the nonlinear elements. This list should also
include  the name of any intrinsic  and external functions  used.  The
syntax  for  data cards
following    the indicator   card  is given in Figure~\ref{F3.2.1}.

The single-character string in field~1 specifies the type of auxiliary
parameter that is to be defined. Possible types are integer ({\tt I}),
real  ({\tt  R}),
logical  ({\tt L}),
intrinsic  function ({\tt M})
or external function ({\tt  F}).
The string  {\tt p-name} in  field~2 then   gives the name of  the
auxiliary parameter. The name must be a valid Fortran name,
see Section~\ref{S2.1.2}, but must not be a reserved one, i.e., one of
the  names assigned to  the  internal variables
or parameters
for the element
in question in the {\tt ELEMENT TYPE}
section of  the   SDIF  (see, Section~\ref{S2.2.14}).    Any auxiliary
parameter that is to be used must be defined in the {\tt TEMPORARIES}
section along with all intrinsic and external function names.

\subsubsection{\label{S3.2.2}The {\tt GLOBALS}  Data Cards}

The {\tt GLOBALS}
indicator card
is used to announce the assignment of general parameter
values.
the syntax for data cards following the indicator card is given in
Figure~\ref{F3.2.2}.

{\renewcommand{\arraystretch}{0.8}
{\small {\tt
\bcftable{|@{}c@{}|@{}l@{}|@{}l@{}|@{}l@{}|@{}l@{}|}
\multicolumn{1}{@{}c@{}}{~~}&
\multicolumn{1}{@{}c@{}}{$<$$>$~}&
\multicolumn{1}{@{}c@{}}{$<$--6-$>$~~~~}&
\multicolumn{1}{@{}c@{}}{$<$--6-$>$~~~~}&
\multicolumn{1}{@{}c@{}}{$<$------------------41-------------------$>$$\,$}\\
\multicolumn{1}{@{}c@{}}{~}&
\multicolumn{1}{@{}c@{}}{\rm F.1}&
\multicolumn{1}{@{}c@{}}{\rm Field 2}&
\multicolumn{1}{@{}c@{}}{\rm Field 3}&
\multicolumn{1}{@{}c@{}}{\rm Field 7}\\
\hline
\multicolumn{5}{|@{}l@{}|}{\tt GLOBALS} \\
\hline
&A &p-name &        &
\&\&\&\&\&\&\&\&\&\&\&\&\&\&\&\&\&\&\&\&\&\&\&\&\&\&\&\&\&\&\&\&\&\&\&\&\&\&\&\&
 \&\\
&A+&        &        &
\&\&\&\&\&\&\&\&\&\&\&\&\&\&\&\&\&\&\&\&\&\&\&\&\&\&\&\&\&\&\&\&\&\&\&\&\&\&\&\&
 \&\\
&I &l-name &p-name &
\&\&\&\&\&\&\&\&\&\&\&\&\&\&\&\&\&\&\&\&\&\&\&\&\&\&\&\&\&\&\&\&\&\&\&\&\&\&\&\&
 \&\\
&I+&        &        &
\&\&\&\&\&\&\&\&\&\&\&\&\&\&\&\&\&\&\&\&\&\&\&\&\&\&\&\&\&\&\&\&\&\&\&\&\&\&\&\&
 \&\\
&E &l-name &p-name &
\&\&\&\&\&\&\&\&\&\&\&\&\&\&\&\&\&\&\&\&\&\&\&\&\&\&\&\&\&\&\&\&\&\&\&\&\&\&\&\&
 \&\\
&E+&        &        &
\&\&\&\&\&\&\&\&\&\&\&\&\&\&\&\&\&\&\&\&\&\&\&\&\&\&\&\&\&\&\&\&\&\&\&\&\&\&\&\&
 \&\\
\hline
\multicolumn{1}{@{}c@{}}{~}&
\multicolumn{1}{@{}c@{}}{$\uparrow \uparrow\;$~}&
\multicolumn{1}{@{}c@{}}{$\uparrow\;$~~~~$\uparrow$~~~~}&
\multicolumn{1}{@{}c@{}}{$\uparrow\;$~~~~$\uparrow$~~~~}&
\multicolumn{1}{@{}c@{}}{$\uparrow\;$~~~~~~~~~~~~~~~~~~~~~~~~~~~~~~~~~~~~~~$\;\uparrow$}\\
\multicolumn{1}{@{}c@{}}{~}&
\multicolumn{1}{@{}c@{}}{{\sz 2~3~}~}&
\multicolumn{1}{@{}l@{}}{{\sz 5~}~~~~{\sz 10}}&
\multicolumn{1}{@{}l@{}}{{\sz 15}~~~~{\sz 20}}&
\multicolumn{1}{@{}c@{}}{{\sz 25}~~~~~~~~~~~~~~~~~~~~~~~~~~~~~~~~~~~~~~$\,${\sz
 65}}
   \\
\ecftable{\label{F3.2.2}Possible data cards for {\tt GLOBALS}}
}}}
The one or two   character string  in field~1  specifies the  type  of
assignment that is to be  made from  the card.
Possible values for the
first character of the string are:
\begin{description}
\item[\tt A]
This card
announces that an auxiliary  parameter is  to be assigned a
value. The  string {\tt   p-name} in field~2 gives  the  name of the
auxiliary parameter that is  to be defined; this name  must be a valid
Fortran name, see Section~\ref{S2.1.2},  and must have been previously
defined in the {\tt TEMPORARIES}
section.  The string  in  field~7  is an  arithmetic   expression. The
assignment
\bdmath
\mbox{auxiliary variable named in field~2} \leftarrow \mbox{field~7}
\edmath
is made, where  again $\leftarrow$ means ``is  given the  value''; any
variable   mentioned   in the    arithmetic expression must  either be
reserved (see Section~\ref{S3.2.1}), or  have been defined in the {\tt
TEMPORARIES}
section. If in this latter case, the variable  is integer  or real, it
must have been allocated a value itself on a previous {\tt GLOBALS}
data card.

\item[\tt I]
This card
announces that an  auxiliary parameter  is to be assigned  a
value whenever a second logical
auxiliary  parameter has  the value   {\tt  .TRUE.}  The string  {\tt
p-name} in field~3 gives the name of the  auxiliary parameter that is
to be defined; this name must be a valid Fortran name,
see Section~\ref{S2.1.2}, and must have been
previously defined in the {\tt TEMPORARIES}
section.  The string in field~7 is an arithmetic expression. The assignment
\bdmath
\mbox{auxiliary variable named in field~3} \leftarrow \mbox{field~7}
\edmath
will be made  if and only   if the logical
auxiliary parameter {\tt l-name} specified in   field  2  has the
value  {\tt .TRUE.};  the
logical  parameter   must have  been previously   defined  in the {\tt
TEMPORARIES}
section  and  allocated   a value in   the {\tt  GLOBALS}
section.  The arithmetic expression must obey the rules set out in the
{\tt A} section above.

\item[\tt E]
This card
announces that  an auxiliary  parameter is to be assigned  a
value whenever a second logical
auxiliary parameter has the value {\tt
.FALSE.} The string {\tt p-name}  in field~3 gives  the name of the
auxiliary parameter that is to be  defined; this name  must be a valid
Fortran name, see Section~\ref{S2.1.2},  and  must have been previously
defined in the {\tt TEMPORARIES}
section.  The string in field~7 is an
arithmetic expression. The assignment
\bdmath
\mbox{auxiliary variable named in field~3} \leftarrow \mbox{field~7}
\edmath
will be  made if and   only if the  logical auxiliary  parameter,
{\tt l-name}, specified in   field~2 has  the value   {\tt .FALSE.};  the
logical parameter  must  have  been  previously  defined in  the  {\tt
TEMPORARIES}
section   and allocated a  value  in   the {\tt GLOBALS}
section.  The arithmetic expression must obey the rules set out in the
{\tt A} section above.
\end{description}

The  data started on  an  {\tt  A}, {\tt  I}  and {\tt  E}  card
may be
continued on a card whose first field contains an {\tt A+}, {\tt I+} or
{\tt E+} respectively.
Such cards contain an arithmetic expression in
field~7 and no further data;  the  arithmetic expression must obey the
rules   set  out in   the {\tt A}  section   above.  At  most nineteen
continuations of a single assignment are allowed.

The {\tt GLOBALS}
section is intended for the definition  of auxiliary
variables which occur in more than one element  type.
If an auxiliary
variable occurs  in a single element  type, it  may  be defined in the
{\tt INDIVIDUALS}
section (see Section~\ref{S3.2.3}).

\subsubsection{\label{S3.2.3}The {\tt INDIVIDUALS} Data Cards}

The  {\tt  INDIVIDUALS}
indicator   card   is   used  to  announce the
definition  of function  and derivative
values  and the transformation
between elemental
and internal variables  for the  types of nonlinear
element functions required.
The syntax for data  cards following the
indicator card is given in Figure~\ref{F3.2.3}.

{\renewcommand{\arraystretch}{0.8}
{\small {\tt
\bcftable{|@{}c@{}|@{}l@{}|@{}l@{}|@{}l@{}|@{}l@{}|@{}l@{}|@{}l@{}|@{}l@{}|}
\multicolumn{1}{@{}c@{}}{~~}&
\multicolumn{1}{@{}c@{}}{~}&
\multicolumn{1}{@{}c@{}}{~}&
\multicolumn{1}{@{}c@{}}{~}&
\multicolumn{4}{@{}c@{}}{$<$-------------41----{\rm Field~7}--------------$>$}\\
\multicolumn{1}{@{}c@{}}{~}&
\multicolumn{1}{@{}c@{}}{$<$$>$~}&
\multicolumn{1}{@{}c@{}}{$<$---10---$>$}&
\multicolumn{1}{@{}c@{}}{$<$---10---$>$}&
\multicolumn{1}{@{}c@{}}{$<$----12----$>$~~~}&
\multicolumn{1}{@{}c@{}}{$<$---10---$>$}&
\multicolumn{1}{@{}c@{}}{$<$----12----$>$}&
\multicolumn{1}{@{}c@{}}{~~~~}\\
\multicolumn{1}{@{}c@{}}{}&
\multicolumn{1}{@{}c@{}}{\rm F.1}&
\multicolumn{1}{@{}c@{}}{\rm Field 2}&
\multicolumn{1}{@{}c@{}}{\rm Field 3}&
\multicolumn{1}{@{}c@{}}{\rm Field 4}&
\multicolumn{1}{@{}c@{}}{\rm Field 5}&
\multicolumn{1}{@{}c@{}}{\rm Field 6}&
\multicolumn{1}{@{}c@{}}{~}\\
\hline
\multicolumn{8}{|@{}l@{}|}{\tt INDIVIDUALS} \\
\hline
&T &etype-name &          &              &         &              & \\
&R &iv-nam     &ev-nam   &numerical-vl &ev-nam  &numerical-vl & \\
&A &p-name     &          &
\multicolumn{4}{@{}l@{}|}{\&\&\&\&\&\&\&\&\&\&\&\&\&\&\&\&\&\&\&\&\&\&\&\&\&\&\&
 \&\&\&\&\&\&\&\&\&\&\&\&\&\&}\\
&A+&            &          &
\multicolumn{4}{@{}l@{}|}{\&\&\&\&\&\&\&\&\&\&\&\&\&\&\&\&\&\&\&\&\&\&\&\&\&\&\&
 \&\&\&\&\&\&\&\&\&\&\&\&\&\&}\\
&I &l-name     &p-name   &
\multicolumn{4}{@{}l@{}|}{\&\&\&\&\&\&\&\&\&\&\&\&\&\&\&\&\&\&\&\&\&\&\&\&\&\&\&
 \&\&\&\&\&\&\&\&\&\&\&\&\&\&}\\
&I+&            &          &
\multicolumn{4}{@{}l@{}|}{\&\&\&\&\&\&\&\&\&\&\&\&\&\&\&\&\&\&\&\&\&\&\&\&\&\&\&
 \&\&\&\&\&\&\&\&\&\&\&\&\&\&}\\
&E &l-name     &p-name   &
\multicolumn{4}{@{}l@{}|}{\&\&\&\&\&\&\&\&\&\&\&\&\&\&\&\&\&\&\&\&\&\&\&\&\&\&\&
 \&\&\&\&\&\&\&\&\&\&\&\&\&\&}\\
&E+&            &          &
\multicolumn{4}{@{}l@{}|}{\&\&\&\&\&\&\&\&\&\&\&\&\&\&\&\&\&\&\&\&\&\&\&\&\&\&\&
 \&\&\&\&\&\&\&\&\&\&\&\&\&\&}\\
&F &            &          &
\multicolumn{4}{@{}l@{}|}{\&\&\&\&\&\&\&\&\&\&\&\&\&\&\&\&\&\&\&\&\&\&\&\&\&\&\&
 \&\&\&\&\&\&\&\&\&\&\&\&\&\&}\\
&F+&            &          &
\multicolumn{4}{@{}l@{}|}{\&\&\&\&\&\&\&\&\&\&\&\&\&\&\&\&\&\&\&\&\&\&\&\&\&\&\&
 \&\&\&\&\&\&\&\&\&\&\&\&\&\&}\\
&G &iv-nam     &          &
\multicolumn{4}{@{}l@{}|}{\&\&\&\&\&\&\&\&\&\&\&\&\&\&\&\&\&\&\&\&\&\&\&\&\&\&\&
 \&\&\&\&\&\&\&\&\&\&\&\&\&\&}\\
&G+&            &          &
\multicolumn{4}{@{}l@{}|}{\&\&\&\&\&\&\&\&\&\&\&\&\&\&\&\&\&\&\&\&\&\&\&\&\&\&\&
 \&\&\&\&\&\&\&\&\&\&\&\&\&\&}\\
&H &iv-nam     &iv-nam   &
\multicolumn{4}{@{}l@{}|}{\&\&\&\&\&\&\&\&\&\&\&\&\&\&\&\&\&\&\&\&\&\&\&\&\&\&\&
 \&\&\&\&\&\&\&\&\&\&\&\&\&\&}\\
&H+& & &
\multicolumn{4}{@{}l@{}|}{\&\&\&\&\&\&\&\&\&\&\&\&\&\&\&\&\&\&\&\&\&\&\&\&\&\&\&
 \&\&\&\&\&\&\&\&\&\&\&\&\&\&}\\
\hline
\multicolumn{1}{@{}c@{}}{~}&
\multicolumn{1}{@{}c@{}}{$\uparrow \uparrow\;$~}&
\multicolumn{1}{@{}c@{}}{$\uparrow\;$~~~~$\uparrow$~~$\;\uparrow$}&
\multicolumn{1}{@{}c@{}}{$\uparrow\;$~~~~$\uparrow$~~$\;\uparrow$}&
\multicolumn{1}{@{}l@{}}{$\uparrow\;$~~~~~~~~~$\;\uparrow$}&
\multicolumn{1}{@{}c@{}}{$\;\uparrow\;$~~~~~~~$\;\uparrow$}&
\multicolumn{1}{@{}c@{}}{$\uparrow\;$~~~~~~~~~$\;\uparrow$}&
\multicolumn{1}{@{}c@{}}{~~$\;\uparrow$}\\
\multicolumn{1}{@{}c@{}}{~}&
\multicolumn{1}{@{}c@{}}{{\sz 2~3~}~}&
\multicolumn{1}{@{}c@{}}{{\sz 5~}~~~~{\sz 10}~~{\sz 14}}&
\multicolumn{1}{@{}c@{}}{{\sz 15}~~~~{\sz 20}~~{\sz 24}}&
\multicolumn{1}{@{}l@{}}{{\sz 25}~~~~~~~~~{\sz 36}}&
\multicolumn{1}{@{}c@{}}{$\;${\sz 40}~~~~~~~{\sz 49}}&
\multicolumn{1}{@{}c@{}}{{\sz 50}~~~~~~~~~$\,${\sz 61}}&
\multicolumn{1}{@{}c@{}}{~~$\,\,${\sz 65}}\\
\ecftable{\label{F3.2.3}Possible data cards for {\tt INDIVIDUALS}}
}}}

The one- or two-character string in field~1 specifies the type  of data
contained on the card.
Possible values for the first character  of the string are:
\begin{description}
\itt{ T}
This card
announces that a new element type
is to  be considered. The
string {\tt  etype-name} in field~2  gives the name of  the element
type;  the name may be  up to  ten characters long  and must have been
defined in   the {\tt ELEMENT TYPE}   section  of the SDIF  file  (see
Section~\ref{S2.2.14}).
\itt{ R}
This card   announces that information  concerning the  transformation
between the elemental and  internal variables for  the element type is
to be  given. Such information  is appropriate  only for element types
which have  been  defined with internal  variables in the {\tt ELEMENT
TYPE}
section  of   the  SDIF  file   (see   Section~\ref{S2.2.14}).     The
transformation is specified by  the matrix $W$  of Section~\ref{S1.3};
only nonzero coefficients of $W$ need be specified here.

The  string {\tt  inv-name}  in  field~2  contains the  name of  an
internal variable  (i.e., row  of   $W$).   The name  must be  a valid
Fortran name,
see Section~\ref{S2.1.2},  and have been defined on  an {\tt IV}
data line in the {\tt ELEMENT TYPE}
section of the SDIF file.  The  strings {\tt iv-nam} in fields~3 and
(optionally) 5  then  give  the names of   elemental  variables (i.e.,
columns of $W$).  The names must be valid Fortran  names and have been
defined on {\tt EV}
data  lines in  the  {\tt ELEMENT TYPE}
section of the SDIF  file. The strings in  fields~4 and (optionally) 6
contain the numerical values of the  coefficients of $W$ corresponding
to the row given in  field~2 and the columns  given  in fields~3 and 5
respectively.  These numerical values may each be  up to 12 characters
long.  The entries of $W$ may be defined in any order.

As an example, the  transformation \req{1.2.5} could be   entered with
three {\tt R}
data cards.
On the first,  field~2  would hold the name
given to   the internal variable  $u_1$; field~3   would hold the name
given to  the elemental variable $v_1$  and field~4 would contain 1.0.
Similarly field~5 would hold the name given  to the elemental variable
$v_2$ and  field~6 would also contain 1.0.    On   the second, field~2
would also   hold the name  given  to  the internal variable   $u_ 1$;
field~3 would now hold the name given  to the elemental variable $v_3$
and field~4 would contain -2.0.  On the third card, field~2 would hold
the name given to the internal  variable $u_2$; field~3 would hold the
name given to the elemental  variable $v_1$ and field~4  would contain
1.0. Field~5 would now hold  the name  given to the elemental variable
$v_3$ and field~6 would contain -1.0.

\itt{A}
This  card
announces that  an  auxiliary  parameter, specific   to the
current element type,
is to be assigned  a  value.   The string {\tt
p-name} in field~2 gives the name of the auxiliary parameter that is
to  be  defined;   this name   must  be  a valid Fortran   name,   see
Section~\ref{S2.1.2},  and have been  previously  defined in the  {\tt
TEMPORARIES}
section.    The  string in  field~7  is   an  arithmetic
expression.  The assignment
\bdmath
\mbox{auxiliary variable named in field~2} \leftarrow \mbox{field~7}
\edmath
is made,  where again $\leftarrow$  means ``is  given the value''; any
variable  mentioned  in   the arithmetic  expression  must  either  be
reserved (see Section~\ref{S3.2.1}), or  have been defined in the {\tt
TEMPORARIES}
section. If in this latter  case, the variable is integer
or real,  it  must have  been allocated a   value itself either   on a
previous {\tt GLOBALS}
data card
or on a previous {\tt A}, {\tt  E} or
{\tt I}  card  for the   current element type
in the   {\tt ELEMENTS}
section.

\itt{I}
This card   announces that an   auxiliary parameter,  specific  to the
current element type,
is  to be  assigned  a  value whenever  a second
logical auxiliary
parameter has  the  value {\tt  .TRUE.} The  string,
{\tt p-name}, in field~3 gives  the name of  the  auxiliary parameter
that is to  be defined; this  name must be  a valid Fortran  name,
see Section~\ref{S2.1.2}, and have been previously defined in the {\tt
TEMPORARIES}
section.  The string  in field~7  is  an  arithmetic expression.   The
assignment
\bdmath
\mbox{auxiliary variable  named in field~3} \leftarrow \mbox{field~7}
\edmath
will  be made if  and only  if the logical  auxiliary parameter,
{\tt l-name}, specified in field~2 has the value {\tt .TRUE.};  the logical
parameter must have been previously defined in the {\tt TEMPORARIES}
section   and  allocated  a  value  in the   {\tt  GLOBALS}
or  {\tt INDIVIDUALS}
section.  The arithmetic expression must obey the rules set out in the
{\tt A}
section above.

\itt{E}
This card
announces  that  an auxiliary  parameter,  specific to  the
current element type,
is  to be  assigned a  value  whenever a  second
logical  auxiliary
parameter has the value  {\tt .FALSE.} The  string,
{\tt p-name},  in field~3 gives  the name of  the auxiliary parameter
that is to be defined; this  name must  be  a  valid Fortran name, see
Section~\ref{S2.1.2}, and have been  previously defined  in  the  {\tt
TEMPORARIES}
section.   The  string   in  field~7 is   an arithmetic
expression.  The assignment
\bdmath
\mbox{auxiliary variable named in field~3} \leftarrow \mbox{field~7}
\edmath
will be made  if and only if  the   logical auxiliary parameter,
{\tt l-name}, specified in field~2 has the value {\tt .FALSE.}; the logical
parameter must have  been previously defined  in the {\tt TEMPORARIES}
section   and   allocated  a value  in   the   {\tt   GLOBALS}
or {\tt INDIVIDUALS}
section.  The arithmetic expression must obey the rules set out in the
{\tt A}
section above.

\itt{F}
This card
specifies the value of the nonlinear element.
The string in field~7 is an arithmetic expression; the assignment
\bdmath
\mbox{nonlinear element function} \leftarrow \mbox{field~7}
\edmath
is made; any variable mentioned in the  expression must obey the rules
set out in the {\tt A}
section above.

\itt{G}
This card
specifies the value of  a component of the  gradient
of  the  nonlinear element.  The  string,   {\tt  iv-nam}, in  field~2
contains  the name of  an  internal  variable.   The  component of the
gradient specified  on  the card
will  be taken  with respect  to this
variable.  The string must be a valid Fortran name,
see  Section~\ref{S2.1.2}, and have been  defined on  an {\tt IV}
data line, for a nonlinear element
defined with internal variables,
or an {\tt EV}
data  line,  for  an   element  without explicit   internal
variables,  in the {\tt  ELEMENT TYPE}
section of  the SDIF file.  The  string in  field~7  is  an arithmetic
expression; the assignment
\bdmath
\mbox{derivative of element w.r.t. variable in field~2} \leftarrow
\mbox{field~7}
\edmath
is made; any variable mentioned in the arithmetic expression must obey
the rules set out in the {\tt A}
section  above.
{\tt G}
cards
are  optional.  However, once
the  user starts to   form the  gradient for an
element type,
any component  not explicitly  specified will be assumed
to have the value zero.

\itt{H}
This card
specifies the value of a  component of the Hessian matrix of
the nonlinear element.  The  strings {\tt iv-nam} in fields~2  and 3
contain the names of internal variables.  The component of the Hessian
specified on the  card will be taken with  respect to these variables.
Either string must be a valid Fortran  name, see Section~\ref{S2.1.2},
and have been  defined  on an  {\tt  IV}
data  line, for  a  nonlinear element
defined with internal variables,
or an {\tt EV}
data line, for
an element without explicit  internal variables, in  the  {\tt ELEMENT
TYPE}
section of the  SDIF  file.   The string in  field~7  is an arithmetic
expression; the assignment
\bdmath
\mbox{second derivative  of element w.r.t. variables in fields~2 and 3}
\leftarrow \mbox{field~7}
\edmath
is made; any variable mentioned in the arithmetic expression must obey
the rules set  out in the  {\tt A}
section  above.   {\tt H}
cards
are  optional.  However, once  the  user starts to  specify  the
Hessian matrix  for an element type,  any component not specified will
be assumed to have  the value  zero.  The   matrix  is  assumed  to be
symmetric and so the user needs only supply values for one of
\bdmath
\frac{\partial^2f}{\partial u_i \partial u_j}
\tim{or}
\frac{\partial^2f}{\partial u_j \partial u_i}
\ms (i \neq j)
\edmath
it does not matter which.  Observe  that defaulting Hessian components
to zero gives a very simple way of  inputing sparse matrices;
however,  as  we stressed in  the  introduction,   we do not generally
recommend this method of specifying invariant subspaces.
\end{description}

The data started on an {\tt A}, {\tt I}, {\tt E}, {\tt F},  {\tt G} and
{\tt H} card
may be  continued on a card whose  first field contains an
{\tt A+},  {\tt I+},  {\tt  E+},   {\tt  F+}, {\tt G+}  or  {\tt   H+}
respectively.
Such cards  contain an arithmetic expression in field~7
and no further data; the arithmetic expression must obey the rules set
out in the {\tt A} section above.  At most nineteen continuations of a
single assignment are allowed.

The data for a single element type
must occur on consecutive cards and
in the order given  in Figure~\ref{F3.2.3},  excepting that  {\tt  A},
{\tt  I} and {\tt  E}
cards
may be intermixed.  A new element type
is deemed to have started  whenever a {\tt T}
card is  encountered.  The {\tt F}
card is  compulsory for  all   element   types;
elements  with  useful
transformations from elemental to internal  variables must  also  have
{\tt R}
cards.
The data for a particular card type is considered to have been
completed whenever another card type is encountered.

\subsection{\label{S3.3}Two Further Examples}

In Section~\ref{S1.4},   we gave an example. An   SEIF   file for this
example is given  in Figure~|ref{F3.3.1}.  The  problem is again given
the name {\tt DOC}.  The two types of nonlinear  element were assigned
the names {\tt ELEMENT1/2}  by the previous SDIF  file.  The elemental
variables were  given names beginning  with {\tt V}  and the  internal
variables for the second nonlinear element
started with {\tt U}.  The constant 0.0 occurs in the derivatives
of  both elements,   so  an
auxiliary variable is assigned  to hold its  value (although, we could
have just not specified these particular components,  which would then
have taken  their  default  zero  value).  The function    value   and
derivatives of the second element type
use  both sines and  cosines of
$u_2$ and again auxiliary variables are assigned to hold these values,
this  time  as    variables local  to  {\tt  ELEMENT2}.    The  second
derivatives  are  sufficiently   straightforward   to  compute that we
provide them.

{\renewcommand{\arraystretch}{0.8}
{\small {\tt
\bcftable{|@{}c@{}|@{}l@{}|@{}l@{}|@{}l@{}|@{}l@{}|@{}l@{}|@{}l@{}|@{}l@{}|}
\multicolumn{1}{@{}c@{}}{~~}&
\multicolumn{1}{@{}c@{}}{~}&
\multicolumn{1}{@{}c@{}}{~}&
\multicolumn{1}{@{}c@{}}{~}&
\multicolumn{4}{@{}c@{}}{$<$--------------41----{\rm Field~7}------------$>$}\\
\multicolumn{1}{@{}c@{}}{~}&
\multicolumn{1}{@{}c@{}}{$<$$>$~}&
\multicolumn{1}{@{}c@{}}{$<$---10---$>$}&
\multicolumn{1}{@{}c@{}}{$<$---10---$>$}&
\multicolumn{1}{@{}c@{}}{$<$----12----$>$~~~}&
\multicolumn{1}{@{}c@{}}{$<$---10---$>$}&
\multicolumn{1}{@{}c@{}}{$<$----12----$>$}&
\multicolumn{1}{@{}c@{}}{~~~~}\\
\multicolumn{1}{@{}c@{}}{}&
\multicolumn{1}{@{}c@{}}{\rm F.1}&
\multicolumn{1}{@{}c@{}}{\rm Field 2}&
\multicolumn{1}{@{}c@{}}{\rm Field 3}&
\multicolumn{1}{@{}c@{}}{\rm Field 4}&
\multicolumn{1}{@{}c@{}}{\rm Field 5}&
\multicolumn{1}{@{}c@{}}{\rm Field 6}&
\multicolumn{1}{@{}c@{}}{~}\\
\hline
\multicolumn{3}{|@{}l@{}|}{ELEMENTS}   &DOC& & & & \\
\multicolumn{3}{|@{}l@{}|}{TEMPORARIES}&   & & & & \\
&R &CS      &   &       &   &      & \\
&R &SN      &   &       &   &      & \\
&R &ZERO    &   &       &   &      & \\
&R &SIN     &   &       &   &      & \\
&R &COS     &   &       &   &      & \\
\multicolumn{3}{|@{}l@{}|}{GLOBALS}    &   & & & & \\
&G &ZERO    &   &\multicolumn{4}{@{}l@{}|}{0.0D0}\\
\multicolumn{3}{|@{}l@{}|}{INDIVIDUALS}&   & & & & \\
&T &ETYPE1  &   &       &   &      & \\
&F &        &   &\multicolumn{4}{@{}l@{}|}{V1*V2}\\
&G &V1      &   &\multicolumn{4}{@{}l@{}|}{V2}\\
&G &V2      &   &\multicolumn{4}{@{}l@{}|}{V1}\\
&H &V1      &V1 &\multicolumn{4}{@{}l@{}|}{ZERO}\\
&H &V1      &V2 &\multicolumn{4}{@{}l@{}|}{1.0D0}\\
&H &V2      &V2 &\multicolumn{4}{@{}l@{}|}{ZERO}\\
&T &ETYPE2  &   &       &   &      & \\
&R &U1      &V1 & 1.0D0 &   &      & \\
&R &U2      &V2 & 1.0D0 &V3 &1.0D0 & \\
&A &CS      &   &\multicolumn{4}{@{}l@{}|}{COS(U2)}\\
&A &SN      &   &\multicolumn{4}{@{}l@{}|}{SIN(U2)}\\
&F &        &   &\multicolumn{4}{@{}l@{}|}{U1*SN}\\
&G &U1      &   &\multicolumn{4}{@{}l@{}|}{SN}\\
&G &U2      &   &\multicolumn{4}{@{}l@{}|}{U1*CS}\\
&H &U1      &U1 &\multicolumn{4}{@{}l@{}|}{ZERO}\\
&H &U1      &U2 &\multicolumn{4}{@{}l@{}|}{CS}\\
&H &U2      &U2 &\multicolumn{4}{@{}l@{}|}{-U1*SN}\\
\multicolumn{3}{|@{}l@{}|}{ENDATA}     &   & & & & \\
\hline
\multicolumn{1}{@{}c@{}}{~}&
\multicolumn{1}{@{}c@{}}{$\uparrow \uparrow\;$~}&
\multicolumn{1}{@{}c@{}}{$\uparrow\;$~~~~$\uparrow$~~$\;\uparrow$}&
\multicolumn{1}{@{}c@{}}{$\uparrow\;$~~~~$\uparrow$~~$\;\uparrow$}&
\multicolumn{1}{@{}l@{}}{$\uparrow\;$~~~~~~~~~$\;\uparrow$}&
\multicolumn{1}{@{}c@{}}{$\uparrow\;$~~~~~~~$\;\uparrow$}&
\multicolumn{1}{@{}c@{}}{$\uparrow\;$~~~~~~~~~$\;\uparrow$}&
\multicolumn{1}{@{}c@{}}{~~$\;\uparrow$}\\
\multicolumn{1}{@{}c@{}}{~}&
\multicolumn{1}{@{}c@{}}{{\sz 2~3~}~}&
\multicolumn{1}{@{}c@{}}{{\sz 5~}~~~~{\sz 10}~~{\sz 14}}&
\multicolumn{1}{@{}c@{}}{{\sz 15}~~~~{\sz 20}~~{\sz 24}}&
\multicolumn{1}{@{}l@{}}{{\sz 25}~~~~~~~~~{\sz 36}}&
\multicolumn{1}{@{}c@{}}{{\sz 40}~~~~~~~{\sz 49}}&
\multicolumn{1}{@{}c@{}}{{\sz 50}~~~~~~~~~{\sz 61}}&
\multicolumn{1}{@{}c@{}}{~~{\sz 65}}\\
\ecftable{\label{F3.3.1}SEIF file for the element types for the example
\protect\\ of Section~\protect\ref{S1.4}}
}}}

{\renewcommand{\arraystretch}{0.8}
{\small {\tt
\bcftable{|@{}c@{}|@{}l@{}|@{}l@{}|@{}l@{}|@{}l@{}|@{}l@{}|@{}l@{}|@{}l@{}|}
\multicolumn{1}{@{}c@{}}{~~}&
\multicolumn{1}{@{}c@{}}{~}&
\multicolumn{1}{@{}c@{}}{~}&
\multicolumn{1}{@{}c@{}}{~}&
\multicolumn{4}{@{}c@{}}{$<$--------------41----{\rm Field~7}------------$>$}\\
\multicolumn{1}{@{}c@{}}{~}&
\multicolumn{1}{@{}c@{}}{$<$$>$~}&
\multicolumn{1}{@{}c@{}}{$<$---10---$>$}&
\multicolumn{1}{@{}c@{}}{$<$---10---$>$}&
\multicolumn{1}{@{}c@{}}{$<$----12----$>$~~~}&
\multicolumn{1}{@{}c@{}}{$<$---10---$>$}&
\multicolumn{1}{@{}c@{}}{$<$----12----$>$}&
\multicolumn{1}{@{}c@{}}{~~~~}\\
\multicolumn{1}{@{}c@{}}{}&
\multicolumn{1}{@{}c@{}}{\rm F.1}&
\multicolumn{1}{@{}c@{}}{\rm Field 2}&
\multicolumn{1}{@{}c@{}}{\rm Field 3}&
\multicolumn{1}{@{}c@{}}{\rm Field 4}&
\multicolumn{1}{@{}c@{}}{\rm Field 5}&
\multicolumn{1}{@{}c@{}}{\rm Field 6}&
\multicolumn{1}{@{}c@{}}{~}\\
\hline
\multicolumn{3}{|@{}l@{}|}{ELEMENTS}   &DOC2& & & & \\
\multicolumn{3}{|@{}l@{}|}{INDIVIDUALS}&    & & & & \\
&T &SQUARE  &   &       &   &      & \\
&F &        &   &\multicolumn{4}{@{}l@{}|}{V**2}\\
&G &V       &   &\multicolumn{4}{@{}l@{}|}{2.0D0*V}\\
&H &V       &V  &\multicolumn{4}{@{}l@{}|}{2.0D0}\\
\multicolumn{3}{|@{}l@{}|}{ENDATA}     &   & & & & \\
\hline
\multicolumn{1}{@{}c@{}}{~}&
\multicolumn{1}{@{}c@{}}{$\uparrow \uparrow\;$~}&
\multicolumn{1}{@{}c@{}}{$\uparrow\;$~~~~$\uparrow$~~$\;\uparrow$}&
\multicolumn{1}{@{}c@{}}{$\uparrow\;$~~~~$\uparrow$~~$\;\uparrow$}&
\multicolumn{1}{@{}l@{}}{$\uparrow\;$~~~~~~~~~$\;\uparrow$}&
\multicolumn{1}{@{}c@{}}{$\uparrow\;$~~~~~~~$\;\uparrow$}&
\multicolumn{1}{@{}c@{}}{$\uparrow\;$~~~~~~~~~$\;\uparrow$}&
\multicolumn{1}{@{}c@{}}{~~$\;\uparrow$}\\
\multicolumn{1}{@{}c@{}}{~}&
\multicolumn{1}{@{}c@{}}{{\sz 2~3~}~}&
\multicolumn{1}{@{}c@{}}{{\sz 5~}~~~{\sz 10}~~~{\sz 14}}&
\multicolumn{1}{@{}c@{}}{{\sz 15}~~~{\sz 20}~~~{\sz 24}}&
\multicolumn{1}{@{}l@{}}{{\sz 25}~~~~~~~~~{\sz 36}}&
\multicolumn{1}{@{}c@{}}{{\sz 40}~~~~~~~{\sz 49}}&
\multicolumn{1}{@{}c@{}}{{\sz 50}~~~~~~~~~{\sz 61}}&
\multicolumn{1}{@{}c@{}}{~~{\sz 65}}\\
\ecftable{\label{F3.3.2}SEIF file for the element types for the example
\protect\\ of Section~\protect\ref{S1.5}}
}}}

We gave a second example in Section~\ref{S1.5}. An SEIF file for this example
is given in Figure~\ref{F3.3.2} on page \pageref{F3.3.2}.
The problem is again given the name {\tt DOC2}.
The only type of nonlinear element
was assigned  the name {\tt SQUARE}  in the previous  SDIF   file, its
elemental variable
was called  {\tt V}  and there was no  useful range
transformation.



\section[The SIF for Nontrivial Groups]{\label{S4}The Standard Input Format
for \protect\\ Nontrivial Groups}
\setcounter{figure}{0}


In  addition  to the  problem  data and  the  nonlinear  element types
described in Section~\ref{S2} and \ref{S3}, the user  might  also wish
to specify the nontrivial group functions, and their derivatives,
in  a  systematic  way.  A particular  nontrivial group   function  is
defined  in  terms  of its  group  type
and   variable;
both  of    these    quantities  are  specified  in
Section~\ref{S2}. Thus, the only details  which remain to be specified
are the function and derivative values of the group types.

Once again, we present an approach to this issue.   As before, data is
specified  in a  file.  The   file  comprises an   ordered mixture  of
indicator and  data cards;
the  latter  allow function  and derivative
definitions in appropriate high-level language statements.

\subsection[Introduction to the Standard Group Type \protect\\ Input
 Format]{\label{S4.1}Introduction to the Standard Group Type \protect\\ Input
 Format}

\subsubsection{\label{S4.1.1}The Values and Derivatives Required}

It is assumed that a nonlinear group type i
s specified in terms of its
group-type variable
as described on a {\tt GROUP TYPE}
data card in an SDIF file,  see Section~\ref{S2.2.16}. An optimization
procedure is likely to require  the  values of the group functions
and their  first  and   second  derivatives
(taken  with  respect   to the
variable).    We now  describe how  to  set up   the data for  a given
problem.

\subsubsection{\label{S4.1.2}Indicator Cards}

As  before, the  user must prepare  an  input file, the SGIF (Standard
Group type Input Format)
file, consisting of indicator  and data cards.
The former  contain a simple keyword
to  specify the type of data that
follows.  Possible indicator cards are given in Figure~\ref{F4.1}.

{\small
\bcftable{lllc}
\multicolumn{1}{c}{Keyword} &
\multicolumn{1}{c}{Comments} &
\multicolumn{1}{c}{Presence} &
Described in \S \\
\hline
{\tt GROUPS}      &  same as {\tt NAME} & mandatory & \ref{S2.2.1} \\
{\tt TEMPORARIES} &                     & optional  & \ref{S3.2.1} \\
{\tt GLOBALS}     &                     & optional  & \ref{S3.2.2} \\
{\tt INDIVIDUALS} &                     & optional  & \ref{S4.2.1} \\
{\tt ENDATA}      &                     & mandatory & \ref{S2.2.2} \\
\hline
\ecftable{\label{F4.1}Possible indicator cards}
}

Indicator   cards
must  appear in  the  order shown.  The  cards  {\tt
TEMPORARIES}, {\tt GLOBALS} and {\tt INDIVIDUALS}
are optional.

The data cards
are of a single kind, using four fields, fields~1, 2, 3
and 7, exactly as described in Section~\ref{S3.1.2}.

\subsubsection{\label{S4.1.3} An Example}

Before we give  the complete syntax for  an  SGIF  file, we finish the
illustrative    example that we   started in  Section~\ref{S2.1.4} and
Section~\ref{S3.1.3} and show how to specify an input file appropriate
for the  problem of Section~\ref{S1.6}.  The  format is fairly similar
to that for the SEIF file of Section~\ref{S3}.  Once again,  there are
many possible ways of specifying a particular problem;  we give one in
Figure~\ref{F4.1.3}.

{\renewcommand{\arraystretch}{0.8}
{\footnotesize {\tt
\bcftable{r|@{}c@{}|@{}l@{}|@{}l@{}|@{}l@{}|@{}l@{}|}
\multicolumn{1}{@{}c@{}}{~~}&
\multicolumn{1}{@{}c@{}}{~}&
\multicolumn{1}{@{}c@{}}{$<$$>$~}&
\multicolumn{1}{@{}c@{}}{$<$---10---$>$}&
\multicolumn{1}{@{}c@{}}{$<$---10---$>$}&
\multicolumn{1}{@{}c@{}}{$<$-------------------41------------------$>$}\\
\multicolumn{1}{@{}c@{}}{\sz line}&
\multicolumn{1}{@{}c@{}}{~}&
\multicolumn{1}{@{}c@{}}{\rm F.1}&
\multicolumn{1}{@{}c@{}}{\rm Field 2}&
\multicolumn{1}{@{}c@{}}{\rm Field 3}&
\multicolumn{1}{@{}c@{}}{\rm Field 7}\\
\cline{2-6}
1&\multicolumn{3}{|@{}l@{}|}{GROUPS}&EG3          &       \\
2&\multicolumn{3}{|@{}l@{}|}{TEMPORARIES}  & &  \\
3&&R &TWOP1 &  & \\
4&\multicolumn{3}{|@{}l@{}|}{INDIVIDUALS} & & \\
5&&T  &PSQUARE & & \\
6&&A  &TWOP1  &  &2.0*P1 \\
7&&F  &  &  &P1*ALPHA*ALPHA \\
8&&G & &
&TWOP1*ALPHA      \\
9&&H       &          &    &TWOP1       \\
10&\multicolumn{3}{|@{}l@{}|}{ENDATA} & & \\
\cline{2-6}
\multicolumn{1}{@{}c@{}}{~}&
\multicolumn{1}{@{}c@{}}{$\uparrow$}&
\multicolumn{1}{@{}c@{}}{$\uparrow \uparrow\;$~}&
\multicolumn{1}{@{}c@{}}{$\uparrow\;$~~~~$\uparrow$~~$\;\uparrow$}&
\multicolumn{1}{@{}c@{}}{$\uparrow\;$~~~~$\uparrow$~~$\;\uparrow$}&
\multicolumn{1}{@{}c@{}}{$\uparrow\;$~~~~~~~~~~~~~~~~~~~~~~~~~~~~~~~~~~~~~$\;\uparrow$}\\
\multicolumn{1}{@{}c@{}}{~}&
\multicolumn{1}{@{}c@{}}{\sz 1}&
\multicolumn{1}{@{}c@{}}{{\sz 2~3~}~}&
\multicolumn{1}{@{}c@{}}{{\sz 5~}~~~~{\sz 10}~~{\sz 14}}&
\multicolumn{1}{@{}c@{}}{{\sz 15}~~~~{\sz 20}~~{\sz 24}}&
\multicolumn{1}{@{}c@{}}{{\sz 25}~~~~~~~~~~~~~~~~~~~~~~~~~~~~~~~~~~~~~{\sz
 65}}\\
\ecftable{\label{F4.1.3}SGIF file for the element types for the example
\protect\\ of Section~\protect\ref{S1.6}}
}}}

The file must always start with a {\tt GROUPS}
card,
on  which a name
(in this case {\tt  EG3}) for  the example may  be given (line 1), and
must end with an {\tt ENDATA}
card (line 10).

We  next need to specify the   names and  attributes of  any auxiliary
quantities  and functions that  we intend  to use   in  our high level
description  of the group  functions.
These are  needed to   allow for consistency checks
in the subsequent high-level  language statements
and must always occur  in the {\tt  TEMPORARIES}
section of the input  file.  Line~3 indicates  that we shall  be using
temporary quantities {\tt TWOP1} and the character {\tt R}
in the first field  of this lines  states  that the  quantity  will be
associated with a floating point (real) value.

We now make the actual definitions of the function and derivative
values for the nontrivial  group type
used; we recall  that there is a
single nontrivial group type   {\tt  PSQUARE} and that its  attributes
(name of group-type variable
and parameter) have been  described in the
SDIF file set up in Section~\ref{S2.1.4}.  This definition takes place
within the {\tt INDIVIDUALS}
section.  The presence of the character {\tt T}
in field~1  of line~5 announces  that the data for the  group
type {\tt PSQUARE} is to follow. All  the  data for this group must be
specified before another group type
is considered.   We note  that the
quantity $2 p_1$ occurs  in both  first  and second derivatives
of the
group type function  and so the  auxiliary quantity {\tt TWOP1} is set
on line~6 to hold this value.  The first field of a line on which such
an assignment  is  made  contains the character   {\tt A}.
The  value  (line~7),   its  first  derivative  (line~8)    and second
derivative (line~9) with  respect to  the group-type
variable are now given.  A Fortran
expression for  these values  occurs in field~7 on
each of these lines; the lines contain the  characters {\tt F}, {\tt G}
and {\tt H}
respectively in field~1 for such assignments.

If there had  been more than  a single group  type
with  one or  more expressions  in common, these  expressions
could have been assigned to
previously   attributed quantities in  a {\tt   GLOBALS}
section.  This  section   would then  have  appeared between the  {\tt
TEMPORARIES}
and {\tt INDIVIDUALS}
sections.

\subsection{\label{S4.2} Data Cards}

The {\tt  GROUPS}
and {\tt  ENDATA}
indicator  cards
perform the  same
function  as   the cards    {\tt   NAME}
 and  {\tt      ENDATA}
in     Section~\ref{S2.2.1}  and   \ref{S2.2.2}   Likewise,  the  {\tt
TEMPORARIES}
and {\tt GLOBALS}
data cards
have exactly the same syntax
as those in  Section~\ref{S3.2.1} and \ref{S3.2.2}, excepting that the
reserved parameters are now the group-type
variables specified in the
{\tt GROUP TYPE}
section of the SDIF file.

\subsubsection{\label{S4.2.1} The {\tt INDIVIDUALS} Data Cards}

The {\tt  INDIVIDUALS}
indicator card   is used to announce  the  definition of  function and
derivative
values for   the   types of
nontrivial   group  functions
required. The  syntax  for  data cards
following the indicator card is given in Figure~\ref{S4.2.1}.

{\renewcommand{\arraystretch}{0.8}
{\small {\tt
\bcftable{|@{}c@{}|@{}l@{}|@{}l@{}|@{}l@{}|@{}l@{}|}
\multicolumn{1}{@{}c@{}}{~~}&
\multicolumn{1}{@{}c@{}}{$<$$>$~}&
\multicolumn{1}{@{}c@{}}{$<$---10---$>$}&
\multicolumn{1}{@{}c@{}}{$<$---10---$>$}&
\multicolumn{1}{@{}c@{}}{$<$-------------------41------------------$>$}\\
\multicolumn{1}{@{}c@{}}{~}&
\multicolumn{1}{@{}c@{}}{\rm F.1}&
\multicolumn{1}{@{}c@{}}{\rm Field 2}&
\multicolumn{1}{@{}c@{}}{\rm Field 3}&
\multicolumn{1}{@{}c@{}}{\rm Field 7}\\
\hline
&T &gtype-name&        & \\
&A &p-name     &
 &\&\&\&\&\&\&\&\&\&\&\&\&\&\&\&\&\&\&\&\&\&\&\&\&\&\&\&\&\&\&\&\&\&\&\&\&\&\&\&
 \&\&\\
&A+&           &
 &\&\&\&\&\&\&\&\&\&\&\&\&\&\&\&\&\&\&\&\&\&\&\&\&\&\&\&\&\&\&\&\&\&\&\&\&\&\&\&
 \&\&\\
&I &l-name     &p-name
 &\&\&\&\&\&\&\&\&\&\&\&\&\&\&\&\&\&\&\&\&\&\&\&\&\&\&\&\&\&\&\&\&\&\&\&\&\&\&\&
 \&\&\\
&I+&           &
 &\&\&\&\&\&\&\&\&\&\&\&\&\&\&\&\&\&\&\&\&\&\&\&\&\&\&\&\&\&\&\&\&\&\&\&\&\&\&\&
 \&\&\\
&E &l-name     &p-name
 &\&\&\&\&\&\&\&\&\&\&\&\&\&\&\&\&\&\&\&\&\&\&\&\&\&\&\&\&\&\&\&\&\&\&\&\&\&\&\&
 \&\&\\
&E+&           &
 &\&\&\&\&\&\&\&\&\&\&\&\&\&\&\&\&\&\&\&\&\&\&\&\&\&\&\&\&\&\&\&\&\&\&\&\&\&\&\&
 \&\&\\
&F &           &
 &\&\&\&\&\&\&\&\&\&\&\&\&\&\&\&\&\&\&\&\&\&\&\&\&\&\&\&\&\&\&\&\&\&\&\&\&\&\&\&
 \&\&\\
&F+&           &
 &\&\&\&\&\&\&\&\&\&\&\&\&\&\&\&\&\&\&\&\&\&\&\&\&\&\&\&\&\&\&\&\&\&\&\&\&\&\&\&
 \&\&\\
&G &           &
 &\&\&\&\&\&\&\&\&\&\&\&\&\&\&\&\&\&\&\&\&\&\&\&\&\&\&\&\&\&\&\&\&\&\&\&\&\&\&\&
 \&\&\\
&G+&           &
 &\&\&\&\&\&\&\&\&\&\&\&\&\&\&\&\&\&\&\&\&\&\&\&\&\&\&\&\&\&\&\&\&\&\&\&\&\&\&\&
 \&\&\\
&H &           &
 &\&\&\&\&\&\&\&\&\&\&\&\&\&\&\&\&\&\&\&\&\&\&\&\&\&\&\&\&\&\&\&\&\&\&\&\&\&\&\&
 \&\&\\
&H+&           &
 &\&\&\&\&\&\&\&\&\&\&\&\&\&\&\&\&\&\&\&\&\&\&\&\&\&\&\&\&\&\&\&\&\&\&\&\&\&\&\&
 \&\&\\
\hline
\multicolumn{1}{@{}c@{}}{~}&
\multicolumn{1}{@{}c@{}}{$\uparrow \uparrow\;$~}&
\multicolumn{1}{@{}c@{}}{$\uparrow\;$~~~~$\uparrow$~~$\;\uparrow$}&
\multicolumn{1}{@{}c@{}}{$\uparrow\;$~~~~$\uparrow$~~$\;\uparrow$}&
\multicolumn{1}{@{}c@{}}{$\uparrow$~~~~~~~~~~~~~~~~~~~~~~~~~~~~~~~~~~~~~
~$\uparrow$}\\
\multicolumn{1}{@{}c@{}}{~}&
\multicolumn{1}{@{}c@{}}{{\sz 2~3~}~}&
\multicolumn{1}{@{}c@{}}{{\sz 5~}~~~~{\sz 10}~~{\sz 14}}&
\multicolumn{1}{@{}c@{}}{{\sz 15}~~~~{\sz 20}~~{\sz 24}}&
\multicolumn{1}{@{}c@{}}{{\sz 25}~~~~~~~~~~~~~~~~~~~~~~~~~~~~~~~~~~~~~~{\sz
 65}$\,$}
   \\
\ecftable{\label{F4.2.1}Possible data cards for {\tt INDIVIDUALS}}
}}}

The one- or two-character string in field~1 specifies  the type of data
contained on the card.
Possible values for the  first character of the string are:
\begin{description}
\itt{T}
This card announces that  a new group  type
is  to be considered.  The
string {\tt gtype-name} in field~2 gives the name of the group type;
the name may be up  to ten characters  long and must have been defined
in  the  {\tt   GROUP  TYPE}
section    of   the SDIF    file   (see Section~\ref{S2.2.15}).

\itt{A}
This card  announces that an  auxiliary   parameter, specific  to  the
current  group type,
is   to be assigned   a  value.  The string {\tt
p-name} in field~2 gives the name of the auxiliary parameter that is
to  be  defined;  this  name  must be  a  valid  Fortran   name,
see Section~\ref{S2.1.2}, and have been previously defined in the {\tt
TEMPORARIES}
section.   The   string in field~7   is an  arithmetic expression. The
assignment
\bdmath
\mbox{auxiliary variable named in field~2} \leftarrow \mbox{field~7}
\edmath
is  made; any variable  mentioned in  the   arithmetic expression must
either be reserved (see  Section~\ref{S4.2}), or  have been defined in
the {\tt TEMPORARIES}
section. If, in this  latter case,  the variable
is integer or real, it must have been allocated  a value itself either
on a previous {\tt GLOBALS}
data card
or on a previous {\tt A}
card  for the current  element type
in the {\tt INDIVIDUALS}
section.

\itt{I}
This  card announces  that  an  auxiliary parameter,   specific to the
current  group type,
is to be  assigned  a value whenever  a  second
logical  auxiliary parameter has the value   {\tt  .TRUE.} The string,
{\tt p-name}, in field~3 gives the  name  of the  auxiliary parameter
that is to be defined; this  name  must be a  valid Fortran  name,
see
Section~\ref{S2.1.2}, and  have  been previously  defined in  the {\tt
TEMPORARIES}
section.   The string in  field~7   is  an arithmetic expression.  The
assignment
\bdmath
\mbox{auxiliary variable named in field~3} \leftarrow \mbox{field~7}
\edmath
will  be made  if and only if the  logical auxiliary  parameter,
{\tt l-name}, specified in field 2 has the value  {\tt .TRUE.}; the logical
parameter must have been previously  defined in  the {\tt TEMPORARIES}
section   and   allocated  a value  in    the  {\tt  GLOBALS}
or {\tt INDIVIDUALS}
section.  The  arithmetic expression must  obey the rules
set out in the {\tt A}
section above.

\itt{E}
This  card
announces  that an auxiliary   parameter,  specific to the
current  group type,
is to be  assigned a  value   whenever   a second
logical auxiliary  parameter
has the  value  {\tt .FALSE.} The string,
{\tt p-name}, in field~3 gives  the name  of the auxiliary  parameter
that is  to be defined;  this name  must be a  valid Fortran name,
see
Section~\ref{S2.1.2}, and  have  been previously  defined  in the {\tt
TEMPORARIES}
section.  The string  in   field~7  is an arithmetic   expression. The
assignment
\bdmath
\mbox{auxiliary variable named in field~3} \leftarrow \mbox{field~7}
\edmath
will  be made if  and only if  the logical auxiliary  parameter,
{\tt l-name}, specified in field 2 has the value {\tt .FALSE.}; the logical
parameter
must have been  previously defined in the {\tt TEMPORARIES}
section  and allocated    a  value in  the   {\tt   GLOBALS}
or {\tt INDIVIDUALS}
section.  The arithmetic expression must obey the rules set out in the
{\tt A}
section above.

\itt{F}
This card
specifies the value of the nontrivial group.
The string in
field~7 is an arithmetic expression; the assignment
\bdmath
\mbox{nontrivial group function} \leftarrow \mbox{field~7}
\edmath
is made; any variable mentioned in  the expression must obey the rules
set out in the {\tt A}
section above.

\itt{G}
This card
specifies the value of the first derivative
of  the  nonlinear group  function  with  respect  to  its  group-type
variable.
The string  in  field~7  is  an arithmetic expression;  the
assignment
\bdmath
\mbox{first derivative of group function} \leftarrow \mbox{field~7}
\edmath
is made; any variable mentioned in the arithmetic expression must obey
the rules set out in the {\tt A}
section above.

\itt{H}
This card
specifies   the value of   the second derivative of the  the
nonlinear group function with respect to its group-type variable.
The
string in field~7 is an arithmetic expression; the assignment
\bdmath
\mbox{second derivative of group function} \leftarrow \mbox{field~7}
\edmath
is made; any variable mentioned in the arithmetic expression must obey
the rules  set  out  in the {\tt A}
section  above.
\end{description}

The data started on an {\tt A}, {\tt I}, {\tt E}, {\tt F}, {\tt  G} and
{\tt H} card
may be continued on a card  whose first field  contains an
{\tt A+}, {\tt   I+}, {\tt  E+},  {\tt  F+}, {\tt    G+} or {\tt   H+}
respectively.
Such cards
contain an arithmetic expression in  field~7 and no further
data; the arithmetic expression  must obey the  rules set out  in  the
{\tt A} section  above.  At most nineteen continuations  of  a  single
assignment are allowed.

The data for a single group type
must occur  on consecutive  cards and
in the order given in Figure~\ref{F4.2.1}.  A new group type is deemed
to have started whenever a {\tt  T}
card is  encountered.  The {\tt F}
card
is compulsory for all group types.

\subsection{\label{S4.3} Two Further Examples}

In  Section~\ref{S1.4},  we gave  an  example. An  SGIF file for  this
example is given in Figure~\ref{F4.3.1}.

{\renewcommand{\arraystretch}{0.8}
{\small {\tt
\bcftable{|@{}c@{}|@{}l@{}|@{}l@{}|@{}l@{}|@{}l@{}|}
\multicolumn{1}{@{}c@{}}{~~}&
\multicolumn{1}{@{}c@{}}{$<$$>$~}&
\multicolumn{1}{@{}c@{}}{$<$---10---$>$}&
\multicolumn{1}{@{}c@{}}{$<$---10---$>$}&
\multicolumn{1}{@{}c@{}}{$<$-------------------41------------------$>$}\\
\multicolumn{1}{@{}c@{}}{~}&
\multicolumn{1}{@{}c@{}}{\rm F.1}&
\multicolumn{1}{@{}c@{}}{\rm Field 2}&
\multicolumn{1}{@{}c@{}}{\rm Field 3}&
\multicolumn{1}{@{}c@{}}{\rm Field 7}\\
\hline
\multicolumn{3}{|@{}l@{}|}{GROUPS}      &DOC  & \\
\multicolumn{3}{|@{}l@{}|}{TEMPORARIES} &     & \\
&R &ALPHA2     &        & \\
&R &TWO        &        & \\
\multicolumn{3}{|@{}l@{}|}{INDIVIDUALS} &     & \\
&T &GTYPE1     &        & \\
&A &TWO        &        &2.0D+0      \\
&F &           &        &ALPHA*ALPHA \\
&G &           &        &TWO*ALPHA   \\
&H &           &        &TWO         \\
&T &GTYPE2     &        & \\
&A &ALPHA2     &        &ALPHA*ALPHA      \\
&F &           &        &ALPHA2*ALPHA2    \\
&G &           &        &4.0*ALPHA2*ALPHA \\
&H &           &        &12.0*ALPHA2      \\
\multicolumn{3}{|@{}l@{}|}{ENDATA} &     & \\
\hline
\multicolumn{1}{@{}c@{}}{~}&
\multicolumn{1}{@{}c@{}}{$\uparrow \uparrow\;$~}&
\multicolumn{1}{@{}c@{}}{$\uparrow\;$~~~~$\uparrow$~~$\;\uparrow$}&
\multicolumn{1}{@{}c@{}}{$\uparrow\;$~~~~$\uparrow$~~~$\uparrow$}&
\multicolumn{1}{@{}c@{}}{$\uparrow\;$~~~~~~~~~~~~~~~~~~~~~~~~~~~~~~~~~~~~~~$\;\uparrow$}\\
\multicolumn{1}{@{}c@{}}{~}&
\multicolumn{1}{@{}c@{}}{{\sz 2~3~}~}&
\multicolumn{1}{@{}c@{}}{{\sz 5~}~~~~{\sz 10}~~{\sz 14}}&
\multicolumn{1}{@{}c@{}}{{\sz 15}~~~~{\sz 20}~~{\sz 24}}&
\multicolumn{1}{@{}c@{}}{{\sz 25}~~~~~~~~~~~~~~~~~~~~~~~~~~~~~~~~~~~~~~{\sz
 65}}\\
\ecftable{\label{F4.3.1}SGIF file for the nontrivial group types
for the example \protect\\ of Section~\protect\ref{S1.4}}
}}}

The  problem  is again  given the  name {\tt  DOC}.  The  two types of
nontrivial groups  were assigned   the names {\tt GTYPE1/2}    by  the
previous  SDIF file, each with group-type  variables
{\tt ALPHA}.  The function and derivatives
values of the second group type, $g(\alpha) =  \alpha^4$, all  use some
product of $\alpha^2$,  so an auxiliary  variable is  assigned to hold
this value, the variable being local to the group type.  Likewise, the
derivatives of the first group type,
$g(\alpha) = \alpha^2$ both use some product of 2.0,  so another auxiliary variable
is assigned to hold its value.

We gave a second example in Section~\ref{S1.5}. An  SGIF file for this
example is given in  Figure~\ref{F4.3.2} on page \pageref{F4.3.2}.  The
problem is again  given
the name {\tt DOC2}.  The single  nontrivial group type  was given the
name {\tt  SINE} by the  previous   SDIF  file, with   the  group-type
variable {\tt  ALPHA} and the  single parameter {\tt P}.  The function
and  second derivatives both  depend on the  product  of the
parameter
with the sine of the group type variable,
so  an auxiliary variable is assigned to hold this value.

{\renewcommand{\arraystretch}{0.8}
{\small {\tt
\bcftable{|@{}c@{}|@{}l@{}|@{}l@{}|@{}l@{}|@{}l@{}|}
\multicolumn{1}{@{}c@{}}{~~}&
\multicolumn{1}{@{}c@{}}{$<$$>$~}&
\multicolumn{1}{@{}c@{}}{$<$---10---$>$}&
\multicolumn{1}{@{}c@{}}{$<$---10---$>$}&
\multicolumn{1}{@{}c@{}}{$<$-------------------41------------------$>$}\\
\multicolumn{1}{@{}c@{}}{~}&
\multicolumn{1}{@{}c@{}}{\rm F.1}&
\multicolumn{1}{@{}c@{}}{\rm Field 2}&
\multicolumn{1}{@{}c@{}}{\rm Field 3}&
\multicolumn{1}{@{}c@{}}{\rm Field 7}\\
\hline
\multicolumn{3}{|@{}l@{}|}{GROUPS}      &DOC2 & \\
\multicolumn{3}{|@{}l@{}|}{TEMPORARIES} &     & \\
&R &ISINA      &        & \\
&M &SIN        &        & \\
&M &COS        &        & \\
\multicolumn{3}{|@{}l@{}|}{INDIVIDUALS} &     & \\
&T &SINE       &        & \\
&A &ISINA      &        &P*SIN(ALPHA)\\
&F &           &        &ISINA       \\
&G &           &        &P*COS(ALPHA)\\
&H &           &        &-ISINA      \\
\multicolumn{3}{|@{}l@{}|}{ENDATA} &     & \\
\hline
\multicolumn{1}{@{}c@{}}{~}&
\multicolumn{1}{@{}c@{}}{$\uparrow \uparrow\;$~}&
\multicolumn{1}{@{}c@{}}{$\uparrow\;$~~~~$\uparrow$~~$\;\uparrow$}&
\multicolumn{1}{@{}c@{}}{$\uparrow\;$~~~~$\uparrow$~~~$\uparrow$}&
\multicolumn{1}{@{}c@{}}{$\uparrow\;$~~~~~~~~~~~~~~~~~~~~~~~~~~~~~~~~~~~~~~$\;\uparrow$}\\
\multicolumn{1}{@{}c@{}}{~}&
\multicolumn{1}{@{}c@{}}{{\sz 2~3~}~}&
\multicolumn{1}{@{}c@{}}{{\sz 5~}~~~~{\sz 10}~~{\sz 14}}&
\multicolumn{1}{@{}c@{}}{{\sz 15}~~~~{\sz 20}~~{\sz 24}}&
\multicolumn{1}{@{}c@{}}{{\sz 25}~~~~~~~~~~~~~~~~~~~~~~~~~~~~~~~~~~~~~~{\sz
 65}}\\
\ecftable{\label{F4.3.2}SGIF file for the nontrivial group type
for the example \protect\\ of Section~\protect\ref{S1.5}}
}}}

\section{\label{S5} Free Form Input}
\setcounter{figure}{0}

So far, we have been  quite  specific in the  format that we allow. In
this section,  we  consider  a  second  format   which, though closely
connected to the first, allows one  to input problems  in a less rigid
fashion. Although we refer to this second format as {\em free format},
the freedom really lies in how the  data can be  laid  out in an input
file, not in any extra enhancements to the content of a file.

The input style discussed  in Sections~\ref{S2}--\ref{S4} is  known as
fixed format.
Each  SDIF/SEIF/SGIF file  is  assumed to be in  fixed  format  unless
otherwise   specified.   A  fixed  format  file  has data  arranged in
specified fields
of given length and normally does not allow for  much data on a single
card.
A free form
file,   on the  other hand,  is  one  where  considerable data may  be
conveyed  on   a single  line.   The  data   does not   have to lie in
prespecified fields.
However, we  shall   insist  that  {\em any   free  form file  can  be
translated to fixed format  and interpreted correctly, in this format,
in a single sequential pass through the file}.

We allow a further pair of indicator cards
in any SDIF/SEIF/SGIF file.
These   cards,     like   those  described     in  Section~\ref{S2.1},
Section~\ref{S3.1.2} and Section~\ref{S4.1.2},     contain  a   single
keyword   starting in   column~1.    The new keywords   are given   in
Figure~\ref{F5.1}.

\bcftable{ll}
\multicolumn{1}{c}{Keyword} & \multicolumn{1}{c}{Presence}\\
\hline
{\tt FREE FORMAT} & optional \\
{\tt FIXED FORMAT} & optional \\
\ecftable{\label{F5.1}Additional indicator cards}

Any data that lies between a {\tt FREE} {\tt FORMAT}
card  and  the  next {\tt  FIXED} {\tt FORMAT}
or  {\tt  ENDATA}
card is considered to be in free format.
Likewise, any data that lies between a {\tt FIXED FORMAT} card
and the next {\tt FREE FORMAT}
or
{\tt  ENDATA}
card is considered  to be in fixed format.
The file is considered to be in fixed format when the {\tt NAME}
(SDIF file), {\tt ELEMENTS}
(SEIF  file)  or {\tt  GROUPS}
(SGIF file) card
is first encountered and  thus no initial {\tt FIXED
FORMAT}
card is required.

Fixed    format      data     is   exactly     as         described in
Sections~\ref{S2}--\ref{S4}.   The  data  on a free   format data card
consists of  a number of {\em  strings} separated by {\em separators}.
The characters  ``{\tt   -}'',  ``{\tt ;}'', ``{\tt   \$}'' and  ``~''
(blank)
are separators and should not therefore be used as significant
characters  within strings.  For example,  in  free format, {\tt X1;2}
will be interpreted   as two  strings {\tt X1}    and   {\tt 2}.   The
separators have the following meanings:
\begin{description}
\itt{~}
(blank) indicates that the previous string has finished and that a new
string will follow. One or  more blanks  is interpreted   as a  single
blank.

\itt{;}
indicates that the previous string has finished and  that a new string
will follow. Moreover, if the  file is translated  into  fixed format,
the new string will appear on a new card.

\itt{\_}
indicates that the  previous string  has finished  and  that the  next
string is empty.  Each {\tt  -}  indicates a separate empty string so
that {\tt \_\_\_} indicates three empty strings.

\itt{\$}
indicates that the previous string has finished and that the remainder
of the card is  to be considered as  a comment (and thus  ignored when
the file is interpreted).
\end{description}

A free format
card
may contain  up to 160  characters.  On translation  into  fixed
format,
a free format card will be divided into one or more fixed format
cards depending on  how many  card  separators ``{\tt ;}''
are encountered.  Each fixed format card may  hold up  to six strings;
these strings are numbered 1 to 6.

String~1  is examined to  see if the first 12  characters identify the
card as an indicator. If so, these characters are  placed in columns~1
to 12 on the card and the remaining  strings discarded. Otherwise, the
card is a data card and the first two-characters of string~1, together
with the most recently identified indicator card are used to determine
the structure
of the remainder of  the card;
two  character  code
must   occur   as    field~1   in  the  indicated section    of
Sections~\ref{S2}--\ref{S4} of this  report.  The first~2, 10, 10,  12
(41 on some SEIF/SGIF cards), 10, and  12 characters of  strings 1--6,
respectively, are extracted and placed on a single data  card starting
in columns~2, 5, 15, 25, 40, and 50, respectively.  Left-over parts of
strings are discarded. The assembled card is now in fixed format
and may be interpreted as such.  Thus although  a free format card may
appear to allow more flexibility, the  requirement that the translated
card
conforms  to  the  fixed  input   format
places considerable responsibility on  the user to specify the content
of strings correctly.

As an example, a free format
variant of the SDIF file given in Figure~\ref{F2.3.1} might be:
{\renewcommand{\arraystretch}{0.8} {\small {\tt \begin{verbatim}
NAME          DOC
FREE FORMAT
GROUPS;E GROUP1;E GROUP2;E GROUP3
VARIABLES;_X1 GROUP1 1.0;_X2 GROUP3 1.0;_X3
BOUNDS;FR BN1 X1;LO BN1 X2 -1.0;LO BN1 X3 1.0
       UP BN1 X2 1.0;UP BN1 X3 2.0
ELEMENT TYPE
EV ETYPE1 V1;EV ETYPE1 V2
EV ETYPE2 V1;EV ETYPE2 V2;EV ETYPE2 V3
IV ETYPE2 U1;IV ETYPE2 U2
ELEMENT USES
T G2E1 ETYPE1;V G2E1 V1_X2;V G2E1 V2_X3
T G3E1 ETYPE2;V G3E1 V1_X2;V G3E1 V2_X1;V G3E1 V3_X3
T G3E2 ETYPE1;V G2E1 V1_X1;V G2E1 V2_X3
GROUP TYPE;GV GTYPE1 ALPHA;GV GTYPE2 ALPHA
GROUP USES
T GROUP1 GTYPE1; GROUP2 GTYPE2
E GROUP2 G2E1;E GROUP3 G3E1;E GROUP3 G2E2
ENDATA
\end{verbatim} }}}

\section{\label{S6}Other Standards and Proposals}
\setcounter{figure}{0}

There have been a number of  other proposed  standards for input.  The
most popular approaches use a high-level modelling language
to specify problems. Typical examples are GAMS
\cite{GAMS88},  AMPL
\cite{FourGayKern87}
and OMP \cite{OMP87}.   Such  approaches are  useful  for  specifying
repetitious structures,
but do not really attempt  to cope with useful
nonlinear structure (like invariant subspaces).
Recent work \cite{FourGayKern89}  hopes   to overcome  this
disadvantage.

We have  recently become aware  of other suggestions for  the input of
large-scale   structured
problems. These  proposals   are based upon
representing nonlinear functions in  their factorable \cite{Lena89} or
functional forms \cite{McCoRahn89}.  Such forms   are the the  logical
extensions of \req{objective}    in which  a function  is   decomposed
completely into basic building blocks.  The  advantage of such schemes
is the  potential for the  automatic  calculation of  derivatives,
but this must be weighed against the difficulty  of describing how the
building  blocks are assembled.    We  await further details  of these
interesting proposals.

\section{\label{S7}Conclusions}
\setcounter{figure}{0}

We  have made   a  proposal for  a    standard  input format  for  the
specification of (large-scale) nonlinear programming problems.  In its
full generality,  the user  needs  to provide  three input files.  The
first describes the structure of
the problem  and the decomposition of
the problem  into  group
and element  functions.
The second  and third then specify  the values  and derivatives
of these functions. It is anticipated that the first file will be used
to provide input parameters for a user's optimization procedure, while
the remaining   two  will  be  used   to generate problem   evaluation
subprograms.

\newpage

%\input{/nimg/sp/refs/refs}
\begin{thebibliography}{10}

\bibitem{GAMS88}
A.~Brooke, D.~Kendrick, and A.~Meeraus.
\newblock {\em {GAMS}: a User's Guide}.
\newblock The Scientific Press, Redwood City, USA., 1988.

\bibitem{IBM69}
International Business~Machine Corporation.
\newblock Mathematical programming system/360 version 2, linear and separable
  programming-user's manual.
\newblock Technical Report H20-0476-2, IBM Corporation, 1969.
\newblock MPS Standard.

\bibitem{Dant63}
G.~B. Dantzig.
\newblock {\em Linear Programming and Extensions}.
\newblock Princeton Uiversity Press, Princeton, USA, 1963.

\bibitem{OMP87}
D.~Decker, F.~Louveaux andf G.~Mortier, G.~Schepens, and A.~V. Looveren.
\newblock {\em Linear and Mixed Integer Programming with OMP}.
\newblock Beyers and Partners, Brasschaat, Belgium, 1987.

\bibitem{DuffErisReid86}
I.~S. Duff, A.~M. Erisman, and J.~K. Reid.
\newblock {\em Direct Methods for Sparse Matrices}.
\newblock Oxford University Press, Oxford, England, 1986.

\bibitem{DuffGrimLewi89}
I.~S. Duff, Roger~G. Grimes, and John~G. Lewis.
\newblock Sparse matrix test problems.
\newblock {\em ACM Transactions on Mathematical Software}, 15(1):1--14, 1989.

\bibitem{FourGayKern87}
R.~Fourer, D.~M. Gay, and B.~W. Kernighan.
\newblock {AMPL}: A mathematical programming language.
\newblock Computer science technical report, AT\&T Bell Laboratories, Murray
  Hill, USA, 1987.

\bibitem{FourGayKern89}
R.~Fourer, D.~M. Gay, and B.~W. Kernighan.
\newblock A high-level language would make a good standard form for nonlinear
  programming problems.
\newblock talk at the CORS/TIMS/ORSA Meeting, Vanvouver, 1989.

\bibitem{Gay85}
D.~M. Gay.
\newblock Electronic mail distribution of linear programming test problems.
\newblock Mathematical Programming Society COAL Newsletter, December 1985.

\bibitem{GillMurrWrig81}
P.~E. Gill, W.~Murray, and M.~H. Wright.
\newblock {\em Practical Optimization}.
\newblock Academic Press, London, 1981.

\bibitem{GrieToin82a}
A.~Griewank and Ph.~L. Toint.
\newblock On the unconstrained optimization of partially separable functions.
\newblock In M.~J.~D. Powell, editor, {\em Nonlinear Optimization 1981}, pages
  301--312, London, 1982. Academic Press.

\bibitem{OSLQP:1998}
{IBM Optimization Solutions and Library}.
\newblock {\em QP Solutions User Guide}.
\newblock IBM Corportation, 1998.

\bibitem{Lena89}
M.~Lenard.
\newblock Standardizing the interface with nonlinear optimizers.
\newblock talk at the CORS/TIMS/ORSA Meeting, Vancouver, 1989.

\bibitem{MaroMesa:1999}
I.~Maros and C.~Meszaros.
\newblock A repository of convex quadratic programming problems.
\newblock {\em Optimization Methods and Software}, 11-12:671--681, 1999.

\bibitem{McCoRahn89}
G.~P. McCormick and P.~Rahnavard.
\newblock Representation of unconstrained optimization.
\newblock talk at the CORS/TIMS/ORSA Meeting, Vancouver, 1989.

\bibitem{Ponc:1990}
D.~B. Poncele\'{o}n.
\newblock {\em Barrier methods for large-scale quadratic programming}.
\newblock PhD thesis, Department of Computer Science, Stanford University,
  Stanford, California, USA, 1990.

\bibitem{DictComp83}
Oxford~University Press.
\newblock {\em Dictionary of Computing}.
\newblock Oxford University Press, Oxford, 1983.

\end{thebibliography}

\end{document}
